\section{Recursos Energéticos y la Producción de Electricidad.}

\subsection{Introducción.}
	\begin{itemize}
		\item \textit{\textbf{Recursos energéticos:}}
			principalmente son combustibles fósiles y nuestra sociedad se ha
			hecho extraordinariamente dependiente de ellos para su desarrollo.
		
		\item \textit{\textbf{Combustibles fósiles:}}
			comprenden principalmente el petróleo y sus derivados (gasolinas,
			gasóleos, gases licuados del petróleo, etc.), el gas natural, el carbón mineral y el uranio.\linebreak
			Al principio de la explotación de estos recursos, se consideraban ilimitados y su impacto
			ambiental era despreciable.
			
		\item \textit{\textbf{Consumo masivo de hidrocarburos:}} está produciendo alteraciones en la atmósfera a nivel mundial
		por la emisión de gases de efecto invernadero, originando un calentamiento global y un cambio
		climático.
		
		\item \textit{\textbf{Esquema de consumo energético actual:}} no es sustentable por:
		\begin{itemize}
			\item \textit{Razones económicas:} próxima escasez de hidrocarburos.
			\item \textit{Razones ecológicas:} alteración de la atmósfera y el suelo.
		\end{itemize}
		Es imperativo el \textbf{desarrollo de nuevas alternativas energéticas} más eficientes y menos agresivas contra el medio ambiente:
		\begin{itemize}
			\item Incremento de las fuentes de energía renovable, consideradas como inagotables.
			\item Resurgimiento de la energía nuclear.
		\end{itemize}
	\end{itemize}
	
		
\subsection{Fuentes de energía no renovable.}
	\begin{itemize}
		\item \textbf{\textit{Definición:}} 
			energía que está almacenada en cantidades fijas, comúnmente en
			el subsuelo. A medida que se consume un recurso no renovable, se va agotando.
		\item \textbf{\textit{Reservas:}}
			sujetas a la factibilidad técnica y económica de su explotación, al
			descubrimiento de nuevos yacimientos y al ritmo de extracción y consumo.
	\end{itemize}
	
		\subsubsection{Fuentes de energía fósil.}
			Se llama energía fósil la que se obtiene de la combustión (oxidación) de ciertas substancias que,
			según la geología, se produjeron en el subsuelo a partir de la acumulación de grandes cantidades de
			residuos de seres vivos, hace millones de años. Son los siguientes recursos:
			\begin{itemize}
				\item \textit{Petróleo y derivados:}
					el petróleo es una mezcla de una gran variedad de hidrocarburos
					(compuestos de carbono e hidrógeno) en fase líquida, mezclados con una gran variedad de
					impurezas. Por destilación y otros procesos, se obtienen las diversas gasolinas, el diésel, etc.
				\item \textit{Gas natural:} 
					está compuesto principalmente por metano y corresponde a la
					fracción más ligera de los hidrocarburos, por lo que se encuentra en los yacimientos en forma
					gaseosa.
				\item \textit{Carbón mineral:}
					es principalmente carbono, también de origen fósil, que
					se encuentra en grandes yacimientos en el subsuelo.
			\end{itemize}
			
		\subsubsection{Fuentes de energía nuclear.}
			Son aquellas que provienen de la desintegración de átomos mediante fisión o la fusión de isótopos para producir energía.
			\begin{itemize}
				\item \textit{Uranio:}
					elemento radiactivo natural. Se encuentra en la naturaleza en casi
					todas las rocas y suelos.
				\item \textit{Torio:}
					tiene usos muy similares al Uranio.
				\item \textit{Energía de fusión nuclear:} 
					pretende imitar el comportamiento de las reacciones que se producen en el Sol. Fusionando un isótopo de Deuterio con otro de Tritio se obtiene un átomo de Helio, un neutrón y energía.
			\end{itemize}
		
	\subsection{Fuentes de energía renovable.}
		Se llama energía renovable la que puede explotarse ilimitadamente, es decir, su cantidad disponible (en
		la Tierra) no disminuye a medida que se aprovecha. La principal fuente de energía renovable es el Sol:
		\begin{itemize}
			\item
				Es una esfera gaseosa, cuyos componentes principales son el hidrógeno, el helio y el carbono.
				Su masa es 330.000 veces la de la Tierra.
			\item
				Se comporta como una perfecta central nuclear y en su seno se desarrollan reacciones termonucleares
				de fusión de núcleos de hidrógeno en helio.
			\item
				En su núcleo, fusiona 620 millones de toneladas métricas ($620\cdot10^9$ kg) de hidrógeno por segundo: $4 H\longrightarrow1 He$ + \textit{Energía.}
		\end{itemize}
		
		\indent Algunos datos sobre el efecto del Sol sobre la Tierra:
		\begin{itemize}
			\item
				Como la masa del Sol es del orden de $2\cdot10^{30}$ kg y contiene el 30$\%$ de hidrógeno, si todo el hidrógeno solar se convirtiera en helio se obtendría una energía de $3.75\cdot10^{44}$ J.
			\item 
				La Tierra recibe una irradiancia de $1370 \dfrac{\textit{W}}{\textit{m}^2}$ (constante solar) en la parte exxterna de la atmósfera, y como la distancia media del Sol a la Tierra es de $1.5\cdot10^{11}$ m, el Sol emita una radiación igual a:
				\[4\pi(1.5\cdot10^{11})^{2}\cdot1370 = 3.8\cdot10^{26}\dfrac{\textit{J}}{\textit{s}}\]
				La Tierra podrá alimentarse de radiaciones durante $3.13\cdot10^{10}$ años.
			\item 
				No toda la radiación interceptada por la Tierra es absorbida; una fracción de la energía incidente es reflejada
				de regreso al espacio, principalmente por las nubes ($\simeq20\%$), por los constituyentes atmosféricos ($\simeq6\%$) y por la superficie terrestre ($\simeq4\%$).
			\item 
				En la superficie terrestre llega una irradiancia alrededor de los $1.000 \dfrac{\textit{W}}{\textit{m}^2}$
		\end{itemize}
			
			\subsubsection{Energía solar.}
				Está constituida simplemente por la porción de la luz que emite el Sol y que es	interceptada por la Tierra. Puede ser:
				\begin{itemize}
					\item \underline{Directa:}
						una de las aplicaciones de la energía solar es directamente como luz solar, por ejemplo, para la iluminación de recintos. En este sentido, cualquier ventana es un colector solar.
					\item \underline{Térmica:}
						se denomina térmica la energía solar cuyo aprovechamiento se logra por medio del
						calentamiento de algún medio: agua o aceite. Pueden ser:
						\begin{itemize}
							\item \textit{De baja temperatura:} 
								con colectores para producción de ACS (Agua Caliente Sanitaria).
							\item \textit{De alta temperatura:} 
								centrales termosolares para producción de energía eléctrica mediante espejos parabólicos o con heliostatos con receptor central en torre.
						\end{itemize}
					\item \underline{Fotovoltaica:}
						es el aprovechamiento de la energía solar por medio de células fotoeléctricas,
						capaces de convertir la luz en potencial eléctrico, sin pasar por un efecto térmico. El conjunto de
						células fotoeléctricas se denomina panel fotovoltaico. Se encuentra en las centrales fotovoltaicas.
				\end{itemize}
				
			\subsubsection{Energía eólica.}
				Es la energía que se extrae del viento que procede de la energía solar y del movimiento de rotación
				de la Tierra. La aplicación más importante es con la utilización de aerogeneradores en parque eólicos (\textit{onshore} y \textit{offshore}).
			\subsubsection{Energía de la biomasa.}
				\textit{Definición de biomasa:} conjunto de materiales de origen biológico, vegetal, animal o procedente de
				la transformación natural o artificial de estos materiales, utilizados para la producción de energía
				eléctrica o térmica.\\
				\indent Es un tipo de producción de energía gestionable. Depende de la disponibilidad de biomasa. Se utiliza en:	
				\begin{itemize}
					\item Centrales térmicas de combustión de biomasa con turbinas de vapor.
					\item Centrales térmicas de gasificación de biomasa con cogeneración:
					\begin{itemize}
						\item MACI (Motores Alternativos de Combustión Interna).
						\item Turbina de gas.
					\end{itemize}
				\end{itemize}
				
			\subsubsection{Energías marinas.}
				\begin{itemize}
					\item \textit{Diferencia de temperatura oceánica (OTEC):} 
						Consiste en aprovechar la diferencia de temperatura que existe entre la superficie del océano (unos 20 \textdegree C o más en las zonas tropicales) y la correspondiente a unas decenas de metros debajo de la superficie (cercana a 4 \textdegree C).
					\item \textit{Energía de las olas: central undimotriz:}
						También se puede aprovechar el vaivén de las olas del mar para generar energía eléctrica. Las olas son, a
						su vez, producidas en parte ,por el efecto del viento sobre el agua y por el movimiento rotacional de la Tierra.
					\item \textit{Energía de las mareas: central mareomotriz:}
						Depende de la atracción gravitatoria del Sol y la Luna En algunas regiones costeras se dan unas mareas
						especialmente altas y bajas. La amplitud de la marea en algunos puntos de la Tierra puede alcanzar los 10 m.
					\item \textit{Energía de las corrientes marinas:}
						A profundidades de 20 a 30 m existen unas corrientes marinas de baja velocidad (2 a 3 m/s) que dependen de
						los ciclos de las mareas.
					\item \textit{Gradiente salino o Potencia osmótica:}
						Aprovechar la diferencia de salinidad entre el agua de los océanos y el agua de los ríos.
				\end{itemize}
				
			\subsubsection{Energía hidráulica.}
				Se obtiene a partir de caídas de agua, artificiales o naturales. Estrictamente, también esta es una forma derivada de la energía solar, porque el Sol provee la fuerza impulsora del ciclo hidrológico.\\
				\indent Se dividen en grandes y pequeñas centrales hidroeléctricas.
		
			\subsubsection{Hidrógeno. Pila de combustible.}
				El uso del hidrógeno como portador energético para complementar los mercados de la electricidad y
				combustibles líquidos presenta ventajas de versatilidad de fuentes de suministro, almacenamiento eficaz y
				bajas emisiones en los puntos de consumo.\\
				\indent La pila de combustible combina hidrógeno y oxígeno a través de una membrana de intercambio protónico,
				puede generar energía eléctrica obteniéndose como único residuo agua.\\
				\indent Para conseguir hidrógeno hay que consumir energía eléctrica. El hidrógeno se puede obtener:
				\begin{itemize}
					\item \textit{Blanco:} en estado bruto en el subsuelo.
					\item \textit{Verde:} del agua mediante electrolisis con energías renovables y biometano.
					\item \textit{Rosa:} del agua mediante electrolisis con energía nuclear.
					\item \textit{Gris:} del gas natural.
					\item \textit{Azul:} del gas natural, pero se captura el $\textit{CO}_2$ residual.
					\item \textit{Marrón:} del carbón.
				\end{itemize}
				Para transformar de vuelta el hidrógeno en electricidad hay dos métodos:
				\begin{itemize}
					\item \textit{Grupo electrógeno:} utilizando un motor de combustión interna.
					\item \textit{Pila de combustible:} $H_2 + \dfrac{1}{2}O_2 \longrightarrow H_2O$ + Calor + Energía Eléctrica.
				\end{itemize}
		
			\subsubsection{Energía geotérmica.}
				La energía geotérmica es un tipo de energía renovable que se basa en el aprovechamiento del calor
				que existe en el subsuelo de nuestro planeta. Es decir, utilizar el calor de las capas internas de la
				Tierra y con él genera energía.\\
				\indent La temperatura de la Tierra va a aumentando conforme descendemos y nos acercamos al núcleo
				terrestre. El gradiente térmico hace aumentar la temperatura del suelo entre 2 \textdegree C y 4 \textdegree C por cada 100 metros que descendemos. Hay diversas zonas del planeta donde este gradiente es mucho mayor y se debe
				a que la corteza terrestre es más delgada en ese punto.
		
		\subsection{Combustibles fósiles.}
			\subsubsection{Reservas y recursos mundiales.}
				Las cantidades de materia prima energética que pueden aprovecharse para su transformación en energía útil en condiciones económicas rentables se denominan reservas (explotables). Cuando hay razones suficientes para la
				existencia de cantidades mayores, a estas se les denomina recursos (previsibles).\\
				\indent La evaluación de las reservas energéticas existentes en nuestro planeta se estimaban en\linebreak
				 $33000\cdot10^{18}$ J, a los cuales habría que añadir $349395\cdot10^{18}$ J correspondientes a recursos de difícil explotación.
		
			\subsubsection{Evolución de las reservas de petróleo y gas.}
				\begin{center}
					\begin{tabular}{|c|c|c|c|c|c|c|}
							\hline
								& \textbf{1984} & \textbf{1994} & \textbf{2004} & \textbf{2020} & \textbf{Duración Reservas} & \textbf{Evolución R/P}\\
							\hline
							Petróleo - $10^9$ barriles & 761 & 1017 & 1185 & 1732 & \multirow{2}{*}{42 años} & \multirow{2}{*}{31 a 42 años}\\
							Petróleo - $\%$ & 100 & 134 & 156 & & & \\
							\hline
							Gas - $10^{12}$ $\textit{m}^3$	& 96 & 142 & 179 & 200 & \multirow{2}{*}{67 años} & \multirow{2}{*}{59 a 67 años}\\
							Gas - $\%$ & 100 & 148 & 186 & & & \\
							\hline
					\end{tabular}
				\end{center}
		
		
		
		
		
		
		
		
		
		
		
		
		
		
		
		
		
		
		
		
		
		
		
		
		
		
		
		
		