\documentclass{book}
\usepackage[gen]{eurosym}
\usepackage[spanish]{babel}
\usepackage{amsmath}
\usepackage{graphicx}
\usepackage{subfigure}
\usepackage{listings}
\usepackage{tikz}
\usepackage{circuitikz}
\usetikzlibrary{babel}
\usepackage{array}
\usepackage{makecell}
\usepackage{tabularray}
\usepackage{subcaption}
\usepackage{booktabs}
\usepackage{siunitx}
\usepackage{pgfplots}
\usepackage{pgfplotstable}
\usepgfplotslibrary{colormaps}
\pgfplotsset{compat=1.17}
\usepgfplotslibrary{groupplots}
\usepackage{pdflscape}
\usepackage{amssymb}

\usepackage{multicol}
\usepackage{multirow}
\usepackage{float}
\usepackage{xcolor}
\usepackage{pgf-pie}
\usepackage{pgffor}
\usetikzlibrary{arrows}
\usepackage{titlesec}
\usepackage[T1]{fontenc}
\usepackage{tgpagella}
\usepackage{amsmath}
\usepackage[a4paper, top=2cm, bottom=2cm, left=3cm, right=3cm, marginparwidth = 1.75cm]{geometry}

\graphicspath{ {./res/tema1/} }

\setlength\parindent{12pt}

\title{Apuntes de Diseño de Centrales Eléctricas}
\author{Bogurad Barañski Barañska \and Adrián Teixeira de Uña}
\date{\today}

\def\innerradius{3cm}
\def\outerradius{4cm}
\newcommand{\grado}{°C}

% Macro principal para el gráfico de donut
\newcommand{\donutchart}[1]{
	% Calcula el total
	\pgfmathsetmacro{\totalnum}{0}
	\foreach \value/\colour/\name in {#1} {
		\pgfmathparse{\value + \totalnum}
		\global\let\totalnum=\pgfmathresult
	}
	
	\begin{tikzpicture}
		% Calcula el grosor y la línea media del gráfico
		\pgfmathsetmacro{\wheelwidth}{\outerradius - \innerradius}
		\pgfmathsetmacro{\midradius}{(\outerradius + \innerradius) / 2}
		
		% Gira el gráfico para comenzar desde la parte superior
		\begin{scope}[rotate=90]
			% Bucle a través de cada conjunto de valores
			\pgfmathsetmacro{\cumnum}{0}
			\foreach \value/\colour/\name in {#1} {
				\pgfmathsetmacro{\newcumnum}{\cumnum + \value / \totalnum * 360}
				% Calcula el valor porcentual
				\pgfmathsetmacro{\percentage}{\value / \totalnum * 100}
				% Calcula el ángulo medio de los segmentos de color para colocar las etiquetas
				\pgfmathsetmacro{\midangle}{-(\cumnum + \newcumnum) / 2}
				% Esto es necesario para que las etiquetas se alineen correctamente
				\pgfmathparse{(-\midangle < 180 ? "west" : "east")}
				\edef\textanchor{\pgfmathresult}
				\pgfmathsetmacro\labelshiftdir{1 - 1 * (-\midangle > 180)}
				% Dibuja los segmentos de color
				\fill[\colour] (-\cumnum:\outerradius) arc (-\cumnum:-(\newcumnum):\outerradius) -- (-\newcumnum:\innerradius) arc (-\newcumnum:-(\cumnum):\innerradius) -- cycle;
				% Establece el ángulo acumulado antiguo al nuevo valor
				\global\let\cumnum=\newcumnum
			}
		\end{scope}
	\end{tikzpicture}
}

\begin{document}
\frontmatter	
	\mainmatter	
		\maketitle
		\tableofcontents
		\addcontentsline{toc}{chapter}{Índice}
		\section{Tema 1: Fundamentos y propiedades de los fluidos}
\subsection{Hipótesis de medio continuo}
Un fluido se caracteriza por un volumen (V) y una longitud característica (L) donde:

\[L \approx V^{\frac{1}{3}}\]

\begin{figure}[H]
	\centering
	\includegraphics[width=0.5\linewidth]{imagenesTema1/caracteristicasFluido}
	\caption{Magnitudes fundamentales de un fluido.}
	\label{fig:caracteristicasfluido}
\end{figure}
Como el tamaño de una molécula es de $d_0 \approx 10^{-11} \ a \ 10^{-10} m$. Por ello, la longitud característica debe ser mucho mayor que $d_0$ ($L>>d_0$) para así comprender el número suficiente de moléculas y poder estudiar la mecánica de fluidos de manera macroscópica.\\

Además, la longitud debe ser suficiente para que exista equilibrio termodinámico local y así poder aplicar las ecuaciones de estado:

\begin{itemize}
	\item Camino libre medio ($\lambda$) de interacción por choque entre moléculas.
	\begin{itemize}
		\item En líquidos: $\lambda \approx d_o$
		\item En gases: $\lambda >> d_o$
	\end{itemize}
	\[L >> \lambda\]
\end{itemize}

\begin{figure}[H]
	\centering
	\includegraphics[width=0.35\linewidth]{imagenesTema1/caminoLibreMedio}
	\caption{Camino libre medio.}
	\label{fig:caminolibremedio}
\end{figure}

En este fluido, es necesario poder medir:
\begin{enumerate}
	\item \underline{\textbf{Densidad}}: el diferencial de volumen debe ser una muestra significativa a nivel estadístico.
	
	\[\rho(\vec{r},t)=\lim_{{V \to 0}} \frac{\Delta m}{\Delta V}=\frac{dm}{dV} \ \left[{\frac{kg}{m^3}}\right]\] 
	
	
	\begin{itemize}
		\item El fluido es un gas si: $\rho \neq cte \rightarrow \rho=f(\vec{r},t)$
		\item El fluido es un líquido si: $\rho = cte \rightarrow \rho=f(t)$
	\end{itemize}
	
	Si la función depende del tiempo, se dice que está en forma paramétrica.
	
	\begin{itemize}
		\item Peso específico
		\[\gamma=\rho g  \rightarrow g: \text{campo gravitatorio} \left[{\frac{m}{s^2}}\right]\]
		\item Densidad relativa
		\[\rho_{rel}=\frac{\rho}{\rho_{ref}}\]
		
		\begin{itemize}
			\item Líquidos: $\rho_{ref}=\rho_{agua}\approx 10^3 \frac{kg}{m^3}$
				\item Gases: $\rho_{ref}=\rho_{aire_{CN}}\approx 1 \frac{kg}{m^3}$
		\end{itemize}
	\end{itemize}
		
	\item \underline{\textbf{Velocidad}}:
	\[\vec{v}(\vec{r},t)=\lim_{{\Delta V \to 0}} \frac{\sum m_i \vec{v_i}}{\sum m_i} \left[{\frac{m}{s}}\right]\]
	\item \underline{\textbf{Presión}}: Es una magnitud absoluta (siempre mayor que 0):
	
	\[P=\frac{d(\vec{F}\cdot\vec{n})}{dS} =\frac{d{F_n}}{dS} [Pa]\]
	\[1 bar = 10^5 Pa\]
	\[1 atm = 101325 Pa\]
	\[1 mmHg =\rho_{Hg}gh=132.32 Pa\]
	\[1 mca \text{(metros columna agua)} = \rho_{H_2O}gh=9.8\cdot 10^3 Pa\]
	\begin{itemize}
		\item Presión manométrica ($P_{man}$): Se mide normalmente con un manómetro diferencial:
		
		\[P_{man} = P - P_{atm} \rightarrow P>P_{atm}\]
		\item Presión vacuométrica ($P_{vac}$): Se mide normalmente con un vacuómetro.
		
		\[P_{vac}= P_{atm}-P \rightarrow P<P_{atm}\]
		
		\item Presión de vapor ($P_v$): Se refiere al equilibrio de fase líquido - gas. Si la presión es menor que la presión de vapor \textbf{cavita}.
		\item Cavitación: Generación de burbujas en el líquido por estar por debajo de la presión de vapor que posteriormente al subir la presión explotan con violencia. 
	\end{itemize}
	
\end{enumerate}

\subsection{Ecuaciones equilibrio termodinámico local}
En un gas ideal, si las condiciones son subsónicas se cumple que:
\[\frac{P}{\rho}=R_gT \rightarrow R_g\frac{R}{mmr} \rightarrow R ={8.314} \frac{J}{mol K}\]

El fluido está en condiciones subsónicas si:
\[\left| \vec{v}(\vec{r},t) \right|<a=\sqrt {\left. \frac{\partial P}{\partial \rho} \right|}_{S=cte}\]

\begin{itemize}
	\item Ecuación isoentrópica: Procesos rápidos.
	\[PV^\alpha=cte\]
	\item Ecuación isoterma: Procesos lentos.
	\[PV=cte\]
\end{itemize}
\subsection{Fuerzas y respuestas en sólidos y fluidos}
\begin{enumerate}
	\item \underline{\textbf{Fuerzas en un fluido}}:
	\[F=f(\Delta \dot{x})=C\dot{x} \rightarrow C: \text{constante de amortiguamiento}\]
	
	\item \underline{\textbf{Tensión tangencial o de cizalladura}}($\tau$):
	\[\tau=\lim_{{S \to 0}} \frac{\Delta F_t}{\Delta S}=\frac{dF_t}{dS}\]
	\item \underline{\textbf{Viscosidad}}($\mu$):
	En fluidos newtonianos la viscosidad es relativamente constante: 
	\[\mu=f(T) [Pa \cdot s]\]
	\[\tau =\mu \dot{\varepsilon} = \mu \frac{\Delta v_n}{\Delta l_n} \rightarrow \dot{\varepsilon}\ \text{es la velocidad de deformación} [s^-1]\]
\begin{figure}[H]
	\centering
	\includegraphics[width=0.7\linewidth]{imagenesTema1/viscosidad}
	\caption{Cálculo de viscosidad.}
	\label{fig:viscosidad}
\end{figure}

\[\tau =\frac{F}{A}=\mu \frac{\Delta v_n}{\Delta l_n} =\mu \frac{v - 0}{l_n}= \mu \frac{v}{l_n}\]
	 
	 En fluidos no newtonianos la viscosidad no es constante:
	 
	 \begin{tikzpicture}
	 	\begin{axis}[
	 		domain=0:3,
	 		samples=100,
	 		height=7cm,
	 		width=10cm,
	 		title={},
	 		ylabel=$\boldsymbol\tau$,
	 		xlabel=$\dot{\boldsymbol\varepsilon}$,
	 		grid=major,
	 		axis lines=center,
	 		legend style={at={(0.5,-0.15)}, anchor=north},
	 		]
	 		\addplot[blue, thick] {x};
	 		\legend{Fluido Newtoniano}
	 		
	 		\addplot[red, thick, domain=1:3] {ln(x)+1};
	 		\addlegendentry{Fluido no newtoniano}
	 		
	 		\addplot[red, thick, domain=1:3] {e^x-e^1+1};
	 		\addlegendentry{Fluido no newtoniano}
	 	\end{axis}
	 \end{tikzpicture}
	 
	 Viscosidades típicas:
	 \[\mu_{H_2O}=10^{-3} Pa \cdot s = 1cP \rightarrow P \ \text{Poise}\]
	
	\item \underline{\textbf{Viscosidad cinemática}}:
	
	\[\nu = \frac{\mu}{\rho} \left[\frac{m^2}{s}\right] \rightarrow 1 csk = 10^{-6}\left[\frac{m^2}{s}\right] \rightarrow csk \ \text{centi-stoke}\]
	
	\item \underline{\textbf{Interfases}}: 
	\begin{itemize}
		\item Vaso grande: Existe intercambio de moléculas en la interfase pero las presiones se equilibran.
		 \begin{figure}[H]
		 	\centering
		 	\includegraphics[width=0.5\linewidth]{imagenesTema1/vasoGrande}
		 	\caption{Interfase Vaso grande.}
		 	\label{fig:vasogrande}
		 \end{figure}
		 
		\item Vaso pequeño: Existe efecto de la tensión superficial ($\sigma \left[
		\frac{N}{m}\right]$) descrita mediante la ecuación de Laplace-Young. Solo aplica a fluidos inmiscibles. 
		\begin{figure}[H]
			\begin{minipage}{0.7\textwidth}
				\centering
				\includegraphics[width=0.2\linewidth]{imagenesTema1/vasoPeque}
				\caption{Efecto tensión superficial Vaso pequeño}
				\label{fig:vasopeque}
			\end{minipage}%
			\begin{minipage}{0.3\textwidth}
				\[P_a - P_{liquido}=\sigma K\]
			\[K \ \text{expresión de la curvatura}\]
			\[K=\nabla_s \vec{n} = \left(\frac{1}{R_i}+\frac{1}{R_j}\right) \]
			\[R_i \ y \  R_j \ \text{son radios característicos}\]
			
			\end{minipage}
		\end{figure}

		\item Zona de efecto: La tensión superficial siempre presenta efectos, no obstante solo se aprecia en una región concreta.
		\[\rho g l_c \approx \sigma \rightarrow l_c \approx \left(\frac{\sigma}{\rho g}\right)^{\frac{1}{2}}\]
		
		\begin{figure}[H]
			\centering
			\includegraphics[width=0.4\linewidth]{imagenesTema1/zonaEfectos}
			\caption{Zona de efecto.}
			\label{fig:zonaefectos}
		\end{figure}
		
		\item Casos particulares
		\begin{enumerate}
			\item Chorro
			\begin{figure}[H]
				\begin{minipage}{0.4\textwidth}
				\centering
				\includegraphics[width=0.7\linewidth]{imagenesTema1/chorro}
				\caption{Chorro.}
				\label{fig:chorro}
			\end{minipage}%
			\begin{minipage}{0.3\textwidth}
			\[P_i - P_e =\sigma\left(\frac{1}{R_i}+\frac{1}{R_j}\right)=\frac{\sigma}{R}\]
			
			\end{minipage}
			\end{figure}
			
			\item Gota
			
			\begin{figure}[H]
				\begin{minipage}{0.4\textwidth}
				\centering
				\includegraphics[width=0.7\linewidth]{imagenesTema1/gota}
				\caption{Gota.}
				\label{fig:gota}
					\end{minipage}%
				\begin{minipage}{0.3\textwidth}
				\[P_i - P_e =\sigma\left(\frac{1}{R_i}+\frac{1}{R_j}\right)=\frac{2\sigma}{R}\]
				
				\end{minipage}
				
			\end{figure}
			
			\item Pompa
			
			\begin{figure}[H]
				\begin{minipage}{0.4\textwidth}
				\centering
				\includegraphics[width=0.7\linewidth]{imagenesTema1/pompa}
				\caption{Pompa.}
				\label{fig:pompa}
			\end{minipage}%
			\begin{minipage}{0.3\textwidth}
			\[P_i - P_m =\frac{2\sigma}{R}\]
			\[P_m - P_e =\frac{2\sigma}{R}\]
			\[P_i - P_e =\frac{4\sigma}{R}\]
			
			\end{minipage}
			\end{figure}
			
			\item Plano
			
			\begin{figure}[H]
								\begin{minipage}{0.4\textwidth}
				\centering
				\includegraphics[width=0.7\linewidth]{imagenesTema1/plano}
				\caption{Plano.}
				\label{fig:plano}
			\end{minipage}%
			\begin{minipage}{0.3\textwidth}
			\[P_i - P_e =\sigma\left(\frac{1}{R_i}+\frac{1}{R_j}\right)=0\rightarrow P_i = P_e\]
			
			\end{minipage}
			\end{figure}
			
		\end{enumerate}
	\end{itemize}
\end{enumerate}

\subsection{Mojabilidad}
\begin{itemize}
	\item Un líquido no moja a un sólido si $\theta_c \gtrsim \ang{150} $. Sólido hidrofóbico.
	\item Un líquido moja a un sólido si $\theta_c  \lesssim \ang{45}$. Sólido hidrofílico totalmente.
\end{itemize}
\begin{figure}[H]
	\centering
	\includegraphics[width=0.7\linewidth]{imagenesTema1/mojabilidad}
	\caption{Mojabilidad.}
	\label{fig:mojabilidad}
\end{figure}

		\chapter{Energía eléctrica y desarrollo sostenible.}
\section{Introducción al desarrollo sostenible.}
El desarrollo nació en los años 70 en los países nórdicos y se define como:
\begin{itemize}
	\item [-] El que satisface
	nuestras necesidades actuales sin poner en peligro la
	capacidad de las generaciones futuras para satisfacer sus
	propias necesidades abarcando:
	\begin{itemize}
		\item El capital social
		\item El capital ambiental
		\item El capital económico
	\end{itemize}
\end{itemize}
\subsection{Cumbres climáticas.}
\begin{itemize}
	\item [-] \textbf{Rio de Janeiro (1992):}
		Se crea la Comisión del Desarrollo Sostenible para impulsar la sostenibilidad de las políticas de desarrollo humano y gestionar sus riesgos.
	\item [-] \textbf{Cumbre del Protocolo de Kyoto (1997):}
		Se adquiere un compromiso entre los países industrializados con el \textbf{objetivo} de reducir las emisiones de gases de efecto invernadero un 5,9\% en el periodo 2008 - 2012 con respecto a 1990 (año base). En una fase inicial no incluía a países en desarrollo como China e India por su baja contaminación per capita.
	\item [-] \textbf{Cumbre de París (2015):}
		Se comprometen los países a que la temperatura mundial no aumente más de 2\textdegree C respecto a los niveles preindustriales y limitarlo a 1,5\textdegree C para el 2020.
	\item [-] \textbf{Cumbre de Marrakech (2016):}
		Se ratifican los acuerdos de la Cumbre de París y se compromete reducir el 80\% de las emisiones de CO$_2$ para 2050.
	\item [-] \textbf{Cumbre de Katowice (2018):}
		Limitar a un incremento a final de siglo de 1,5 a 2\textdegree C respecto a los niveles preindustriales.
	\item [-] \textbf{Cumbre de Chile celebrada en Madrid (2019):}
		Los grandes contaminadores se niegan a intensificar los esfuerzos para mantener la temperatura por debajo de 1,5\textdegree C.
	\item [-] \textbf{Cumbre de Glasgow (2021):}
		Se mantiene el objetivo de 1,5\textdegree C para 2030. Acuerdo China - USA para reducir las emisiones de metano. Compromiso de 130 países para poner fin a la deforestación.
	\item [-] \textbf{Cumbre de Sharm el-Sheij en Egipto (2022):}
		Se acuerdan:
		\begin{itemize}
			\item Una alianza global contra la sequía.
			\item Una coalición contra la deforestación.
			\item Impulsar el hidrógeno verde.
			\item Impulsar la energía eólica marina.
		\end{itemize}
	\item [-] \textbf{Cumbre de Dubai(2021):}
		Se acuerda reducir las emisiones mundiales de gases de efecto invernadero un 43\% hasta
		2030 y un 60\% hasta 2035 en relación con los niveles de 2019, y emisiones netas de dióxido de carbono cero
		para 2050.
\end{itemize}
\section{Gases de efecto invernadero.}
\subsection{CO$_2$ equivalente.}
Es una forma de poder reducir el impacto climático a una unidad común y así, poder compararlos. Para calcularlo se emplea el valor \textbf{GWP} (Global Warming Potencial) o \textbf{PCG} (Potencial de calentamiento global) que miden cuanto calor atrapan en comparación con el CO$_2$ para un periodo de tiempo. 


Este valor depende de:
\begin{itemize}
	\item [-] La absorción de radiación infrarroja.
	\item [-] La ubicación del espectro de absorción.
\end{itemize}
\subsection{Dióxido de carbono (CO$_2$).}
Es la sustancia que más contribuye al efecto invernadero. Absorbe gran parte de la radiación solar incidente.
\subsection{Óxido nitroso (N$_2$O) y óxidos de nitrógeno (NO$_x$ ).}
Son los gases de efecto invernadero más destructivos con la capa de ozono. Relacionados con el sector agrario y la quema de combustible.
\subsection{Metano (CH$_4$).}
Tiene un potencial de calentamiento muy elevado GWP = 25. Se emite por el sector ganadero, el el de tratamiento de residuos y durante el transporte de hidrocarburos.
\subsection{Hidrofluorocarbonos (HFC).}
Son gases empleados como refrigerantes. No dañan al ozono pero tienen un GWP = 1000 y una larga permanencia en la atmósfera.
\subsection{Perfluororcarburos (PFC).}
Similares a los HFC.
\subsection{Hexafluoruro de azufre (SF$_6$).}
Se emplea para equipos de distribución de energía eléctrica. Tiene propiedades similares a los HFC y PFC.

\begin{table}[H]
	\centering
	\begin{tabular}{p{2cm}p{2cm}p{2cm}p{2cm}p{2cm}}
		\hline
		\textbf{ } & \textbf{CO$_2$} & \textbf{ClFCs} & \textbf{CH$_4$} & \textbf{N$_2$O} \\ \hline
		Importancia según contribución al efecto invernadero & Más del 50\% & 20 \% aprox. & 12 a 14 \% & 6 a 7 \% \\ \hline
		Tiempo de permanencia en la atmósfera & 50 - 200 años & 75 - 100 años & 7 a 10 años & 150 años aprox. \\ \hline
		Tasa de crecimiento anual (\%)&0,5&4-5&1&0,35\\ \hline 
		Principal origen de la contaminación&Combustión del petróleo, carbón y gas deforestación&Aerosoles y disolventes Espumas industriales Equipos de refrigeración&Pantanos Ganadería Minería&Fertilizantes Combustible fósiles\\ \hline 
	\end{tabular}
	\label{tab:my-table}
\end{table}

\section{Efecto invernadero.}
Como consecuencia de los gases de efecto invernadero se absorbe una mayor cantidad de radiación infrarroja que escaparía de la tierra y, por tanto, aumentando la temperatura atmosférica. 
\begin{figure}[H]
	\centering
	\includegraphics[width=0.7\linewidth]{res/tema2/T-Emisiones}
	\label{fig:t-emisiones}
\end{figure}
\subsection{Forzamiento radiactivo o climático.}
Es la diferencia entre la radiación solar absorbida por la Tierra y la energía irradiada de vuelta al espacio. Esta diferencia se contempla mediante el RCP donde:
\begin{itemize}
	\item [-] RCP = 2 es un escenario con el indicador muy bajo.
	\item [-] RCP = 8,5 es un escenario con el indicador muy alto.
\end{itemize}

\begin{table}[h]
	\centering
	\renewcommand{\arraystretch}{1.5}
	\begin{tabular}{cccccc}
		\hline
		Escenario & Forz. Radiat. (W/m\textsuperscript{2} en 2100) & \multicolumn{2}{c}{GC} & \multicolumn{2}{c}{GtCO$_2$} \\
		\cline{3-6}
		& &Media & Rango & Media & Rango \\
		\hline
		RCP2.6 & 2.8 & 270 & 140 a 410 & 990 & 510 a 1505 \\
		RCP4.5 & 4.5 & 780 & 595 a 1005 & 2880 & 2180 a 3690 \\
		RCP6.0 & 6 & 1060 & 840 a 1250 & 3885 & 3080 a 4585 \\
		RCP8.5 & 8.5 & 1685 & 1415 a 1910 & 6180 & 5163 a 7005 \\
		\hline 
	\end{tabular}
\end{table}

\begin{table}[H]
	\centering
	\renewcommand{\arraystretch}{1.5}
	\begin{tabular}{p{4cm}p{2cm}p{1cm}p{3cm}p{1cm}p{3cm}}
		\hline
		& \textbf{Escenario} &
		\multicolumn{2}{c}{\textbf{2046-2065}}  & \multicolumn{2}{c}{\textbf{2081-2100}} \\ 
	
		& &\textbf{Media} & \textbf{Rango probable} & \textbf{Media} &\textbf{Rango probable} \\ 
		\hline
		Cambio en la   & 
		RCP2,6 & 1 & 0,4 a 1,6 &1 &0,3 a 1,7\\
		temperatura media global&RCP4,5 & 1,4 & 0,9 a 2,0 &1,8 & 1,1 a 2,2 \\
		del aire en superficie & RCP6 & 1,3 & 0,8 a 1,8 & 2,2 & 1,4 a 3,1\\
		(en °C)& RCP8,5 & 2 & 1,4 a 2,6 & 3,7 & 2,6 a 4,8 \\
		\hline
		Elevación media mundial   & RCP2,6 & 0,24 & 0,17 a 0,32 & 0,4  & 0,26 a 0,55 \\
		del nivel del mar& RCP4,5 & 0,26 & 0.19 a 0,33 & 0,47 & 0,32 a 0,63 \\
		(en metros)& RCP6   & 0,25 & 0.18 a 0,32 & 0,48 & 0,33 a 0,63 \\
		& RCP8,5 & 0,3  & 0,22 a 0,38 & 0,63 & 0,45 a 0,82 \\
		\hline
	\end{tabular}
\end{table}
\newpage
\subsection{Evolución de las emisiones de CO$_2$ equivalente en España.}
España se comprometió con la unión europea en reducir emisiones para el periodo 2008 - 2012 en un 15\% respecto a 1990 (Fase I y II). Para el periodo 2013 - 2020 se comprometió a reducir emisiones en un 20\% respecto a los niveles del año base.
 
\begin{figure}[H]
	\centering
	\begin{tikzpicture}
		\begin{axis}[
			xlabel={Año},
			ylabel={kt CO\textsubscript{2}},
			xtick={1990,1995,2000,2005,2010,2015,2020,2025},
			ytick={270, 300, ..., 450},
			grid=both,
			grid style={line width=.1pt, draw=gray!10},
			major grid style={line width=.2pt,draw=gray!50},
			width=12cm,
			height=8cm,
			]
			\addplot[mark=*] coordinates {
				(1990, 287.710)
				(1995, 327.011)
				(2000, 383.276)
				(2005, 438.760)
				(2010, 354.652)
				(2015, 333.623)
				(2018, 328.905)
				(2019, 309.814)
				(2020, 272.244)
				(2021, 288.848)
				(2022, 314.000)
			};
		\end{axis}
	\end{tikzpicture}

\end{figure}

En cuanto a las emisiones asociadas a la generación eléctrica:
\begin{table}[H]
	\centering
	\begin{tabular}{p{3cm}l*{11}{c}}
		\toprule
		tCO\textsubscript{2} $\times$ 1.000.000& 2011 & 2012 & 2013 & 2014 & 2015 & 2016 & 2017 & 2018 & 2019 & 2020 & 2021 & 2022 \\
		\midrule
		Carbón &   41,0 & 51,1 & 37,5 & 41,1 & 50,0 & 35,4 & 42,8 & 36,0 & 12,4 & 4,9 & 4,9 & 7,5 \\
		Fuel + Gas  &  0,0 & 0,0 & 0,0 & 0,0 & 0,0 & 0,0 & 0,0 & 0,0 & 0,0 & 0,0 & 0,0 & 0,0 \\
		Motores diésel &  2,9 & 2,9 & 2,7 & 2,6 & 2,7 & 2,8 & 2,7 & 2,2 & 2,0 & 1,6 & 1,7 & 1,7 \\
		Turbina de gas &  0,9 & 1,0 & 0,7 & 0,8 & 0,7 & 0,5 & 0,7 & 1,0 & 0,7 & 0,4 & 0,5 & 0,7 \\
		Turbina de vapor &  2,3 & 2,4 & 2,2 & 1,8 & 2,0 & 2,3 & 2,4 & 2,2 & 2,0 & 1,3 & 1,0 & 1,1 \\
		Ciclo combinado  & 21,0 & 16,4 & 11,4 & 10,5 & 12,0 & 12,0 & 14,9 & 11,8 & 21,2 & 17,1 & 17,4 & 26,2 \\
		Cogeneración  &  11,6 & 12,3 & 11,7 & 9,2 & 9,6 & 9,8 & 10,7 & 11,0 & 11,3 & 10,1 & 9,7 & 6,6 \\
		Residuos no renovables &  0,3 & 0,4 & 0,4 & 0,5 & 0,6 & 0,6 & 0,6 & 0,6 & 0,5 & 0,7 & 0,8 & 0,6 \\
		Total Emisiones &  80,1 & 86,4 & 66,6 & 66,5 & 77,6 & 63,5 & 74,9 & 64,9 & 50,0 & 36,1 & 35,9 & 44,4 \\ 
		\\
		\hline
		\\
		Factor de emision de CO\textsubscript{2} (tCO\textsubscript{2}/MWh) &  0,29 & 0,31 & 0,24 & 0,25 & 0,29 & 0,24 & 0,29 & 0,25 & 0,19 & 0,15 & 0,14 & 0,16 \\
		\bottomrule
	\end{tabular}
\end{table}

En cuanto a los rendimientos de diversas plantas de generación eléctrica:
\begin{table}[h]
	\label{tab:conversion-efficiency}
	\begin{tabular}{lcccc}
		\toprule
		&Eficiencia conversión&\multicolumn{3}{c}{Emisiones en gramos/kWh} \\
		\cline{3-5}
		&  (\%) & NOx & SO2 & CO2 \\
		\midrule
		Carbón pulverizado (sin descontaminar S) & 36 & 1.29 & 17.2 & 884 \\
		Carbón pulverizado (con descontaminación S) & 36 & 1.29 & 0.86 & 884 \\
		Carbón en lecho fluidificado & 37 & 0.42 & 0.84 & 861 \\
		Ciclo combinado de carbón gasificado & 42 & 0.11 & 0.3 & 758 \\
		Turbina de gas & 39 & 0.23 & 0 & 470 \\
		Ciclo combinado de turbina de gas & 53 & 0.1 & 0 & 345 \\
		\bottomrule
	\end{tabular}
\end{table}

\section{Protocolo de Kioto}
\subsection{Políticas y medidas}
\subsection{Creación de sumideros}
\subsection{Mecanismos flexibles}
		\chapter{Conservación de la masa.}
\section{Teorema del transporte de Reynolds.}
Se parte de una función genérica $\Phi=f(\vec{r},t) $
%\begin{figure}[H]
%	\centering
%	\includegraphics[width=0.7\linewidth]{imagenesTema3/magnitudesReynolds}
%	\caption{Evolución de una magnitud en un volumen fluido.}
%	\label{fig:magnitudesreynolds}
%\end{figure}

\begin{figure}[H]
	\centering
		\begin{circuitikz}
			\tikzstyle{every node}=[font=\large]
			\draw [-latex] (-1.75,15.75) -- (-1.75,21.25)node[pos=1,above]{$\vec{k}$};
			\draw [-latex] (-1.75,15.75) -- (5.75,15.75)node[pos=1,right]{$\vec{i}$};
			\draw [-latex] (-1.75,15.75) -- (-3.25,14.25)node[pos=1,left]{$\vec{j}$};
			\node [font=\large, color={rgb,255:red,128; green,0; blue,255}] at (0.5,20.75) {$V_C(t)$};
			\draw [ color={rgb,255:red,128; green,0; blue,255}, dashed] (2.25,20.25) .. controls (3,21.5) and (4,20.25) .. (5.5,20);
			\draw [ color={rgb,255:red,128; green,0; blue,255}, dashed] (5.5,20) .. controls (6.5,20) and (5.75,18.75) .. (5.75,17.5);
			\draw [ color={rgb,255:red,128; green,0; blue,255}, dashed] (5.75,17.5) .. controls (6,16) and (5,16.25) .. (3.5,16.5);
			\draw [ color={rgb,255:red,128; green,0; blue,255}, dashed] (3.5,16.5) .. controls (2.5,17) and (2,16.25) .. (1.75,17.5);
			\draw [ color={rgb,255:red,128; green,0; blue,255}, dashed] (1.75,17.5) .. controls (1.25,19) and (2,19.25) .. (2.25,20.25);
			\draw [ color={rgb,255:red,128; green,0; blue,255}, short] (-1,20.25) .. controls (0,21) and (0,19.75) .. (1,19.25);
			\draw [ color={rgb,255:red,128; green,0; blue,255}, short] (1,19.25) .. controls (2,19) and (1.5,18.5) .. (2,17.5);
			\draw [ color={rgb,255:red,128; green,0; blue,255}, short] (2,17.5) .. controls (2.5,16.5) and (1.5,16.75) .. (0.75,16);
			\draw [ color={rgb,255:red,128; green,0; blue,255}, short] (0.75,16) .. controls (-0.25,15) and (-0.25,16.25) .. (-1.25,16.25);
			\draw [ color={rgb,255:red,128; green,0; blue,255}, short] (-1.25,16.25) .. controls (-2.75,16.75) and (-1.75,17.5) .. (-2.5,18.75);
			\draw [ color={rgb,255:red,128; green,0; blue,255}, short] (-2.5,18.75) .. controls (-3,20) and (-1.75,19.5) .. (-1,20.25);
			\draw [ color={rgb,255:red,128; green,0; blue,255} ] (-0.75,19.25) circle (0.5cm);
			\draw [ color={rgb,255:red,128; green,0; blue,255}, short] (-1.25,19.25) .. controls (-1,19) and (-0.5,19) .. (-0.25,19.25);
			\draw [ color={rgb,255:red,128; green,0; blue,255}, dashed] (-1.25,19.25) .. controls (-1,19.5) and (-0.5,19.5) .. (-0.25,19.25);
			\node [font=\large, color={rgb,255:red,128; green,0; blue,255}] at (-0.5,20) {dV};
			\draw [ color={rgb,255:red,255; green,128; blue,0}, -latex] (-1.75,15.75) -- (-0.75,19.25)node[pos=0.8,left]{$\vec{r}$};
			\node [font=\large, color={rgb,255:red,128; green,0; blue,255}] at (5.25,20.5) {$V_C(t+dt)$};
			\draw [ color={rgb,255:red,0; green,128; blue,0}, -latex] (-0.75,19.25) -- (3.5,18.5)node[pos=1,above]{$\vec{v}_C(t)$};
			\node [font=\large, color={rgb,255:red,0; green,128; blue,0}] at (0.25,18.5) {$\Phi(\vec{r},$ t)};
		\end{circuitikz}
	\caption{Evolución de una magnitud en un volumen fluido.}
	\label{fig:magnitudesreynolds}
\end{figure}

Por definición de derivada:
\[\frac{d}{dt}\iiint_{V_c(t)} \Phi(\vec{r},t) \,dV=\lim_{\Delta t \to 0} \left[\iiint_{V_c(t+dt)} \Phi(\vec{r},t+dt) \,dV-\iiint_{V_c(t)} \Phi(\vec{r},t) \,dV\right]\]

Se hace el desarrollo de Taylor en t del primer término:
\[\frac{d}{dt}\iiint_{V_c(t)} \Phi(\vec{r},t) \,dV=\iiint_{V_c(t)} \frac{\partial}{\partial t}\Phi(\vec{r},t) \,dV+\lim_{\Delta t \to 0} \frac{1}{\Delta t}\left[\iiint_{V_c(t+dt)} \Phi(\vec{r},t) \,dV-\iiint_{V_c(t)} \Phi(\vec{r},t) \,dV\right]\]

Como solo se estudia la velocidad de compresión o expansión del fluido en la dirección de la superficie de control:
\[dV=\vec{v}_c\cdot\vec{n}dS\Delta t\]

Por tanto:
\begin{equation}\label{eq:ttr1}
\frac{d}{dt}\iiint_{V_c(t)} \Phi(\vec{r},t) \,dV=\iiint_{V_c(t)} \frac{\partial}{\partial t}\Phi(\vec{r},t) \,dV+\oiint_{S_c(t)} \Phi(\vec{r},t)\vec{v}_c\cdot\vec{n} \,dS
\end{equation}

De manera similar se puede aplicar esta deducción a un volumen fluido:

\begin{equation}\label{eq:ttr2}
\frac{d}{dt}\iiint_{V_f(t)} \Phi(\vec{r},t) \,dV=
\red{\underbrace{\black\iiint_{V_f(t)} \frac{\partial}{\partial t}\Phi(\vec{r},t) \,dV}_{\text{Variación local}}}
\black+
\red{\underbrace{\black\oiint_{S_f(t)} \Phi(\vec{r},t)\vec{v}\cdot\vec{n}\,dS}_{\text{Variación convectiva}}}
\black
\end{equation}

En un tiempo $t*$ paramétrico tal que $V_c(t*)=V_f(t*)$ se cumple que
\[ \iiint_{V_c(t*)}\frac{\partial \Phi}{\partial t}\,dV\approx \iiint_{V_f(t*)}\frac{\partial \Phi}{\partial t}\,dV\]

Por tanto, si se hace la siguiente resta \eqref{eq:ttr2} - \eqref{eq:ttr1}. Se obtiene el Teorema de Reynolds aplicado a los problemas:

\[\frac{d}{dt}\iiint_{V_f(t)}\Phi(\vec{r},t)\,dV=\frac{d}{dt}\iiint_{V_c(t)}\Phi(\vec{r},t)\,dV+\oiint_{S_c(t)} \Phi(\vec{r},t)\left[(\vec{v}-\vec{v}_c)\cdot\vec{n}\right] \,dS\]



Si la magnitud $\Phi = \rho$ Se obtiene la ecuación de conservación de la masa en forma integral, que como en todo el volumen fluido no varia es igual a 0:

\[\frac{d}{dt}\iiint_{V_f(t)}\rho\,dV=\frac{d}{dt}\iiint_{V_c(t)}\rho\,dV+\oiint_{S_c(t)} \rho\left[(\vec{v}-\vec{v}_c)\cdot\vec{n}\right] \,dS=0\]

Para todo el volumen fluido:
\[\frac{d}{dt}\iiint_{V_f(t)} \rho \,dV=\iiint_{V_f(t)} \frac{\partial \rho}{\partial t} \,dV+\oiint_{S_f(t)} \rho\vec{v}\cdot\vec{n} \,dS=0\]

Si $V_f(t)\approx dV_f(t)$ entonces aplicando el teorema de gauss se llega a la ecuación diferencial de la masa o forma conservativa:

\
\
\
\begin{center}
	\begin{tikzpicture}
		\draw [fill=lightgray](0,0) rectangle (5, 2);
		\node at (2.5, 1) [align = center]{Teorema de Gauss: \\$\oiint_S \varphi \cdot \vec{n}\,dS=\iiint_V \vec{\nabla}\cdot\varphi\,dV $};
	\end{tikzpicture}
\end{center}

\[\lim_{dV \to 0}\left[\frac{\partial \rho}{\partial t} dV+\vec{\nabla}\cdot\left(\rho\vec{v}\right)dV\right]=0\]
\[\frac{\partial \rho}{\partial t} +\vec{\nabla}\cdot\left(\rho\vec{v}\right)=0\]
Término local de masa: 
\[\frac{\partial \rho}{\partial t}\]
Término convectivo de masa:
\[\vec{\nabla}\cdot\left(\rho\vec{v}\right)\]

\section{Flujo sobre una superficie.}
\begin{itemize}
	\item Flujo másico
	\[G_e=\iint_{S_e} \rho\left(\vec{v}-\vec{v}_c\right)\cdot\vec{n}\,dS\]
	\item Flujo volumétrico
		\[Q_e=\iint_{S_e} \left(\vec{v}-\vec{v}_c\right)\cdot\vec{n}\,dS\]
\end{itemize}

\section{Propiedades en forma diferencial.}
Partiendo de la expresión de la derivada sustancial y de la conservación de la masa en forma diferencial:
\begin{equation} \label{eq:1}
	\frac{\partial \rho}{\partial t} +\vec{\nabla}\cdot\left(\rho\vec{v}\right)=\frac{\partial \rho}{\partial t} +\left(\vec{v}\cdot\vec{\nabla}\right)\rho+\left(\vec{\nabla}\cdot\vec{v}\right)\rho=0
\end{equation}

\begin{equation} \label{eq:4}
	\frac{D \rho}{D t}=\frac{ \partial \rho}{\partial t}+\left(\vec{v}\cdot\vec{\nabla}\right)\rho
\end{equation}

Restando \eqref{eq:1} a \eqref{eq:4}:
\[\frac{D \rho}{D t}=-\left(\vec{\nabla}\cdot\vec{v}\right)\rho\]
\begin{itemize}
	\item Si $\vec{\nabla}\cdot\vec{v} =0 $ Incopresible localmente.
	\item Si $\vec{\nabla}\cdot\vec{v} >0$ Se expande localmente el diferencial de Volumen.
	\item Si $\vec{\nabla}\cdot\vec{v} <0$ Se comprime localmente el diferencial de Volumen.
\end{itemize}

		\section{Tema 4: Conservación de la cantidad de movimiento}
\subsection{Teorema del transporte de Reynolds para la cantidad de movimiento}
Sea un volumen sometido a un conjunto de fuerzas:
\begin{figure}[H]
	\centering
	\includegraphics[width=0.7\linewidth]{imagenesTema4/magnitudesFuerzas}
	\caption{Magnitudes principales que influencian la cantidad de movimiento.}
	\label{fig:magnitudesfuerzas}
\end{figure}

Aplicando al teorema del transporte de Reynolds la función $\Phi=\rho\vec{v}$ y una similitud con la segunda ley de Newton se obtiene para volúmenes fluidos:
\[\iiint_{V_f}\vec{f}_V\,dV+\oiint_{S_f}\vec{f}_s\,dS=
\frac{d}{dt}\iiint_{V_f}\rho\vec{v}\,dV=
\iiint_{V_f}\frac{\partial \rho\vec{v}}{\partial t}\,dV+\oiint_{S_f}\rho\vec{v}\left(\vec{v}\cdot\vec{n}\right)\,dS\]

Para un volumen de control:
\[\iiint_{V_f}\vec{f}_V\,dV+\oiint_{S_f}\vec{f}_s\,dS=
\frac{d}{dt}\iiint_{V_f}\rho\vec{v}\,dV=
\iiint_{V_c}\frac{\partial \rho\vec{v}}{\partial t}\,dV
+\oiint_{S_c}\rho\vec{v}\left[\left(\vec{v}-\vec{v}_c\right)\cdot\vec{n}\right]\,dS\]

En estas expresiones aparecen dos tipos de fuerzas:
\begin{enumerate}
	\item Fuerzas volumétricas $\vec{f}_v$
	\item Fuerzas superficiales $\vec{f}_s$
\end{enumerate}
\subsection{Ecuaciones de Navier-Stokes}

\subsection{Número de Reynolds}

\subsection{Teorema de Bernouilli}

		\chapter{Despacho económico a corto plazo.}
	\section{Despacho económico diario de generación. Mercado diario.}
		\textbf{Objetivo:} repartir la demanda total del sistema $P_D\,[MW]$ entre los generadores disponibles, de forma que el coste total de generación, en $\euro/MWh/\text{día}$, sea el mínimo posible, con las mínimas pérdidas en la red y menores emisiones de $CO_{2,eq}$.
		
		
		El \textbf{coste total de generación} es variable en cada hora, debido a que las centrales que intervienen tienen eficiencias y costes muy distintos.
		
		
		La resolución del despacho económico (D.E.) considera una serie de \textbf{restricciones técnicas} que limitan el uso de las centrales.
		
		
		Es necesario considerar la opción de \textbf{acoplar o desacoplar los grupos de generación} según la variación de la demanda.
		
		
		\textbf{Programación Horaria de Grupos Térmicos:} problema de operar un sistema de energía eléctrica a coste mínimo, debido a que
		
		\begin{itemize}
			\item[-] los costes fijos de operación de una central térmica pueden ser comparativamente altos, luego no es viable operar a un nivel de producción bajo \textrightarrow Desacoplar centrales cuando hay baja demanda.
			\item[-] las energías renovables ofertan a coste 0.
			\item[-] las centrales nucleares ofertan a coste casi 0 (por considerarse no regulables).
		\end{itemize}
		
		
		\textbf{Coordinación hidráulica:} contribuye a la disminución de los costes de generación en horas punta.
	
		\subsection{Problema de optimización.}
			Sea un sistema eléctrico con $m$ generadores y $n$ nudos. La resolución de un \textbf{flujo de carga óptimo} (OPF, Optimal Power Flow) consiste en elegir las potencias $P_{G_i}$ de los $m$ generadores disponibles y los módulos de las tensiones $U_i$ en los $n$ nudos, de forma que se minimice el coste total de generación $C_T$:
			
			\[C_T = \sum_{i = 1}^m C_i\,\left[\dfrac{\euro}{h}\right] \Rightarrow 
			OPF = \min{ \left\{ \sum_{i = 1}^n C_{G_i} (P_{G_i}) \right\} } \]
			
			
			Son problemas de \textbf{optimización lineal}, \textbf{lineal entera mixta} o \textbf{no lineal} cuando se tienen en cuenta las pérdidas eléctricas, sujetos a restricciones. 
			
		\subsection{Restricciones en la optimización.}
			Las potencias generadas por cada grupo deben estar comprendidas entre un \textbf{límite mínimo}, dado por la \textbf{turbina}, y un \textbf{máximo}, dado por el \textbf{generador}.
			
			
			El flujo de potencia activa y reactiva en cada línea no puede superar un valor máximo por motivos de la capacidad de la línea.
			
			
			La tensión y frecuencia en los nudos del sistema no debe quedar fuera de los límites impuestos por la calidad de suministro (en redes de transporte de 220-400 kV).
			
			\begin{figure}
					\begin{minipage}{0.5\textwidth}
					\begin{table}[H]
						\centering
						\begin{tabular}{cc}
							\textbf{Rango de} & \textbf{Periodo de tiempo}\\
							\textbf{tensión} & \textbf{de funcionamiento}\\
							\hline
							$0.85-0.9\,p.u.$ & 60 minutos\\
							$0.9-1.0875\,p.u.$ & Ilimitado\\
							$1.0875-1.1\,p.u.$ & 60 minutos
						\end{tabular}
						%\caption{Tiempos máximos de caídas de tensión en redes de transporte.}
						\label{tab:tiemposCdt}
					\end{table}
				\end{minipage}
				\begin{minipage}{0.5\textwidth}
					\begin{table}[H]
						\centering
						\begin{tabular}{cc}
							\textbf{Rango de} & \textbf{Periodo de tiempo}\\
							\textbf{frecuencias} & \textbf{de funcionamiento}\\
							\hline
							$47.5-48.5\,Hz$ & 30 minutos\\
							$48.5-49.0\,Hz$ & Ilimitado\\
							$49.0-51.0\,Hz$ & Ilimitado\\
							$51.0-51.5\,Hz$ & 30 minutos
						\end{tabular}
						%\caption{Tiempos máximos de caídas de tensión en redes de transporte.}
						\label{tab:tiemposfrecuencias}
					\end{table}	
				\end{minipage}	
			\end{figure}	
			
	\section{Unidades térmicas generadoras. Curva de consumo.}
			Una central térmica generadora queda descrita por:
			\begin{itemize}
				\item Cantidad de calor de entrada requerido: $H_{G_i}\,\left[\dfrac{MJ}{h}\right]$.
				\item Cantidad de energía eléctrica entregada (trifásica): $P_{G_i}\,[MW]$.
			\end{itemize}
			
			\[H_{G_i} = \alpha_i + \beta_i \cdot P_{G_i} + \gamma_i \cdot P_{G_i}^2\,\left[\dfrac{MJ}{h}\right]\]
			
			\begin{itemize}
				\item $\alpha_i$ son los costes fijos $[\euro/h]$ de operación (salarios de los trabajadores) y consumo de combustible de autoconsumo cuando la producción es 0.
				\item $\beta_i$ y $\gamma_i$ son parámetros positivos que caracterizan la dependencia de la curva de costes o de consumo con su nivel de generación.
			\end{itemize}
			
			\begin{figure}[H]
				\begin{minipage}{0.4\textwidth}
					\begin{figure}[H]
						\centering
							\begin{circuitikz}[scale = 0.5]
								\tikzstyle{every node}=[font=\normalsize]
								\draw [->, >=Stealth] (3.5,10.25) -- (3.5,16)node[pos=1,right]{$H_{G_i}\,[MW/h]$};
								\draw [->, >=Stealth] (3.5,10.25) -- (9.5,10.25)node[pos=1,right]{$P_{G_i}\,[MW]$};
								\draw [ color={rgb,255:red,0; green,128; blue,255}, short] (3.5,11.25) .. controls (6.5,11.5) and (8,11.75) .. (9,15.25);
								\draw [dashed] (4.5,11.25) -- (4.5,10.25)node[pos=1,below]{$P_{G_i}^{min}$};
								\draw [dashed] (8.5,13.75) -- (8.5,10.25)node[pos=1,below]{$P_{G_i}^{max}$};
							\end{circuitikz}
						
						\label{fig:my_label}
					\end{figure}
				\end{minipage}
				\begin{minipage}{0.6\textwidth}
					Esta es la función que define la \textbf{curva de consumo}, y expresa la cantidad de combustible consumido por hora en función de la producción eléctrica. Mide la eficiencia de la unidad de producción. Puede ser continua o a tramos.
					
					\vspace{0.25cm}
					Esta curva se obtiene de pruebas que se le realizan al grupo turbina-generador para varios niveles de salida: 100, 75, 50 y 25\%, por ejemplo.
				\end{minipage}
			\end{figure}
	
		\subsection{Característica de coste de las unidades térmicas generadoras.}
			Multiplicando la cantidad de calor $H_{G_i}$ por el coste de combustible se obtiene la función de coste total de cada unidad de producción $C_i(P_{G_i})$, en $\euro/h$ y en función de $P_{G_i}$.
			
			\begin{figure}[H]
				\begin{minipage}{0.4\textwidth}
					\begin{figure}[H]
						\centering
						\begin{circuitikz}[scale = 0.5]
							\tikzstyle{every node}=[font=\normalsize]
							\draw [->, >=Stealth] (3.5,10.25) -- (3.5,16)node[pos=1,right]{$C_i(P_{G_i})\,[\euro/h]$};
							\draw [->, >=Stealth] (3.5,10.25) -- (9.5,10.25)node[pos=1,right]{$P_{G_i}\,[MW]$};
							\draw [ color={rgb,255:red,0; green,128; blue,255}, short] (3.5,11.25) .. controls (6.5,11.5) and (8,11.75) .. (9,15.25);
							\draw [dashed] (4.5,11.25) -- (4.5,10.25)node[pos=1,below]{$P_{G_i}^{min}$};
							\draw [dashed] (8.5,13.75) -- (8.5,10.25)node[pos=1,below]{$P_{G_i}^{max}$};
						\end{circuitikz}
						
						\label{fig:my_label}
					\end{figure}
				\end{minipage}
				\begin{minipage}{0.6\textwidth}
					El coste total incluye un coste fijo (que no contiene coste de inversión), costes de operación fijos, consumos propios y costes variables de O\&M, combustible, emisiones de $CO_2$, etc. Los costes variables son dependientes de la potencia activa que entrega el generador. Las centrales hidráulicas tienen una curva similar en función del caudal turbinado.
					
					\vspace{0.25cm}
					Si la curva de consumo es una función cuadrática admite soluciones analíticas:
					\[C_i(P_{G_i}) = \alpha_i + \beta_i\cdot P_{G_i} + \gamma_i\cdot P_{G_i}^2\]
				\end{minipage}
			\end{figure}
			
		\subsection{Conversión de la curva de consumo específico a curva de costes.}
			$1\,th = 1.162\,kWh = 1000\,kcal$
			
			$1\,kWh = 0.86\,th$
			
			$1\,tep = 11.627\,MWh = 10^4\,th$
			
			$1\,kJ = 9.5\,th$
			
		\subsection*{Ejemplo.}
			Una central térmica de carbón, con un PCI de $6050\,th/t$ tiene un coste de $37.91\,\euro/t$. Obtener la curva de coste del generador.
			\[C_1\,\left[\dfrac{\euro}{h}\right] = \dfrac{H_1\,\left[\dfrac{th}{h}\right]}{6050\,\left[\dfrac{th}{t}\right]} \cdot 37.91\,\left[\dfrac{\euro}{t}\right] = 426.56 + 10.756\cdot P_{G_1} + 0.0031\cdot P_{G_1}^2\]
			
		\subsection{Restricciones técnicas de la curva de coste o consumo.}
			\begin{itemize}
				\item Límites operativos:
				\begin{itemize}
					\item Potencia mínima (por caldera y turbina).
					\item Potencia máxima (por generador síncrono).
				\end{itemize}
				
				\item Tiempo mínimo de operación.
				\item Tiempo mínimo de arranque.
				\item Tiempo de respuesta entre intervalos horarios: Rampa de subida y bajada [MW/h].
				\item Coste de arranque.
				\item Coste de parada.
				\item Penalizaciones por emisiones de $CO_2$.
			\end{itemize}
			
		\subsection{Consumo específico o \textit{Heat Rate} (H.R.).}
			Es la medida del rendimiento de una central termoeléctrica. Se define por el cociente entre la energía térmica aportada en forma de combustible y la energía generada en bornes del generador.
			
			\[HR = \dfrac{H_{G_i}}{P_{G_i}} = \dfrac{\alpha_i}{P_{G_i}} + \beta_i + \gamma_i\cdot P_{G_i}\]
			
			La máxima eficiencia de la unidad se obtiene en el mínimo de la función HR, que se da para valores próximos a la potencia máxima del generador.
			
			\begin{figure}[H]
				\begin{minipage}{0.35\textwidth}
					\begin{figure}[H]
						\centering
						\begin{circuitikz}[scale = 0.65]
							\tikzstyle{every node}=[font=\normalsize]
							\draw [->, >=Stealth] (3.5,10.25) -- (3.5,16)node[pos=1,right]{$H\!R\,\left[\dfrac{MJ}{MWh}\right]$};
							\draw [->, >=Stealth] (3.5,10.25) -- (9.5,10.25)node[pos=1.1,above]{$P_{G_i}\,[MW]$};
							\draw [ color={rgb,255:red,0; green,128; blue,255}, short] (4.5,15.25) .. controls (6,11.5) and (8,11.85) .. (8.5,13.25);
							\draw [dashed] (4.5,15.25) -- (4.5,10.25)node[pos=1,below]{$P_{G_i}^{min}$};
							\draw [dashed] (8.5,13.25) -- (8.5,10.25)node[pos=1,below]{$P_{G_i}^{max}$};
							\draw [dashed] (7.25,12.25) -- (7.25,10.25)node[pos=1,below]{$P_{ef}$};
							\draw [dashed] (7.25,12.25) -- (3.5,12.25);
							\draw (7.25,12.25) to[short, -*] (7.25,12.25);
						\end{circuitikz}
						
						\label{fig:my_label}
					\end{figure}
				\end{minipage}
				\begin{minipage}{0.65\textwidth}
					\begin{figure}[H]
						\centering
						\begin{circuitikz}[scale = 0.5]
							\tikzstyle{every node}=[font=\normalsize]
							\draw [->, >=Stealth] (3.5,10.25) -- (3.5,16)node[pos=1,right]{$\eta\,[\%]$};
							\draw [->, >=Stealth] (3.5,10.25) -- (9.5,10.25)node[pos=1.1,above]{$P_{G_i}\,[MW]$};
							\draw [ color={rgb,255:red,0; green,128; blue,255}, short] (4.25,11.25) .. controls (4.75,13.75) and (6.5,15) .. (8.75,15);
							\draw [->, >=Stealth] (12.5,10.25) -- (12.5,16.25)node[pos=1,right]{$H\!R\,\left[\dfrac{MJ}{MWh}\right]$};
							\draw [->, >=Stealth] (12.5,10.25) -- (18.25,10.25)node[pos=1.1,above]{$P_{G_i}\,[MW]$};
							\draw [ color={rgb,255:red,0; green,128; blue,255}, short] (13.5,15) .. controls (14,12.5) and (15.25,11.5) .. (17.5,11.25);
						\end{circuitikz}
						
						\label{fig:my_label}
					\end{figure}
				\end{minipage}
			\end{figure}
			
			\[\eta = \dfrac{3600}{H\!R}\,\left[\dfrac{MWh_e}{MWh_t}\right] \qquad 
			H\!R = \dfrac{3600}{\eta}\,\left[\dfrac{MJ}{MWh}\right]\]
			
		\newpage
		\subsection*{Ejemplo. Cálculo del HR para una turbina de gas.}
			\textbf{Ojo:} el HR de la turbina determinado mediante pruebas es distinto del $H\!R_{ISO}$ que proporciona el fabricante.
			
			\begin{figure}[H]
				\centering
				\includegraphics[width=0.7\linewidth]{res/tema5/turbinaGas}
				\label{fig:turbinagas}
			\end{figure}
			\vspace{-1cm}
			\begin{figure}[H]
				\begin{minipage}{0.325\textwidth}
					\begin{table}[H]
						\renewcommand{\arraystretch}{1.2}
						\centering
						\begin{tabular}{lcc}
							\multicolumn{3}{c}{\textbf{Datos}}\\
							\hline
							\textbf{Mag.} & \textbf{Uds.} & \textbf{Valor}\\
							\hline
							$T_{amb}$ & $^\circ C$ & 15\\
							\hline
							$P_{atm}$ & $mbar$ & 1013\\
							\hline
							\multirow{2}{*}{$H\!R_{ISO}$} & \multirow{2}{*}{$\dfrac{kJ}{kWh}$} & \multirow{2}{*}{9526}\\
							&&\\
							\hline
							\multirow{2}{*}{PCI} & \multirow{2}{*}{$\dfrac{MJ}{Nm^3}$} & \multirow{2}{*}{35.5}\\
							&&\\
							\hline
							$P_{e,\,neta}$ & \multirow{2}{*}{$MW$} & \multirow{2}{*}{240}\\
							(ISO)&&
						\end{tabular}
					\end{table}
				\end{minipage}
				\begin{minipage}{0.675\textwidth}
					$\eta_{ISO} = \dfrac{3600\,\dfrac{kJ}{kWh}}{H\!R_{ISO}} = \dfrac{3600\,\dfrac{kJ}{kWh}}{9526\,\dfrac{kJ}{kWh}}\cdot 100 = 37.7\%$
				
					$\eta_{ISO} = \dfrac{P_{neta,\,ISO}}{P_{GN}}$
					
					$P_{GN} = \dfrac{240}{0.377} = 636.60\,MW = \text{Vol}_{GN}\,\left[\dfrac{Nm^3}{h}\right] \cdot PCI_{GN}\,\left[\dfrac{MJ}{Nm^3}\right]$
					
					$\text{Vol}_{GN} = \dfrac{636.60}{35.5}\cdot 3600\,\dfrac{s}{h} = 64557.1\,\dfrac{Nm^3}{h}$
				\end{minipage}
			\end{figure}
			
			Energía aportada por el GN: 
			
			$\qquad E = \text{Vol}_{GN} \cdot PCI_{GN} = 64557.1 \cdot 35.5 = 2.292\,\dfrac{GJ}{h}$
			
			\vspace{0.2cm}
			Factores de corrección en la potencia por variación de temperatura y presión atmosférica:
			
			
			Condiciones de funcionamiento de la turbina de gas en la central: $T=16^\circ C$ y $P=900\,mbar$
			
			$\qquad P_{neta,\,ISO} = \dfrac{P_{neta,\,sitio}}{f_T \cdot f_P}$
			
			$\qquad f_P = \dfrac{900}{1013} = 0.888;\quad\quad f_T = \dfrac{15}{16} = 0.9375$
			
			$\qquad P_{sitio} = \dfrac{240\,MW}{0.9375 \cdot 0.888} = 199.8\,MW$
			
			\vspace{0.25cm}
			Heat Rate en el sitio de la instalación:
			
			\vspace{0.2cm}
			$\qquad H\!R_{ISO} = \dfrac{H\!R_{sitio}}{F_{TH}\,(\text{dato})}$
			
			$\qquad H\!R_{sitio} = 9.526 \dfrac{kJ}{kWh}\cdot 1.01 = 9.621.26 \dfrac{kJ}{kWh}$
			
			\vspace{0.25cm}
			Rendimiento:
			
			\vspace{0.2cm}
			$\qquad \eta_{sitio} = \dfrac{3600 \dfrac{kJ}{kWh}}{H\!R_{sitio}} =
				\dfrac{3600 \dfrac{kJ}{kWh}}{9621.26 \dfrac{kJ}{kWh}} \cdot 100 = 37.4\%$
				
			$\qquad \eta_{sitio} = \dfrac{P_{neta,\,sitio}}{P_{GN}}$
			
			$\qquad P_{GN} = \dfrac{199.8}{0.374} = 534.22\,MW = \text{Vol}_{GN}\,\left[\dfrac{Nm^3}{h}\right] \cdot PCI_{GN}\,\left[\dfrac{MJ}{Nm^3}\right]$
			
			\vspace{0.25cm}
			Energía aportada por el GN:
			
			$\qquad E = \text{Vol}_{GN} \cdot PCI_{GN} = 54174.88 \cdot 35.5 = 1.923\,\dfrac{GJ}{h}$
	
			\newpage
			
		\subsection*{Ejemplo. Cálculo del HR de dos unidades térmicas de carbón.}
			Sean dos unidades térmicas que queman carbón con sus respectivas curvas de consumo
			específico y limites de funcionamiento:
			
			\begin{figure}[H]
				\begin{minipage}{0.6\textwidth}
					\begin{figure}[H]
						\centering
						\begin{tikzpicture}
							\begin{axis}[
								xlabel={$P_{G_i}$},
								ylabel={$H_i$},
								xmin=0, xmax=210,
								ymin=0, ymax=3200,
								xtick={0,20,40,60,80,100,120,140,160,180,200,220},
								ytick={0,500,1000,1500,2000,2500,3000,3500},
								legend pos=north west,
								ymajorgrids=true,
								grid style=dashed,
								]
								
								\addplot[
								color=blue,
								]
								coordinates {
									(20,289.6)
									(30,401.6)
									(40,518.4)
									(50,640)
									(60,766.4)
									(70,897.6)
									(80,1033.6)
									(90,1174.4)
									(100,1320)
									(110,1470.4)
									(120,1625.6)
									(130,1785.6)
									(140,1950.4)
									(150,2120)
									(160,2294.4)
									(170,2473.6)
									(180,2657.6)
									(190,2846.4)
									(200,3040)
								};
								\addlegendentry{G1}
								
								\addplot[
								color=red,
								]
								coordinates {
									(20,256)
									(30,336)
									(40,424)
									(50,520)
									(60,624)
									(70,736)
									(80,856)
									(90,984)
									(100,1120)
									(110,1264)
									(120,1416)
									(130,1576)
									(140,1744)
									(150,1920)
									(160,2104)
									(170,2296)
									(180,2496)
									(190,2704)
									(200,2920)						
								};
								\addlegendentry{G2}
								
							\end{axis}
						\end{tikzpicture}
					\end{figure}
				\end{minipage}
				\begin{minipage}{0.4\textwidth}
					\begin{table}[H]
						\begin{tabular}{c}
							$H_1(P_{G_1}) = 80 + 10P_{G_1} + 0.024P_{G_1}^2\,\left[\dfrac{GJ}{h}\right]$\\ 
							$20\,MW \leq P_{G_1} \leq 100\,MW$\\
							\\
							\hline
							\\
							$H_2(P_{G_2}) = 120 + 6P_{G_2} + 0.04P_{G_2}^2\,\left[\dfrac{GJ}{h}\right]$\\
							$20\,MW \leq P_{G_2} \leq 100\,MW$
						\end{tabular}
					\end{table}
				\end{minipage}
			\end{figure}
			
			\begin{figure}[H]
				\begin{minipage}{0.6\textwidth}
					\begin{figure}[H]
						\centering
						\begin{tikzpicture}
							\begin{axis}[
								xlabel={$P_{G_i}$},
								ylabel={$H\!R$},
								xmin=0, xmax=210,
								ymin=8, ymax=20,
								xtick={0,20,40,60,80,100,120,140,160,180,200,220},
								ytick={8,10,12,14,16,18},
								legend pos=north west,
								ymajorgrids=true,
								grid style=dashed,
								]
								
								\addplot[
								color=blue,
								]
								coordinates {
									(20	,14.48)
									(30	,13.38666667)
									(40	,12.96)
									(50	,12.8)
									(60	,12.77333333)
									(70	,12.82285714)
									(80	,12.92)
									(90	,13.04888889)
									(100,	13.2)
									(110,	13.36727273)
									(120,	13.54666667)
									(130,	13.73538462)
									(140,	13.93142857)
									(150,	14.13333333)
									(160,	14.34)
									(170,	14.55058824)
									(180,	14.76444444)
									(190,	14.98105263)
									(200,	15.2)
								};
								\addlegendentry{G1}
								
								\addplot[
								color=red,
								]
								coordinates {
									(20	,12.8)
									(30	,11.2)
									(40	,10.6)
									(50	,10.4)
									(60	,10.4)
									(70	,10.51428571)
									(80	,10.7)
									(90	,10.93333333)
									(100,	11.2)
									(110,	11.49090909)
									(120,	11.8)
									(130,	12.12307692)
									(140,	12.45714286)
									(150,	12.8)
									(160,	13.15)
									(170,	13.50588235)
									(180,	13.86666667)
									(190,	14.23157895)
									(200,	14.6)
									
								};
								\addlegendentry{G2}
								
							\end{axis}
						\end{tikzpicture}
					\end{figure}
				\end{minipage}
				\begin{minipage}{0.4\textwidth}
					El Heat Rate será la relación entre el consumo de energía (GJ) y la energía producida (MWh).
					
					\vspace{0.25cm}
					Se observa que en $P_G \approx 50\,MW$ se alcanzan los puntos de HR mínimo.
				\end{minipage}
			\end{figure}		
		
		\begin{figure}[H]
			\begin{minipage}{0.6\textwidth}
				\begin{figure}[H]
					\centering
					\begin{tikzpicture}
						\begin{axis}[
							xlabel={$P_{G_i}$},
							ylabel={$\eta$ (\%)},
							xmin=0, xmax=210,
							ymin=20, ymax=40,
							xtick={0,20,40,60,80,100,120,140,160,180,200,220},
							ytick={20,25,30,35,40},
							legend pos=north west,
							ymajorgrids=true,
							grid style=dashed,
							]
							
							\addplot[
							color=blue,
							]
							coordinates {
								(20	,24.86187845)
								(30	,26.89243028)
								(40	,27.77777778)
								(50	,28.125)
								(60	,28.18371608)
								(70	,28.07486631)
								(80	,27.86377709)
								(90	,27.58855586)
								(100,	27.27272727)
								(110,	26.93144723)
								(120,	26.57480315)
								(130,	26.20967742)
								(140,	25.84085316)
								(150,	25.47169811)
								(160,	25.10460251)
								(170,	24.74126779)
								(180,	24.38290187)
								(190,	24.03035413)
								(200,	23.68421053)
								
							};
							\addlegendentry{G1}
							
							\addplot[
							color=red,
							]
							coordinates {
								(20	,28.125      )
								(30	,32.14285714 )
								(40	,33.96226415 )
								(50	,34.61538462 )
								(60	,34.61538462 )
								(70	,34.23913043 )
								(80	,33.64485981 )
								(90	,32.92682927 )
								(100,	32.14285714)
								(110,	31.32911392)
								(120,	30.50847458)
								(130,	29.69543147)
								(140,	28.89908257)
								(150,	28.125     )
								(160,	27.37642586)
								(170,	26.65505226)
								(180,	25.96153846)
								(190,	25.29585799)
								(200,	24.65753425)
								
								
							};
							\addlegendentry{G2}
							
						\end{axis}
					\end{tikzpicture}
				\end{figure}
			\end{minipage}
			\begin{minipage}{0.4\textwidth}
				Como un $1\,MWh$ equivale a $3.6\,GJ$ se pueden calcular las eficiencias de cada unidad, dividiendo la
				equivalencia entre el valor del HR:
				
				\[\eta\,(\%) = \dfrac{3.6\,\left[\dfrac{GJ}{MWh}\right]\cdot P_{G_i}\,[MW]}{H_i\,\left[\dfrac{GJ}{h}\right]}\]
			\end{minipage}
		\end{figure}
		
	\newpage
		
	\section{Centrales hidráulicas. Curvas de coste y consumo.}
		\subsection{Curva de consumo específico de una central hidráulica de embalse.}
			\begin{figure}[H]
				\begin{minipage}{0.5\textwidth}
					Volumen de agua a desembalsar en un periodo considerado:
					\[P_h = P_h(Q,\,H)\,[MW]\]
					
					$Q \, \left[\dfrac{m^3}{s}\right]$ se puede mantener constante durante un periodo de tiempo.
					
					$H \, \left[m\right]$: salto. Variable a medida que se vacía el embalse.
					
					\vspace{0.5cm}
					Consumo de caudal en función del la potencia:
					\[Q(P_h) = \gamma_h P_h^2 + \beta_h P_h + \alpha_h\,\left[\dfrac{m^3}{h}\right]\]
					
					Restricciones:
					\begin{itemize}
						\item Régimen de lluvias.
						\item Potencia mínima técnica y potencia máxima: $P_{min} \leq P_h \leq P_{max}\,[MW]$
						\item Restricciones en el uso de los embalses.
						\item Tamaño de los embalses y acoplamiento entre ellos.
						\item Volumen de agua almacenado.
					\end{itemize}
				\end{minipage}
				\begin{minipage}{0.5\textwidth}
					\begin{figure}[H]
						\centering
						\begin{circuitikz}[scale = 0.95]
							\tikzstyle{every node}=[font=\normalsize]
							\draw [->, >=Stealth] (6.75,13.25) -- (6.75,17)node[pos=1,right]{$Q\,\left[\dfrac{m^3}{h}\right]$};
							\draw [->, >=Stealth] (6.75,13.25) -- (11.25,13.25)node[pos=1,above]{$P\,[MW]$};
							\draw [ color={rgb,255:red,0; green,128; blue,255}, short] (7.25,14.5) .. controls (10,15.25) and (9.75,15.5) .. (10.25,16.25);
							\draw [ color={rgb,255:red,0; green,128; blue,255}, dashed] (7.25,14.25) .. controls (10,15) and (9.75,15.5) .. (10.25,16);
							\draw [ color={rgb,255:red,0; green,128; blue,255}, dashed] (7.25,14) .. controls (10.5,15.25) and (9.75,15.25) .. (10.25,15.75);
							\draw [dashed] (10.25,16.25) -- (10.25,13.25)node[pos=1,below]{$P_{max}$};
							\draw [dashed] (7.25,14.5) -- (7.25,13.25)node[pos=1,below]{$P_{min}$};
							\node [font=\normalsize, rotate around={-360:(0,0)}] at (8.75,12.5) {Curva de entrada/salida};
							\draw [ color={rgb,255:red,0; green,128; blue,255}, ->, >=Stealth] (9.9,16.5) -- (9.75,15.5)node[pos=0,above]{Altura de salto fija};
							\draw [ color={rgb,255:red,0; green,128; blue,255}, ->, >=Stealth] (8.2,15.5) -- (7.75,14.2)node[pos=0,above]{Altura variable};
							\draw [ color={rgb,255:red,0; green,128; blue,255}, ->, >=Stealth] (8.2,15.5) -- (8,14.5);
							\draw [ color={rgb,255:red,0; green,128; blue,255}, ->, >=Stealth] (8.2,15.5) -- (8.25,14.8);
						\end{circuitikz}
						
						\label{fig:my_label}
					\end{figure}
					\begin{figure}[H]
						\centering
						\begin{circuitikz}[scale = 0.95]
							\tikzstyle{every node}=[font=\normalsize]
							\draw [->, >=Stealth] (6.75,13.25) -- (6.75,17)node[pos=1,right]{$C_{incr}\,\left[\dfrac{m^3}{MWh}\right]$};
							\draw [->, >=Stealth] (6.75,13.25) -- (11.25,13.25)node[pos=1,above]{$P\,[MW]$};
							\draw [ color={rgb,255:red,0; green,128; blue,255}, short] (7.25,14.5) .. controls (9.5,14.75) and (9.75,15) .. (10.25,15.25);
							\draw [dashed] (10.25,15.25) -- (10.25,13.25)node[pos=1,below]{$P_{max}$};
							\draw [dashed] (7.25,14.5) -- (7.25,13.25)node[pos=1,below]{$P_{min}$};
							\node [font=\normalsize, rotate around={-360:(0,0)}] at (8.75,12.5) {Consumos marginales};
						\end{circuitikz}
						
						\label{fig:my_label}
					\end{figure}
				\end{minipage}
			\end{figure}
			
		\subsection{Curva de coste de una central hidráulica de embalse.}
			El volumen de agua embalsada corresponde al \textbf{coste de oportunidad} del sistema. Reservan el agua para turbinarla
			en horas punta, y la central cobra según el precio fijado en esa hora.
			
			
			Para un periodo de operación de una hora, el coste asociado al volumen de agua turbinado queda
			expresado por una ecuación cuadrática:
			\[C(V) = \gamma_h V^2 + \beta_h V + \alpha_h\,\left[\dfrac{\euro}{h}\right]\]
			
			Dado que la central hidráulica genera por hora $G \, \left[\dfrac{MW}{m^3/s}\right]$, es posible convertir el costo de operación en
			función de la potencia generada:
			
			\[Q\,\left[\dfrac{m^3}{h}\right] = \dfrac{V\,[m^3]}{3600\,\left[\dfrac{s}{h}\right]}\]
			
			
			Que generará una potencia:
			
			\[P_h\,[MW] = G\,\left[\dfrac{MW}{m^3/s}\right]\cdot \dfrac{V}{3600} \left[\dfrac{m^3}{s}\right]\]
			
			Y despejando y sustituyendo queda:
			
			\[C_h(V) = \gamma_h \cdot \Biggl(\dfrac{3600\cdot P_h\,[MW]}{G\,\left[\dfrac{MW}{m^3/s}\right]}\Biggl)^2 + \beta_h \cdot 
			\dfrac{3600\cdot P_h\,[MW]}{G\,\left[\dfrac{MW}{m^3/s}\right]} + \alpha_h\]
			
			Las centrales hidráulicas tienen unos costes fijos de operación $\alpha_h$ muy bajos $\left(\sim3\,\dfrac{\euro}{MWh}\right)$
			
		\subsection*{Ejemplo. Expresión de coste en función de la potencia.}
			Se conoce la curva de costes de una central hidráulica, donde se desprecia $\alpha_h$ y se generan $0.9\,\dfrac{MW}{m^3/s}$
			
			\[C_h(V) = 3.125\cdot 10^{-9}\,V^2 + 0.001625\,V\,\left[\dfrac{\euro}{h}\right]\]
			
			La expresión de coste en función de la potencia queda:
			\[V = \dfrac{3600\cdot P_h}{G} = 4000\cdot P_h\]
			\[C_h(P_h) = 0.05\,P_h^2 + 6.5\,P_h\,\dfrac{\euro}{h}\]
			
		\subsection{Central hidráulica reversible. Curva de consumo específico.}
			Consiste en turbinar en periodos de alto coste y bombear en bajo coste. Emplean turbinas reversibles y pueden funcionar con ciclo diario o semanal. Disponen de un salto constante y tienen un rendimiento constante:
			
			\[\eta = \dfrac{E_T}{E_B}\cdot 100 \approx 67\%\]
			
			\begin{figure}[H]
				\begin{minipage}{0.6\textwidth}
					\begin{figure}[H]
						\centering
						\begin{circuitikz}[scale = 0.9]
							\tikzstyle{every node}=[font=\normalsize]
							\draw [->, >=Stealth] (6.75,13.25) -- (6.75,17)node[pos=1,right]{$Q\,\left[\dfrac{m^3}{h}\right]$};
							\draw [->, >=Stealth] (2.25,13.25) -- (11.25,13.25)node[pos=1,above]{$P\,[MW]$};
							\draw [ color={rgb,255:red,0; green,128; blue,255}, short] (7.25,14.5) .. controls (10,15.5) and (9.75,15.5) .. (10.25,16.25);
							\draw [dashed] (10.25,16.25) -- (10.25,13.25)node[pos=1,below]{$P_{max}$};
							\draw [dashed] (7.25,14.5) -- (7.25,13.25)node[pos=1,below]{$P_{min}$};
							\node [font=\normalsize, rotate around={-360:(0,0)}] at (6.75,12.5) {Curva de entrada/salida};
							\draw [ color={rgb,255:red,0; green,128; blue,255}, short] (2.75,15.75) -- (3.25,15.5);
							\draw [dashed] (3,15.5) -- (3,13.25)node[pos=1,below]{$P_b$};
						\end{circuitikz}
						
						\label{fig:my_label}
					\end{figure}
				\end{minipage}
				\begin{minipage}{0.4\textwidth}
					Generación (potencia variable): 
					
					\[Q(P_h) = \gamma_h P_h^2 + \beta_h P_h + \alpha_h\,\left[\dfrac{m^3}{h}\right]\]
					
					Bombeo (potencia constante): 
					
					\[Q(P_b)\,\left[\dfrac{m^3}{h}\right]\]
				\end{minipage}
			\end{figure}
			
	
	\section{Coste total de producción del sistema eléctrico.}
		El coste total de producción de un sistema con $n$ generadores es la suma de los costes individuales:
		\[C(P_G) = \sum_{i=1}^{n} C_i(P_{G_i})\]
		
		Si la demanda total es $P_D^{\text{total}}$ y todos los generadores participan en el despacho económico,
		\[\sum_{i=1}^{n} P_{G_i} = P_D^{\text{total}} + P_{\text{pérdidas}}\]
		
		El despacho básico consiste en minimizar el coste de producción con respecto a las generaciones.
		\[\text{Objetivo: } \min{\left\{C_t = \sum_{i=1}^N C_i\right\}}\]
		
		\newpage
		\subsection{Restricciones en la producción de energía eléctrica.}
			\begin{itemize}
				\item Balance nodal: flujo de potencia, estabilidad y mantenimiento de las tensiones en los nudos.
				\item Restricciones de desigualdad en los generadores: $P_{G_i}^{\text{min}} \leq P_{G_i} \leq P_{G_i}^{\text{max}}$
				\item Límites operativos: capacidad de las líneas de transmisión.
				\item Restricciones de los sistemas hidráulicos: volumen de agua embalsada.
				\item Reserva rodante disponible.
				\item Mínimas pérdidas de energía activa en la red: $RI^2$ y $XI^2$.
				\item Límites de emisiones contaminantes.
				\item Consumo de una cuota mínima contratada de combustible (centrales de gas natural y carbón).
				\item Tiempos de respuesta ante una subida o bajada de cada generador.
				\item Costes de arranque y parada.
			\end{itemize}
			
		\subsection{Relajación Lagrangiana: optimización de funciónes sujetas a restricciones.}
			Sirve para encontrar los máximos y mínimos de funciones de
			varias variables sujetas a restricciones. Permite resolver el problema de coordinación hidrotérmica a
			corto plazo. Reduce el problema restringido con $n$ variables a uno sin restricciones de $n + k$ variables,
			donde $k$ es el número de restricciones.
			
			
			El objetivo es minimizar la función de coste $f(x_1, x_2,\dots x_n)$ sujeta a condiciones de igualdad $g_i(x_1,x_2,\dots x_n) = 0$ y desigualdad $g_i(x_1,x_2,\dots x_n) \leq 0$.
			
			
			Se crea una función disminuida introduciendo un vector de $k$ elementos $\lambda$:
			\[L = f - \sum_{i=1}^{k} \lambda_i g_i\]
			
			Los valores de $x_1,x_2,\dots x_n$ que minimizan $f$ sujeto a funciones $g$ son los que resuelven las ecuaciones
			\[\dfrac{\partial L}{\partial x_i} = \dfrac{\partial f}{\partial x_i} - \sum_{i=1}^{k} \lambda_i \dfrac{\partial g_i}{\partial x_i} = 0\]
			\[\dfrac{\partial L}{\partial \lambda_i} = g_i = 0\]
			
			\subsubsection{Propiedad del vector gradiente.}
				El vector gradiente en un punto es un vector perpendicular a la curva de nivel.
				Cuando las curvas de nivel de las funciones $f$ (función de costes) y $g$ (función de restricciones) son
				tangentes se consigue el punto óptimo de mínimo coste, y sus vectores gradientes son paralelos
				(tienen la misma dirección).
				
				\begin{figure}[H]
					\begin{minipage}{0.5\textwidth}
						\begin{figure}[H]
							\centering
							\begin{circuitikz}
								\tikzstyle{every node}=[font=\normalsize]
								\draw [->, >=Stealth] (6.75,13.25) -- (6.75,17)node[pos=1,right]{y};
								\draw [->, >=Stealth] (6.75,13.25) -- (11.25,13.25)node[pos=1,above]{x};
								\draw [ color={rgb,255:red,0; green,128; blue,255} , rotate around={61:(9.25,15.5)}] (9.25,15.5) ellipse (1.5cm and 2cm);
								\draw [ color={rgb,255:red,0; green,128; blue,255} , rotate around={61:(9,15.75)}] (9,15.75) ellipse (1cm and 1.5cm);
								\draw [ color={rgb,255:red,0; green,128; blue,255} , rotate around={61:(8.5,16)}] (8.5,16) ellipse (0.5cm and 0.75cm);
								\draw [ color={rgb,255:red,255; green,0; blue,0}, short] (11.25,17) .. controls (8.25,16.25) and (9.5,14.25) .. (7.5,13.25);
								\draw [ color={rgb,255:red,0; green,128; blue,255}, ->, >=Stealth] (9.25,15.5) -- (8.5,16);
								\draw [ color={rgb,255:red,255; green,0; blue,0}, ->, >=Stealth] (9.25,15.5) -- (10,15);
								\draw [ color={rgb,255:red,0; green,128; blue,0} , rotate around={61:(9.25,15.5)}, dashed] (9.25,15.5) ellipse (0.5cm and 1cm);
								\draw [ color={rgb,255:red,0; green,128; blue,0}, dashed] (9.25,16) -- (9.75,17.5);
								\node [font=\normalsize, color={rgb,255:red,0; green,128; blue,0}, rotate around={-360:(0,0)}] at (9.75,18.3) {Los vectores gradientes en el};
								\node [font=\normalsize, color={rgb,255:red,0; green,128; blue,0}, rotate around={-360:(0,0)}] at (9.75,18) {punto de tangencia están orientados};
								\node [font=\normalsize, color={rgb,255:red,0; green,128; blue,0}, rotate around={-360:(0,0)}] at (9.75,17.7) {en la misma dirección};
								\draw [ color={rgb,255:red,0; green,128; blue,255}, ->, >=Stealth] (8,15.75) -- (8.25,16);
								\draw [ color={rgb,255:red,0; green,128; blue,255}, ->, >=Stealth] (9,16.25) -- (8.5,16.25);
								\draw [ color={rgb,255:red,0; green,128; blue,255}, ->, >=Stealth] (8,16.5) -- (8.25,16.25);
								\draw [ color={rgb,255:red,0; green,128; blue,255}, ->, >=Stealth] (10.25,15) -- (10,15.5);
								\draw [ color={rgb,255:red,0; green,128; blue,255}, ->, >=Stealth] (9.75,16.5) -- (9.5,16.25);
								\draw [ color={rgb,255:red,0; green,128; blue,255}, ->, >=Stealth] (8,16.75) -- (8.25,16.5);
								\draw [ color={rgb,255:red,0; green,128; blue,255}, ->, >=Stealth] (8,15.25) -- (8.25,15.5);
								\draw [ color={rgb,255:red,0; green,128; blue,255}, ->, >=Stealth] (7.5,15.25) -- (7.75,15.5);
								\draw [ color={rgb,255:red,0; green,128; blue,255}, ->, >=Stealth] (10.25,14) -- (10,14.4);
								\draw [ color={rgb,255:red,0; green,128; blue,255}, ->, >=Stealth] (11,15.75) -- (10.5,15.5);
								\draw [ color={rgb,255:red,255; green,0; blue,0}, ->, >=Stealth] (8.5,14.25) -- (9,14);
								\draw [ color={rgb,255:red,255; green,0; blue,0}, ->, >=Stealth] (10,16.5) -- (10.25,16);
								\node [font=\footnotesize, color={rgb,255:red,255; green,0; blue,0}, rotate around={-360:(0,0)}] at (11.5,16.75) {$g(x,y)=c$};
								\node [font=\footnotesize, color={rgb,255:red,0; green,128; blue,255}, rotate around={-360:(0,0)}] at (9.9,13.75) {$f(x,y)=d_1$};
								\node [font=\footnotesize, color={rgb,255:red,0; green,128; blue,255}, rotate around={-360:(0,0)}] at (9.5,14.5) {$f(x,y)=d_2$};
								\node [font=\footnotesize, color={rgb,255:red,0; green,128; blue,255}, rotate around={-360:(0,0)}] at (7.6,14) {$f(x,y)=d_3$};
								\draw [ color={rgb,255:red,0; green,128; blue,255}, ->, >=Stealth, dashed] (7.5,14.25) -- (8.5,15.5);
							\end{circuitikz}
							
							\label{fig:my_label}
						\end{figure}
					\end{minipage}
					\begin{minipage}{0.5\textwidth}
						Cuando dos vectores apuntan en la misma dirección pero con distinto sentido	podemos multiplicar cualquiera de los dos por una constante $\lambda_0$ para obtener el otro.
						
						
						Si el punto $(x_0,\,y_0)$ cumple la condición de tangencia, entonces
						\[\vec \nabla f(x_0,\,y_0) = \lambda_0 \vec \nabla g(x_0,\,y_0)\]
						\[\lambda_0 = \text{multiplicador de Lagrange}\]
					\end{minipage}
				\end{figure}
				
			\subsubsection{Función lagrangiana.}
				\[L(x,y,\lambda) = f(x,y) - \lambda(g(x,y)-c)\]
				
				Condiciones de mínimo necesarias: $\vec \nabla L = 0 \Rightarrow$ puntos críticos de $L$ (mínimos o máximos).
				
			\subsubsection{Coste incremental de la generación.}
				Derivando la expresión de la curva de coste con respecto a $P_{G_i}$:
				\[C_i(P_{G_i}) = \alpha_i + \beta_i P_{G_i} + \gamma_i P_{G_i}^2\,\left[\dfrac{\euro}{h}\right]\]
				
				Se obtiene el valor del coste incremental $\lambda$, que corresponde a la ecuación de una recta de pendiente $2\gamma_i$ y ordenada en el origen $\beta_i$.
				\[CI_i(P_{G_i}) = \dfrac{dC_i(P_{G_i})}{dP_{G_i}} = \beta_i + 2\gamma_i P_{G_i} = \lambda\,\left[\dfrac{\euro}{MWh}\right]\]
				
				
				\begin{figure}[H]
					\begin{minipage}{0.65\textwidth}
						El objetivo es conseguir el mismo coste incremental $\lambda$ para todos los generadores. No siempre se podrá cumplir, sobre todo si los generadores trabajan en su límite de funcionamiento y el sistema tiene pérdidas.
						
						\vspace{0.25cm}
						El valor común de $\lambda$ será el coste total o marginal óptimo para ese tramo horario:
						\[\lambda = \dfrac{dC(P_G)}{dP_D^{total}}\, \left[\dfrac{\euro}{MWh}\right]\]
					\end{minipage}
					\begin{minipage}{0.35\textwidth}
						\begin{figure}[H]
							\centering
							\begin{circuitikz}[scale = 0.7]
								\tikzstyle{every node}=[font=\normalsize]
								\draw [->, >=Stealth] (5,10.75) -- (5,14.75);
								\draw [->, >=Stealth] (5,10.75) -- (9.5,10.75)node[pos=1,above]{$P_{G_i}$};
								\draw [ color={rgb,255:red,255; green,0; blue,0}, dashed] (5,12.75) -- (9.5,12.75)node[pos=0,left]{$\lambda$};
								\draw [ color={rgb,255:red,0; green,128; blue,255}, short] (5.25,12) -- (7.25,13.5)node[pos=1,above]{$G_A$};
								\draw [ color={rgb,255:red,0; green,128; blue,255}, short] (6.5,11.25) -- (8.25,14.75)node[pos=1,above]{$G_B$};
								\draw [ color={rgb,255:red,0; green,128; blue,255}, short] (6.75,11.25) -- (10,14)node[pos=1,above]{$G_C$};
								\draw [dashed] (6.25,12.75) -- (6.25,10.75)node[pos=1,below]{$P_A$};
								\draw [dashed] (7.25,12.75) -- (7.25,10.75)node[pos=1,below]{$P_B$};
								\draw [dashed] (8.5,12.75) -- (8.5,10.75)node[pos=1,below]{$P_C$};
							\end{circuitikz}
							
							\label{fig:my_label}
						\end{figure}
					\end{minipage}
				\end{figure}
				
				El coste total de producción de un sistema se reduce al mínimo cuando todas las unidades generadoras se cargan de modo que sus costes marginales son los mismos: $\lambda_1 = \lambda_2 = \dots = \lambda_n = \lambda$
				
				
			\subsubsection{Despacho económico óptimo.}
				Si dos generadores térmicos trabajan con costes incrementales distintos $(\lambda_1 \neq \lambda_2)$, siempre es posible realizar la reasignación de las potencias de un generador a otro obteniendo una disminución del coste total de generación.
				
				
				Si, por ejemplo, $\lambda_1 < \lambda_2$, ante un incremento de potencia el generador 1 será más económico y por tanto su incremento de potencia será mayor.
				
				\begin{figure}[H]
					\begin{minipage}{0.5\textwidth}
						\begin{figure}[H]
							\centering
							\begin{circuitikz}[scale = 0.75]
								\tikzstyle{every node}=[font=\normalsize]
								\draw [->, >=Stealth] (5,10.75) -- (5,14.75)node[pos=1,above]{$C_1(P_1)$};
								\draw [->, >=Stealth] (5,10.75) -- (9.5,10.75)node[pos=1,above]{$P_{G_1}$};
								\draw [ color={rgb,255:red,0; green,128; blue,255}, short] (6,12) .. controls (7.25,12.25) and (8.25,13) .. (9,14);
								\draw [short] (6,12) -- (7,12)node[pos=0.5,below]{$\Delta$ $P_1$};
								\draw [short] (6,12) -- (7,12.25);
								\draw [short] (7,12.25) -- (7,12)node[pos=0.5,right]{$\Delta$ $C_1$};
								\node [font=\normalsize, color={rgb,255:red,0; green,128; blue,255}] at (7,13.75) {$\lambda_1 = \dfrac{\Delta C_1}{\Delta P_1}$};
							\end{circuitikz}
							
							\label{fig:my_label}
						\end{figure}
					\end{minipage}
					\begin{minipage}{0.5\textwidth}
						\begin{figure}[H]
							\centering
							\begin{circuitikz}[scale = 0.75]
								\tikzstyle{every node}=[font=\normalsize]
								\draw [->, >=Stealth] (5,10.75) -- (5,14.75)node[pos=1,above]{$C_2(P_2)$};
								\draw [->, >=Stealth] (5,10.75) -- (9.5,10.75)node[pos=1,above]{$P_{G_2}$};
								\draw [ color={rgb,255:red,0; green,128; blue,255}, short] (6,12) .. controls (6.75,12.5) and (7.5,13) .. (7.75,14.5);
								\draw [short] (6,12) -- (7,12)node[pos=0.5,below]{$\Delta$ $P_2$};
								\draw [short] (6,12) -- (7,12.6);
								\draw [short] (7,12.6) -- (7,12)node[pos=0.5,right]{$\Delta$ $C_2$};
								\node [font=\normalsize, color={rgb,255:red,0; green,128; blue,255}] at (9,13.75) {$\lambda_2 = \dfrac{\Delta C_2}{\Delta P_2}$};
							\end{circuitikz}
							
							\label{fig:my_label}
						\end{figure}
					\end{minipage}
				\end{figure}
				
				Si existe una solución óptima, si la demanda varía una cantidad diferencial $dP_D^{total}$ los niveles de generación también varían de forma que exista un equilibrio:
				\[\sum_{i=1}^n P_{G_i} = P_D^{total}\]
				\[dC(P_G) = \sum_{i=1}^n dC_i(P_{G_i}) = \sum_{i=1}^n CI_i(P_{G_i})\,dP_{G_i} = \sum_{i=1}^n \lambda\, dP_{G_i} = \lambda \sum_{i=1}^n dP_{G_i} = \lambda dP_D^{total}\]
				
				
				Luego de las expresiones anteriores se deduce que $\uparrow P_G \Rightarrow \uparrow \lambda$ de forma lineal con la demanda, que se reparte de forma distinta entre todos los generadores.
				
		\subsection{Formulación general del problema de optimización.}
			\[C(P_G) = \sum_{i=1}^n C_i(P_{G_i})\]
			\[F.O. = \min{\left\{\sum_{i=1}^n C_{G_i}(P_{G_i})\right\}}\]
				
			Restricciones de igualdad y desigualdad:
			\[\sum_{i=1}^n P_{G_i} = P_D^{total} + P_{perd}\]
			\[P_{G_i}^{min} \leq P_{G_i} \leq P_{G_i}^{max}\]
			
			Función lagrangiana:
			\[L = \sum_{i=1}^{n} C_i(P_{G_i}) - \lambda \left(\sum_{i=1}^{g} P_{G_i} P_D\right)\]

	\section{Modelos de estudio del despacho económico.}
		\subsection{Modelo simple. Despacho económico sin pérdidas en la red y sin límites en la generación.}
			La función lagrangiana es:
			\[L(P_G,\,\lambda) = \sum_{i=1}^{n} C_i(P_{G_i}) - \lambda \left(\sum_{i=1}^{g} P_{G_i} P_D\right)\]
		
			Las condiciones necesarias de primer orden apra la solución del óptimo son:
			\[\dfrac{\partial L(P_{G_i},\,\lambda)}{\partial P_{G_i}} = CI_i(P_{G_i}) - \lambda = 0 \qquad \because CI_i(P_{G_i}) = \lambda\]
			
			\[\dfrac{\partial L(P_{G_i},\,\lambda)}{\partial \lambda} = - \sum_{i=1}^{n} P_{G_i} + P_D^{total} = 0 \qquad\because P_D^{total} = \sum_{i=1}^{n} P_{G_i}\]
			
		\subsection*{Ejemplo. Modelo simple de despacho económico.}	
			Determinar la potencia que debe entregar cada generador, el coste incremental y el coste total.
			
			
			\underline{Datos:}
			\begin{itemize}
				\item Curvas de coste de funcionamiento:
				\[C_1(P_{G_1}) = 900 + 45\, P_{G_1} + 0.01\, P_{G_1}^2\,\dfrac{UM}{h}\]
				\[C_2(P_{G_2}) = 2500 + 43\, P_{G_2} + 0.003\, P_{G_2}^2\,\dfrac{UM}{h}\]
				
				\item Carga total a suministrar:
				\[P_D = 700\,MW\]
			\end{itemize}
			
			
			\underline{Solución:}
			
			
			Se debe cumplir que $\lambda = CI_1 = CI_2$ y que $P_{G_1} + P_{G_2} = P_D$:
			\[
			\left.
			\begin{matrix}
				\lambda = CI_1 = 45+0.02\,P_1\\
				\lambda = CI_2 = 43+0.006\,P_2\\
				P_1 + P_2 = 700
			\end{matrix}
			\right\}
			\begin{matrix}
				P_1 = 84.6\,MW\\
				P_2 = 615.4\,MW\\
				\lambda = 46.69\,\dfrac{UM}{MWh}
			\end{matrix}
			\]
			
			Coste total de operación mínimo:
			\[C_T = C_1 + C_2 = 900 + 45\,P_1 + 0.01\,P_1^2+2500+43\,P_2+0.003\,P_2^2 = 34876.92\,\dfrac{UM}{h}\]
		
		\subsection{Modelo sin pérdidas y con límites de generación.}
			El mínimo teórico de la turbina y la potencia máxima del generador aportan restricciones adicionales al problema de optimización, que en la práctica resulta en la no igualdad de costes incrementales cuando algún generador alcanza el límite. En consecuencia, para una demanda concreta podrán coexistir en despacho económico generadores trabajando a un coste incremental superior al del sistema (con potencia mínima), coste incremental inferior al del sistema (con potencia máxima) y coste incremental igual al del sistema.
			
			
			La función lagrangiana es:
			\[L(P_G,\lambda) = \sum_{i=1}^{n} C_i(P_{G_i}) - \lambda \left(\sum_{i=1}^{n} P_{G_i} - P_D^{total}\right) - \sum_{i=1}^{n} \mu_i^{max} (P_{G_i} - P_{G_i}^{max}) - \sum_{i=1}^{n} \mu_i^{min} (P_{G_i} - P_{G_i}^{min})\]
			
			Condiciones necesarias de primer orden para la solución óptima:
			\[\dfrac{\partial L(P_{G_i},\lambda)}{\partial P_{G_i}} = CI_i(P_{G_i})-\lambda - \mu_i^{max}-\mu_i^{min} = 0\]
				
			\[\dfrac{\partial L(P_{G_i},\lambda)}{\partial \lambda} = -\sum_{i=1}^{n} P_{G_i} + P_D^{total} = 0\]
			
			
			Restricciones de desigualdad:
			
			$\mu_i^{max} \leq 0$ si $P_{G_i} = P_{G_i}^{max}$
			
			$\mu_i^{max} =    0$ si $P_{G_i} < P_{G_i}^{max}$
			
			$\mu_i^{min} \geq 0$ si $P_{G_i} = P_{G_i}^{min}$
			
			$\mu_i^{min} =    0$ si $P_{G_i} > P_{G_i}^{min}$
			
			
			\vspace{0.25cm}
			En el óptimo:
			
			$CI_i(P_{G_i}) = \lambda + \mu_i^{min} \geq \lambda$ \quad si $P_{G_i} = P_{G_i}^{min}$
			
			$CI_i(P_{G_i}) = \lambda$ \qquad \qquad \qquad \, si $P_{G_i}^{min} < P_{G_i} < P_{G_i}^{max}$
			
			$CI_i(P_{G_i}) = \lambda + \mu_i^{max} \geq \lambda$ \quad si $P_{G_i} = P_{G_i}^{max}$
			
			
			\vspace{0.5cm}
			Los generadores que operan en su límite inferior tienen un $CI \geq \lambda$.
			
			
			Los generadores que operan en su límite superior tienen un $CI \leq \lambda$.
			
			
		\subsection*{Ejemplo. Despacho económico sin pérdidas y con límites de generación.}
			Considerando un rango de valores posibles para $P_D$ de $100$ a $800\,MW$ para las unidades de generación
			del ejemplo anterior, sujetas a los límites:
			\[C_1(P_{G_1}) = 900 + 45\, P_{G_1} + 0.01\, P_{G_1}^2\,\dfrac{UM}{h}; \quad 50\,MW \leq P_{G_1} \leq 200\,MW\]
			\[C_2(P_{G_2}) = 2500 + 43\, P_{G_2} + 0.003\, P_{G_2}^2\,\dfrac{UM}{h}; \quad 50\,MW \leq P_{G_2} \leq 600\,MW\]
			
			Para valores de $\lambda$ hasta 46, $P_{G_1} = 50\,MW$ (límite inferior), mientras que el generador 2, con $P_{G_2} = P_D - 50\,MW$ suministra el resto de carga.
			
			
			Cuando $\lambda_1 = \lambda_2 = 46$, el generador 2 suministra 500 MW, por lo que la carga total a servir en estas condiciones es de 550 MW.
			
			
			Para valores de $\lambda$ tal que $46 \leq \lambda \leq 46.6$ ninguna de las unidades alcanza sus límites y se pueden hallar $P_{G_1}$ y $P_{G_2}$ con las fórmulas del costo incremental:
			\[CI_i(P_{G_i}) = \dfrac{dC_i(P_{G_i})}{dP_{G_i}}\]
			
			
			Para valores de $\lambda > 46.6$, $P_{G_2} = 600\,MW$ (su límite superior) y $P_{G_1}$ = $P_D - 600\,MW$. 
			
			
			Si $\lambda_1 = 49$ ambos generadores máquinas entregan su potencia máxima, 200 y 600 MW respectivamente, alcanzando una potencia total de 800 MW.
			
			
		\subsection{Modelo con pérdidas en la red y límites de funcionamiento.}
			Si las centrales están muy alejadas de los puntos de consumo, hay que considerar las
			pérdidas en las líneas de transporte y distribución. Los más alejados tendrán asociado un coste o factor de penalización. La consideración de las pérdidas altera el DE, ya que el coste marginal con respecto a la
			demanda no es único en toda la red, sino que varía de nudo en nudo.
			
			\subsubsection{Modificación del equilibrio en el despacho económico.}
				Las pérdidas se consideran como demandas adicionales, incrementando las demandas netas.
				
				
				La relación entre la demanda y la generación deja de ser lineal debido al término cuadrado de la corriente en las pérdidas: $P=RI^2$
				
				
				La intensidad depende de las $P_G$ y $P_D$ demandadas. Se debe mantener la tensión constante en los nudos: $U = 1\,p.u.$
				
				
				La consideración de las pérdidas de potencia en el transporte, dependientes de la potencia generada en cada central, implica un térmico adicional en el balance de potencias, quedando el problema de optimización de la forma:
				\[C_T(P_{G_i}) = \sum_{i=1}^{g} C_i(P_{G_i})\]
				
				
				Sujeto a:
				\[\sum_{i=1}^{g} P_{G_i} = P_D + P_{perd}(P_1,\dots P_g) \Rightarrow \sum_{i=1}^{n} P_{G_i} - P_D^{total} - P_{perd} (P_G, P_D) = 0\]
				
				
				Siempre que $P_{G_i}^{min} < P_{G_i} < P_{G_i}^{max}$.
				
				
				\vspace{0.25cm}
				La función lagrangiana es:
				\[L(P_G,\lambda) = \sum_{i=1}^{n} C_i(P_{G_i}) - \lambda \left(\sum_{i=1}^{n} P_{G_i} - P_D^{total} - P_{perd}(P_G, P_D)\right)\]
				
				
				Condiciones necesarias de primer orden para la solución óptima:
				\[\dfrac{\partial L(P_{G_i},\lambda)}{\partial P_{G_i}} = CI_i(P_{G_i})-\lambda \left(1-\left.\dfrac{\partial P_{perd}}{P_{G_i}}\right|_S\right) = 0\]
				
				\[\dfrac{\partial L(P_{G_i},\lambda)}{\partial \lambda} = -\sum_{i=1}^{n} P_{G_i} + P_D^{total} + P_{perd}(P_D, P_G) = 0\]
				
				
				La condición de óptimo implica ahora la igualdad de costes incrementales de los generadores, pero afectados por un \textbf{factor de penalización} que dpenede de las pérdidas incrementales del generador:
				\[CI_i(P_{G_i}) = \lambda\left(1-\dfrac{\partial P_{perd}}{\partial P_{G_i}}\right)\]
				
				
				%Balance de potencias:
				%\[\sum_{i=1}^{n}\left(1-\left.\dfrac{\partial P_{perd}}{\partial P_{G_i}} \right|_S \right) dP_{G-i} = 0\]
				%
				%
				%Costes marginales:
				%\[\lambda_i = \dfrac{\partial C}{\partial P_{D_i}} = \lambda \left(1-\left.\dfrac{\partial P_{perd}}{\partial P_{G_i}} \right|_S \right)\]
				
				
				Pérdidas incrementales en cada generador:
				\[PIT_i = \dfrac{\partial P_p}{\partial P_{G_i}}\]
				
				
				Factor de penalización de cada generador:
				\[L_i = \dfrac{1}{1-PIT_i}\]
				
				
				El óptimo queda definido por las ecuaciones:
				\[CI_i\cdot L_i = \lambda \quad /\,i = 1,2\dots g\]
				\[\sum_{i=1}^{g} P_{G_i} = P_D + P_p(P_1,P_2\dots P_g)\]
				
				
			\subsubsection{Conclusiones.}
				Se observa que ya no es condición de óptimo que cada generador funcione con el mínimo coste
				incremental. Los $CI_i$ están ponderados por los factores $L_i$. Si $L_i$ es muy elevado, al unidad generadora es menos atractiva, ya que está muy alejado del consumidor final.
				
				
				
				Los generadores no operan a costes marginales iguales, tal como en el caso sin pérdidas, sino que
				varían según sus pérdidas.
				
			\subsubsection{Cálculo de las pérdidas incrementales de cada generador.}
				Una de las formas de plantear el problema es a través de la ecuación de pérdidas por métodos empíricos. A partir de datos históricos del sistema eléctrico o de múltiples estudios del mismo, se puede obtener una ecuación cuadrática de las pérdidas en función de las potencias generadas considerando los coeficientes de pérdidas $B$:
				\[P_p = \sum_{i=1}^{m}\sum_{j=1}^{m} P_{G_i}\cdot B_{ij} \cdot P_{G_j} + \sum_{i=1}^{m} B_{0i} \cdot P_{G_i} + B_{00}\]
				
				$B_{ij}$ son pérdidas ($RI^2$).
				
				$B_{0i}$ es un término lineal.
				
				$B_{00}$ son pérdidas fijas en los transformadores, etc.
				
				
				\vspace{0.25cm}
				Conocidos los coeficientes de pérdidas Bij (MW-1), las pérdidas incrementales de transmisión vendrán dadas por:
				\[PIT_i = \dfrac{\partial P_p}{\partial P_{G_i}} = 2 \sum_{j=1}^{m} B_{ij} \cdot P_{G_j} + B_{0i}\]
				
				
				Los coeficientes $B$ no son constantes, sino que varían en función del estados de carga del sistema. En la práctica, estos coeficientes pueden considerarse constantes, siempre que las condiciones del sistema no difieran drásticamente de las del caso base. En un sistema real, dada la variación de la potencia demandada a lo largo de un día, es necesario utilizar más de un conjunto de coeficientes $B$ durante el ciclo de carga diario.
				
				
				\vspace{0.25cm}
				Restricción de desigualdad:
				\[P_D + \sum_{i=1}^{m}\sum_{j=1}^{m}  P_{G_i}\cdot B_{ij} \cdot P_{G_j} + \sum_{i=1}^{m} B_{0i} \cdot P_{G_i} + B_{00} - \sum_{i=1}^{m} P_{G_i} = 0\]
				
				
				\vspace{0.25cm}
				Factor de penalización:
				\[L_i = \dfrac{1}{1-PIT_i} = \dfrac{1}{1-2 \sum\limits_{i=1}^{m} B_{ij} \cdot P_{G_j} + B_{0i}}\]
				
				
				\vspace{0.25cm}
				Ecuaciones de coordinación:
				\[CI_i(P_{G_i}) = \lambda \left(1 - \left(2 \sum\limits_{i=1}^{m} B_{ij} \cdot P_{G_j} + B_{0i} \right) \right)\]
				\[P_D + \sum_{i=1}^{m}\sum_{j=1}^{m}  P_{G_i}\cdot B_{ij} \cdot P_{G_j} + \sum_{i=1}^{m} B_{0i} \cdot P_{G_i} + B_{00} - \sum_{i=1}^{m} P_{G_i} = 0\]
				
				
			\subsubsection*{Ejemplo.}
				Un sistema de potencia tiene dos plantas generadoras y los coeficientes de pérdidas $B$ correspondientes a la red de transmisión son:
				
				\begin{figure}[H]
					\begin{minipage}{0.5\textwidth}
						\[B =
						\left[
						\begin{matrix}
							0.00500 & -0.00003\\
							-0.00003 & 0.00500
						\end{matrix}
						\right]
						\]
						\[
						B_0 =
						\left[
						\begin{matrix}
							0.00030\\
							0.00040
						\end{matrix}
						\right]\] 
						\[B_{00} = 0.00006\]
						Los datos están en valores por unidad sobre la base de $100\,MVA$. Determinar la expresión de las pérdidas en la red de transmisión.
					\end{minipage}
					\begin{minipage}{0.5\textwidth}
						\begin{figure}[H]
							\centering
							\begin{circuitikz}
								\tikzstyle{every node}=[font=\normalsize]
								\draw  (9.25,17.5) circle (0.5cm) node {\normalsize $G_1$} ;
								\draw  (12.25,17.5) circle (0.5cm) node {\normalsize $G_2$} ;
								\draw [short] (9.25,17) -- (9.25,16.5);
								\draw [short] (12.25,17) -- (12.25,16.5);
								\draw [short] (8.75,16.5) -- (9.75,16.5)node[pos=1,right]{1};
								\draw [short] (10,13.5) -- (11.5,13.5)node[pos=0.5,above]{3};
								\draw [->, >=Stealth] (10.75,13.5) -- (10.75,12.75)node[pos=0.5,right]{D};
								\draw [short] (9,16.5) -- (9,14.75);
								\draw [short] (9,14.75) -- (10.25,14.75);
								\draw [short] (10.25,14.75) -- (10.25,13.5);
								\draw [short] (11.25,14.75) -- (11.25,13.5);
								\draw [short] (11.25,14.75) -- (12.5,14.75);
								\draw [short] (12.5,14.75) -- (12.5,16.5);
								\draw [short] (12.75,16.5) -- (11.75,16.5)node[pos=1,left]{2};
								\draw [short] (9.5,16.5) -- (9.5,16);
								\draw [short] (9.5,16) -- (12,16);
								\draw [short] (12,16) -- (12,16.5);
							\end{circuitikz}
							
							\label{fig:my_label}
						\end{figure}
					\end{minipage}
				\end{figure}
				
				\underline{Solución:}
					\[P_p = P_{G_1}^2 B_{11} + P_{G_2}^2 B_{22} + 2 P_{G_1}P_{G_2}B_{12} + P_{G_1}B_{01} + P_{G_2}B_{02} + B_{00}\]
					
					$B_{11} = B_{22} = 0.005\cdot 100 = 0.5$
					
					$B_{12} = B_{21} = -0.00003\cdot 100 = -0.003$
					
					$B_{01} = 0.03$
					
					$B_{02} = 0.04$
					
					$B_{03} = 0.006$
					
					Cada central tendrá asociado un coste de penalización:
					
					\[PIT_i = \dfrac{\partial P_p}{\partial P_{G_i}} \Rightarrow
						\left\{
							\begin{matrix}
								L_1 = \dfrac{1}{1 - PIT_1} = \dfrac{1}{1 - (P_{G_1} - 0.006 P_{G_2} + 0.03)}\\
								L_2 = \dfrac{1}{1 - PIT_2} = \dfrac{1}{1 - (P_{G_2} - 0.006 P_{G_1} + 0.04)}
							\end{matrix}
						\right.
					\]
					
				\subsubsection{Método iterativo para obtener $\mathbf{\lambda}$.}
					Las ecuaciones que proporciona el despacho económico de los generadores son, generalmente, no lineales, debiendo utilizarse un algoritmo iterativo para su resolución. De la interpretación de $\lambda$ como coste marginal del sistema se deriva un algoritmo iterativo muy utilizado en las implementaciones prácticas del despacho económico:
					\begin{enumerate}
						\item Dada la demanda $P_D$, elegir valores iniciales de $P_{G_i}$.
						\item Calcular los $PIT_i$ de cada generador.
						\item Elegir un $\lambda$ inicial:
							\begin{enumerate}
								\item Determinar la potencia del generador $i$ de $L_i \cdot CI_i = \lambda$.
								\item Dependiendo del error en el balance de potencias:
								\[\sum_i P_{G_i} < P_D \Rightarrow \text{incrementar }\lambda\]
								\[\sum_i P_{G_i} > P_D \Rightarrow \text{decrementar }\lambda\]
								\item Si el error $\varepsilon$ en el balance de potencias del sistema no es despreciable, volver al apartado a).
								\[\left|\sum_i P{G_i} - P_D\right| < \varepsilon\]
 							\end{enumerate}
 						\item Si los cambios en las potencias son relativamente pequeños, el proceso de iteración finaliza. En caso contrario, volver al punto 2.
					\end{enumerate}
				
				\subsubsection{Método de la secante.}
					Partiendo de 2 puntos, se supone que para la iteración $\lambda_{m-1}$ hay un error $E_{m-1}$, y para la iteración $\lambda_m$ un error $E_m$. Por tanto se cumplirá que para un $\lambda_{m+1}$ habrá un $E=0$, realizando una extrapolación lineal de los otros dos puntos.
				
					\[\lambda_{m+1} = \lambda_{m} - \varepsilon_m \left(\dfrac{\lambda_{m-1} - \lambda_{m}}{\varepsilon_{m-1} - \varepsilon_m}\right)\]
					
					\begin{figure}[H]
						\begin{minipage}{0.6\textwidth}
							\textbf{Procedimiento:}	
							\begin{enumerate}
								\item Se realiza un D.E. con o sin pérdidas para obtener un $\lambda_{m-1}$ común, con un error de $E_{m-1}$.
								\item Se realiza un nuevo D.E. tomando un valor de $\lambda_m$ mayor, que dará un error de $E_m$.
								\item Con los valores anteriores se calcula $\lambda_{m+1}$, que cumple con la condición de que $E=0$:
								\item Si $E<10^{-3}$ se finaliza el proceso y se toman los valores finales como válidos. En caso contrario se vuelve a tomar un valor de $\lambda_{m}$
							\end{enumerate}
						\end{minipage}
						\begin{minipage}{0.4\textwidth}
							\begin{figure}[H]
								\centering
								\begin{circuitikz}[scale = 0.8]
									\tikzstyle{every node}=[font=\normalsize]
									\draw [->, >=Stealth] (6.5,16.75) -- (6.5,19.25)node[pos=1,above]{E(+)};
									\draw [->, >=Stealth] (6.5,16.75) -- (6.5,14.25)node[pos=1,below]{E(-)};
									\draw [->, >=Stealth] (6.5,16.75) -- (11.5,16.75)node[pos=1,above]{$\lambda$};
									\draw [ color={rgb,255:red,0; green,128; blue,255}, short] (7.25,14.75) -- (11.25,18);
									\draw [dashed] (7.5,15) -- (6.5,15)node[pos=1,left]{$E_{m-1}$};
									\draw [dashed] (8.5,15.75) -- (6.5,15.75)node[pos=1,left]{$E_m$};
									\draw [dashed] (8.5,15.75) -- (8.5,16.75)node[pos=1,above]{$\lambda_m$};
									\draw [dashed] (7.5,15) -- (7.5,16.75)node[pos=1,above]{$\lambda_{m-1}$};
									\draw (9.7,16.75) to[short, -*] (9.7,16.75);
									\node [font=\normalsize] at (10,16.4) {$\lambda_{m+1}$};
								\end{circuitikz}
								
								\label{fig:my_label}
							\end{figure}
						\end{minipage}
					\end{figure}
									
				\subsubsection*{Ejemplo.}
					Considerando las características de costes de los generadores del ejemplo anterior:
					\[C_1(P_{G_1}) = 900 + 45\, P_{G_1} + 0.01\, P_{G_1}^2\,\dfrac{UM}{h}; \quad 50\,MW \leq P_{G_1} \leq 200\,MW\]
					\[C_2(P_{G_2}) = 2500 + 43\, P_{G_2} + 0.003\, P_{G_2}^2\,\dfrac{UM}{h}; \quad 50\,MW \leq P_{G_2} \leq 600\,MW\]
					La carga total $P_D$ que debe ser suministrada es de $700\,MW$. La expresión simplificada de las pérdidas es de la forma:
					\[P_p = B_{11}P_{G_1}^2 + B_{22}P_{G_2}^2\,MW\]
					Los coeficientes de pérdidas:
					\[B_{11} = 3\cdot 10^{-5} \qquad B_{22} = 9\cdot 10^{-5}\]
					Determinar, utilizando el método de iteración en $\lambda$, la potencia que debe entregar cada máquina, las pérdidas
					en el sistema, el coste incremental y el coste total.
					
					\vspace{0.25cm}
					\underline{Solución:}
					\[\lambda = L_1 CI_1 = \dfrac{1}{1-0.00006 P_1}(45 + 0.02 P_1) \qquad \lambda = L_2 CI_2 = \dfrac{1}{1-0.00018 P_1}(43 + 0.006 P_2)\]
					\[P_p = 0.00003 P_1^2 + 0.00009 P_2^2\]
					\[P_1 + P_2 - P_p - 700 = 0\]
					
					
					Para utilizar el método de iteración en $\lambda$ conviene escribir las ecuaciones de la forma:
					\[P_1 = \dfrac{\lambda - 45}{0.02 + 0.00006\lambda}\qquad P_2 = \dfrac{\lambda - 43}{0.006 + 0.00018\lambda}\]
					\[P_p = 0.00003 P_1^2 + 0.00009 P_2^2\]
					\[P_1 + P_2 - P_p - 700 = 0\]
					
					
					Para comenzar el proceso iterativo se puede considerar como valor de partida para $\lambda$ el determinado en el caso de despacho sin pérdidas y con límites: $\lambda = 46.69$. En la tabla siguiente se muestra el desarrollo del proceso y los resultados obtenidos mediante iteraciones:
					
					\begin{table}[H]
						\centering
						\begin{tabular}{ccccccc}
							n	 & $\lambda$ & $P_1$ & $P_2$ & $P_p$ & $P_1+P_2+P_3-P_D$ & $C_T = C_1 + C_2$\\
							-	 & $[UM/MWh]$ & $[MW]$ & $[MW]$ & $[MW]$ & $[MW]$ & $[UM/h]$\\
							\hline
							1	 & 46,690	& 74,12		& 256,18	& 6,07  & -375,78	& 18002,67\\
							2	 & 50,000	& 217,39	& 466,67	& 21,02 &	-36,96	& 34375,20\\
							3	 & 50,500	& 238,82	& 497,02	& 23,94 &	11,89	& 36830,05\\
							4	 & 50,300	& 230,25	& 484,92	& 22,75 &	-7,58	& 35848.67\\
							5	 & 50,350	& 232,40	& 487,95	& 23,05 &	-2,70	& 36094,09\\
							6	 & 50,370	& 233,25	& 489,16	& 23,17 &	-0,75	& 36192,24\\
							7	 & 50,375	& 233,47	& 489,46	& 23,20 &	-0,27	& 36216,78\\
							8	 & 50,380	& 233,68	& 489,77	& 23.23 &	0.22	& 36241,31\\
							9	 & 50,378	& 233,60	& 489,65	& 23,21 &	0,03	& 36231,50\\
							10 	 & 50,377	& 233,55	& 489,59	& 23,21 &	-0,07	& 36226,59
							
						\end{tabular}
					\end{table}
					
					Comparando la iteración 9 con los casos sin pérdidas se aprecia que el coste incremental y total ahora son mayores y que la potencia entregada entregada por el generador 1 aumenta y la entregada por el generador 2 disminuye, a pesar de que el coste incremental de esta última es menor, debido a que su factor de penalización es mayor.
					
					
					Tomando como punto de partida el valor de $\lambda_{m-1} = 46,69$, con $\varepsilon_{m-1} = -375,78\,MW$ se realiza otra iteración con un valor superior de $\lambda_m = 50.350$, que da un valor de $\varepsilon_m = -2.70\,MW$. Sustituyendo en la expresión
					\[\lambda_{m+1} = \lambda_{m} - \varepsilon_m \left(\dfrac{\lambda_{m-1} - \lambda_{m}}{\varepsilon_{m-1} - \varepsilon_m}\right)\]
					
					
					se obtiene un valor de $\lambda_{m+1} = 5.0376$ y un valor de $\varepsilon_{m+1} = -0.165\,MW = -16.5\cdot 10^{-8}\,p.u.$, que es muy próximo a 0, y por lo tanto se asume válido.
			
	\section{Programación de arranques y paradas de centrales.}
		El problema de la programación de arranques y paradas de centrales térmicas
		consiste en determinar qué centrales van a estar en servicio y cuánto van a generar en
		cada periodo con el objeto de optimizar los costes de producción, teniendo en cuenta
		la evolución de la demanda a cubrir por las centrales térmicas a lo largo del
		horizonte de programación.
		
		
		Suponiendo que el horizonte de programación es de 24 horas y que existen ng
		generadores térmicos, la función de costes a optimizar se puede expresar de la
		siguiente forma:
		\[\sum_{g=1}^{n}\left(A_g + \sum_{t=1}^{24} C_{g,t}(P_{g,t})\right)\]
		$C_{g,t}$ es el coste de generación del generador $g$ a la hora $t$.
		$A_g$ es la suma de los otros costes que incurre en el generador $g$, típicamente costes de arranque y parada de la central.
		
		
		Las restricciones que debe cumplir el problema de optimización se pueden clasificar
		en dos grupos:
		\begin{itemize}
			\item Restricciones impuestas por el sistema eléctrico.
			\item Restricciones particulares de cada central.
		\end{itemize}
		
		\subsection{Costes de arranque y parada.}
			El coste de arranque puede variar desde un valor máximo (coste de arranque en frío) hasta un valor
			menor si la central se desacoplo recientemente y aún está cerca de su temperatura de operación.

			Tiempos de arranque de algunas centrales:
			\begin{itemize}
				\item Nuclear: 7 días.
				\item Carbón: 12 horas.
				\item Fuel/gas: 8 horas.
				\item Ciclo combinado: 4 horas.
				\item Hidráulica: minutos.
				\item Eólica y fotovoltaica: minutos (no controlable).
			\end{itemize}
			
			Los costes de parada son típicamente constantes para cada central y representan el desaprovechamiento de combustible y
			la mano de obra necesaria para desacoplar una central. Son generalmente mucho más pequeños que los costes de arranque.
			
		\subsection{Coordinación hidrotérmica.}
			Conocida la energía total disponible (cantidad de agua embalsada) para la generación
			en centrales hidroeléctricas, la coordinación hidrotérmica determina qué fracción de
			la demanda se va a cubrir mediante dichas centrales en cada intervalo del horizonte
			de programación, siempre con el objeto de reducir al máximo el coste de
			generación diario.
			
			
			Es habitual trabajar con una curva de costes del parque térmico en función de la
			potencia cubierta mediante generación térmica: $CT(PT)$.
			
			
			La generación hidráulica se reserva para las horas punta de mayor consumo: mayor coste de
			oportunidad.
			
			
			Suponiendo que el horizonte de programación es de 24 horas y que existen $ng$
			generadores térmicos y $nh$ generadores hidráulicos:
			\[P_{Dt} = \sum_{t=1}^{24} P_{T_t} + P_{H_t}\]
			
			\subsubsection{Restricciones de las centrales hidráulicas.}
				\begin{itemize}
					\item \textbf{Balance energético de cada periodo:}
						\[P_{Dt} = P_{T_t} + \sum_{h=1}^{nh} P_{h,t}\quad/\,t=1,2,\dots24\]
						Donde $P_{h,t}$ es la contribución de las centrales hidráulicas en cada periodo $t$.
					
					\item \textbf{Potencia máxima y mínima de cada grupo hidráulico.}
						\[P_h^{min} \leq P_{h,t} \leq P_h^{max}\]
						
					\item \textbf{Energía embalsada disponible en el horizonte de programación.}
						\[V_h \geq \sum_{t=1}^{24} P_{h,t}\]
						Siendo $V_h$ la energía total disponible en MWh.
				\end{itemize}
				
			\subsubsection{Balance de las centrales hidráulicas.}
				\textbf{\textit{Volumen de agua embalsada.}}
				
				\[V_{h,t} = V_{h,t-1} + A_{h,t} - D_{h,t} - Q_{h,t}(P_{h,t})\]
				\begin{itemize}
					\item $V_{h,t-1}$ es el volumen de agua en el embalse asociado a la central $h$ en el intervalo $t$, sujeto a límites máximo y mínimo.
					\item $A_{h,t}$ son las aportaciones externas al embalse en el intervalo $t$, conocidas o estimadas.
					\item $D_{h,t}$ es el caudal vertido por el embalse en el intervalo $t$, excedente que no es utilizado para la generación eléctrica.
					\item $Q_{h,t}$ es el caudal medio utilizado por la central $h$ en el intervalo $t$, con límites mínimo y máximo.
				\end{itemize}
				
				\textbf{\textit{Centrales de una misma cuenca.}}
				\[V_{h,t} = V_{h,t-1} + A_{h,t} - D_{h,t} - Q_{h,t}(P_{h,t}) + \sum_e \left(Q_{e,t-\tau}(P_{e,t-\tau}) + D_{e,t-\tau}\right)\]
				Siendo $D_h$ el conjunto de embalses aguas arriba del embalse $h$ y $\tau$ el tiempo que tarda el caudal vertido por la central $e$ en estar disponible en el embalse $h$.
				
				\newpage
				\textbf{\textit{Centrales de bombeo reversibles.}}
				
				
				En el caso de las centrales hidráulicas de bombeo puro (centrales cuya única aportación apreciable al agua
				embalsada es la bombeada por la propia central), la ecuación de continuidad del agua embalsada es la
				siguiente:
				\[V_{b,t} = V_{b,t-1} - Q_{G_{b,t}}(P_{G_{b,t}}) + Q_{C_{b,t}}(P_{C_{b,t}})\]
				
				El balance energético en cada periodo deberá incluir tanto la generación como el consumo de las centrales de
				bombeo:
				\[P_{Dt} = P_{Tt} + \sum_{h=1}^{nh}P_{h,t} + \sum_{b=1}^{nb}P_{G_{b,t}}-P_{C_{b,t}}\]
				
			\subsubsection*{Ejemplo.}
				Una central hidroeléctrica con una potencia máxima de $600\,MW$ dispone de una energía almacenada en el embalse de $3600\,MWh$. La demanda de energía evoluciona según la figura:
				
				\begin{figure}[H]
					\centering
						\begin{circuitikz}[scale = 0.9]
							\tikzstyle{every node}=[font=\normalsize]
							\draw [short] (3.75,14) .. controls (5.25,13.5) and (6,12.25) .. (6.5,10.75);
							\draw [short] (6.5,10.75) -- (8.5,10.75);
							\draw [short] (8.5,10.75) -- (8.5,14);
							\draw [short] (8.5,14) -- (8.75,14);
							\draw [short] (8.75,14) -- (9.5,10.75);
							\draw [short] (9.5,10.75) -- (11,10.75);
							\draw [short] (9.25,11.75) -- (10.5,11.75);
							\draw [short] (10.5,11.75) -- (10.5,10.75);
							\draw [->, >=Stealth] (10,11.25) -- (11.75,11.25)node[pos=1,above]{P $\leq$ $600\,MW$};
							\draw [ color={rgb,255:red,0; green,128; blue,255}, short] (5,13.25) -- (8.5,13.25);
							\node [font=\normalsize] at (7.25,12.25) {$3600\,MWh$};
							\draw [->, >=Stealth] (14,10.75) -- (14,14.25)node[pos=1,right]{$P_D\,[MW]$};
							\draw [->, >=Stealth] (14,10.75) -- (19.5,10.75)node[pos=1,right]{$t\,[h]$};
							\draw [dashed] (16.75,10.75) -- (16.75,13)node[pos=0,below]{12};
							\draw [dashed] (16,10.75) -- (16,11.75)node[pos=0,below]{8};
							\draw [dashed] (19.25,10.75) -- (19.25,13)node[pos=0,below]{24};
							\draw [ color={rgb,255:red,255; green,0; blue,0}, short] (14,11.75) -- (16,11.75);
							\draw [ color={rgb,255:red,255; green,0; blue,0}, short] (16,11.75) -- (16,13.5);
							\draw [ color={rgb,255:red,255; green,0; blue,0}, short] (16,13.5) -- (16.75,13.5);
							\draw [ color={rgb,255:red,255; green,0; blue,0}, short] (16.75,13.5) -- (16.75,13);
							\draw [ color={rgb,255:red,255; green,0; blue,0}, short] (16.75,13) -- (19.25,13);
							\draw [dashed] (16,13.5) -- (14,13.5)node[pos=1,left]{2500};
							\draw [dashed] (16.75,13) -- (14,13)node[pos=1,left]{2000};
							\draw [dashed] (14,11.75) -- (14,11.75)node[pos=1,left]{1000};
						\end{circuitikz}
					
					\label{fig:my_label}
				\end{figure}
				
				La curva de coste equivalente del conjunto de centrales térmicas es la siguiente:
				\[C_T = 3250 - 0.3P_T + 0.0024 P_T^2\,\left[\dfrac{\euro}{h}\right]\]
				
				
				Determinar el programa óptimo de generación de la central hidráulica y el ahorro que se obtiene al utilizar energía almacenada en el embalse.
				
				\vspace{0.25cm}
				\underline{Solución:}
				
				
				Si toda la demanda fuera asumida por las centrales térmicas el coste de generación sería:
				\[P_T = 1000 \Rightarrow C = 5350\,\left[\dfrac{\euro}{h}\right]\]
				\[P_T = 2500 \Rightarrow C = 17500\,\left[\dfrac{\euro}{h}\right]\]
				\[P_T = 2000 \Rightarrow C = 12250\,\left[\dfrac{\euro}{h}\right]\]
				Coste total: $8\cdot 5350 + 4\cdot 17500 + 12\cdot 12250 = 259800\,\euro$
				
				
				Luego hay que minimizar
				\[\sum_{t=1}^{3}d_t\cdot C(P_{T_t})\]
				
				
				Las restricciones que debe cumplir el problema de optimización son las siguientes:
				\[P_{Dt} = \sum_{t=1}^{24} P_{T_t} + P_{H_t} \quad /\,t = 1,2,3 \text{ y } P_{H_t} \text{ la potencia de la central en cada periodo.}\]
				\[0 \leq P_{H_t} \leq 600\]
				\[3600 = \sum_{t=1}^{3}d_t\cdot P_{H_t}\]
				
				\begin{itemize}
					\item De 0:00 a 8:00h la central hidráulica no genera, por lo que el coste es del parque térmico:
						\[P_T = 1000 \Rightarrow C = 5350\,\euro/h\]
					\item De 8:00 a 12:00h la central hidráulica trabaja con una potencia media de $600\,MW$, consumiendo una energía de $2400\,MWh$. El coste horario es:
						\[P_T = 1900 \Rightarrow C = 11344\,\euro/h\]
					\item De 12:00 a 24:00h la central hidráulica trabaja con una potencia media de $100\,MW$, consumiendo una energía de $1200\,MWh$. El coste horario es:
						\[P_T = 1900 \Rightarrow C = 11344\,\euro/h\]
				\end{itemize}
				
				En consecuencia, el coste óptimo diario resulta $224304\,\euro$ y el ahorro obtenido de la utilización de la energía hidráulica es $35496\,\euro$.
				
		\section{Tema 6. Medida de potencia y energía.}
		
		\chapter{Semejanza hidrodinámica.}
\begin{figure}[H]
	\centering
	\begin{circuitikz}
		\tikzstyle{every node}=[font=\normalsize]
		\draw [short] (16.75,16) -- (16.75,12.25);
		\draw [ color={rgb,255:red,0; green,128; blue,255}, short] (16.5,15.75) .. controls (16.5,15) and (16.5,14.5) .. (15.25,14.5);
		\draw [ color={rgb,255:red,0; green,128; blue,255}, short] (16.5,12.5) .. controls (16.5,13.25) and (16.5,13.75) .. (15.25,13.75);
		\draw [ color={rgb,255:red,0; green,128; blue,255}, short] (15.25,14.5) -- (13,14.5);
		\draw [ color={rgb,255:red,0; green,128; blue,255}, short] (15.25,13.75) -- (13,13.75);
		\draw [ color={rgb,255:red,0; green,128; blue,255}, ->, >=Stealth] (13,14.125) -- (13.75,14.125)node[pos=1,right]{$Q$};
		\node [font=\normalsize] at (15.25,15.25) {$P_a$};
		\draw [short] (16.75,15.5) -- (17,15.75);
		\draw [short] (16.75,15.25) -- (17,15.5);
		\draw [short] (16.75,15) -- (17,15.25);
		\draw [short] (16.75,14.75) -- (17,15);
		\draw [short] (16.75,14.5) -- (17,14.75);
		\draw [short] (16.75,14.25) -- (17,14.5);
		\draw [short] (16.75,14) -- (17,14.25);
		\draw [short] (16.75,13.75) -- (17,14);
		\draw [short] (16.75,13.25) -- (17,13.5);
		\draw [short] (16.75,13) -- (17,13.25);
		\draw [short] (16.75,12.75) -- (17,13);
		\draw [short] (16.75,12.5) -- (17,12.75);
		\draw [short] (16.75,13.5) -- (17,13.75);
	\end{circuitikz}
\end{figure}

\begin{figure}[H]
	\centering
	\begin{circuitikz}
		\tikzstyle{every node}=[font=\normalsize]
		\draw [short] (10.75,15) -- (10.75,13.25);
		\draw [short] (10.75,13.25) -- (14,13.25);
		\draw [short] (14,13.25) -- (12.25,15);
		\draw [short] (12.25,15) -- (10.75,15);
		\draw  (11.25,13) circle (0.25cm);
		\draw  (13.25,13) circle (0.25cm);
		\draw [short] (10.5,12.75) -- (14.25,12.75);
		\draw [short] (11,12.75) -- (10.75,12.5);
		\draw [short] (11.25,12.75) -- (11,12.5);
		\draw [short] (11.5,12.75) -- (11.25,12.5);
		\draw [short] (11.75,12.75) -- (11.5,12.5);
		\draw [short] (12,12.75) -- (11.75,12.5);
		\draw [short] (12.25,12.75) -- (12,12.5);
		\draw [short] (12.5,12.75) -- (12.25,12.5);
		\draw [short] (12.75,12.75) -- (12.5,12.5);
		\draw [short] (13,12.75) -- (12.75,12.5);
		\draw [short] (13.25,12.75) -- (13,12.5);
		\draw [short] (13.5,12.75) -- (13.25,12.5);
		\draw [short] (13.75,12.75) -- (13.5,12.5);
		\draw [short] (14,12.75) -- (13.75,12.5);
		\draw [ color={rgb,255:red,255; green,0; blue,0}, short] (14.5,13.5) .. controls (13,14) and (13,15) .. (12.5,15.5);
		\draw [ color={rgb,255:red,255; green,0; blue,0}, short] (12.5,15.5) .. controls (12,15.75) and (11.75,15.25) .. (10.75,15.25);
		\draw [ color={rgb,255:red,255; green,0; blue,0}, short] (14.5,13.75) .. controls (13.25,14.25) and (13.25,15.25) .. (12.5,15.75);
		\draw [ color={rgb,255:red,255; green,0; blue,0}, short] (12.5,15.75) .. controls (12,16) and (11.75,15.5) .. (10.75,15.5);
		\draw [ color={rgb,255:red,255; green,0; blue,0}, short] (14.5,14) .. controls (13.5,14.75) and (13.25,15.75) .. (12.5,16);
		\draw [ color={rgb,255:red,255; green,0; blue,0}, short] (12.5,16) .. controls (11.5,16) and (11.5,15.75) .. (10.75,15.75);
		\draw [ color={rgb,255:red,255; green,0; blue,0}, short] (12.5,16.25) .. controls (11.5,16.25) and (11.75,16) .. (10.75,16);
		\draw [ color={rgb,255:red,255; green,0; blue,0}, short] (12.5,16.25) .. controls (13.25,16) and (13.75,15.25) .. (14.5,14.25)node[pos=0.5,above,sloped]{$\rho_{\infty}, \, P_{\infty}$};
		\draw [ color={rgb,255:red,255; green,0; blue,0}, short] (15.25,16.25) -- (15.25,13.25)node[pos=0,above]{$v_{\infty}$};
		\draw [ color={rgb,255:red,255; green,0; blue,0}, ->, >=Stealth] (15.25,13.5) -- (14.75,13.5);
		\draw [ color={rgb,255:red,255; green,0; blue,0}, ->, >=Stealth] (15.25,14) -- (14.75,14);
		\draw [ color={rgb,255:red,255; green,0; blue,0}, ->, >=Stealth] (15.25,14.5) -- (14.75,14.5);
		\draw [ color={rgb,255:red,255; green,0; blue,0}, ->, >=Stealth] (15.25,15) -- (14.75,15);
		\draw [ color={rgb,255:red,255; green,0; blue,0}, ->, >=Stealth] (15.25,15.5) -- (14.75,15.5);
		\draw [ color={rgb,255:red,255; green,0; blue,0}, ->, >=Stealth] (15.25,16) -- (14.75,16);
	\end{circuitikz}
\end{figure}

\begin{figure}[H]
	\centering
	\begin{circuitikz}
		\tikzstyle{every node}=[font=\normalsize]
		\draw [short] (10.75,15) -- (10.75,13.25);
		\draw [short] (10.75,13.25) -- (14,13.25);
		\draw [short] (14,13.25) -- (12.25,15);
		\draw [short] (12.25,15) -- (10.75,15);
		\draw  (11.25,13) circle (0.25cm);
		\draw  (13.25,13) circle (0.25cm);
		\draw [short] (10.5,12.75) -- (14.25,12.75);
		\draw [short] (11,12.75) -- (10.75,12.5);
		\draw [short] (11.25,12.75) -- (11,12.5);
		\draw [short] (11.5,12.75) -- (11.25,12.5);
		\draw [short] (11.75,12.75) -- (11.5,12.5);
		\draw [short] (12,12.75) -- (11.75,12.5);
		\draw [short] (12.25,12.75) -- (12,12.5);
		\draw [short] (12.5,12.75) -- (12.25,12.5);
		\draw [short] (12.75,12.75) -- (12.5,12.5);
		\draw [short] (13,12.75) -- (12.75,12.5);
		\draw [short] (13.25,12.75) -- (13,12.5);
		\draw [short] (13.5,12.75) -- (13.25,12.5);
		\draw [short] (13.75,12.75) -- (13.5,12.5);
		\draw [short] (14,12.75) -- (13.75,12.5);
		\draw [ color={rgb,255:red,255; green,0; blue,0}, <->, >=Stealth] (10.75,15.5) -- (14,15.5)node[pos=0.5,above]{$L_m$};
		\draw [ color={rgb,255:red,255; green,0; blue,0}, dashed] (14,15.5) -- (14,13.25);
		\draw [ color={rgb,255:red,255; green,0; blue,0}, dashed] (10.75,15.5) -- (10.75,15);
	\end{circuitikz}
\end{figure}

\begin{figure}[H]
	\centering
	\begin{circuitikz}
		\tikzstyle{every node}=[font=\normalsize]
		\draw [short] (11,17.5) -- (11,16.25);
		\draw [short] (11,16.25) -- (12,16.25);
		\draw [short] (12.5,16.25) -- (13.5,16.25);
		\draw [short] (13.5,16.25) -- (13.5,17.5);
		\draw [short] (12,16.25) -- (12,15.5);
		\draw [short] (12.5,16.25) -- (12.5,15.5);
		\draw [ color={rgb,255:red,0; green,128; blue,255}, ->, >=Stealth] (12.25,16) -- (12.25,15.25)node[pos=1,below]{Q};
		\draw [->, >=Stealth] (13,15.75) -- (12.5,15.75)node[pos=0,right]{$D_p$};
		\draw [->, >=Stealth] (11.5,15.75) -- (12,15.75);
		\draw [->, >=Stealth] (14,17.25) -- (13.5,17.25)node[pos=0,right]{$A_p$};
		\draw [->, >=Stealth] (10.5,17.25) -- (11,17.25);
		\draw [ color={rgb,255:red,0; green,128; blue,255}, short] (11,17) -- (13.5,17);
		\node [font=\normalsize, color={rgb,255:red,0; green,128; blue,255}] at (12.25,16.75) {$\rho_p$};
		\draw [->, >=Stealth] (11.5,17.75) -- (11.5,17.25)node[pos=0,right]{$\vec g$};
		\draw [ color={rgb,255:red,0; green,128; blue,255}, <->, >=Stealth] (10.75,17) -- (10.75,16.25)node[pos=0.5,left]{$H_0$};
	\end{circuitikz}
\end{figure}

\begin{figure}[H]
	\centering
	\begin{circuitikz}
		\tikzstyle{every node}=[font=\normalsize]
		\draw [ color={rgb,255:red,255; green,0; blue,0} ] (14.25,16.25) circle (1.75cm);
		\draw [ color={rgb,255:red,255; green,0; blue,0} ] (9.75,15.5) circle (0.75cm);
		\draw [short] (8.5,17) -- (8.5,15.25);
		\draw [short] (9.25,17) -- (9.25,15.75);
		\draw [short] (8.5,15.25) -- (11.5,15.25);
		\draw [short] (9.25,15.75) -- (11.5,15.75);
		\draw [ color={rgb,255:red,0; green,128; blue,255}, <->, >=Stealth] (9.25,16.5) -- (11.5,16.5)node[pos=0.5,above]{$L_t$};
		\draw [ color={rgb,255:red,0; green,128; blue,255}, <->, >=Stealth] (9.25,16) -- (10.25,16)node[pos=1,right]{$L_e$};
		\draw [ color={rgb,255:red,0; green,128; blue,255}, short] (9.25,15.75) .. controls (9.5,15.5) and (9.5,15.5) .. (10.25,15.5);
		\draw [ color={rgb,255:red,0; green,128; blue,255}, short] (9.25,15.25) .. controls (9.5,15.5) and (9.5,15.5) .. (10.25,15.5);
		\draw [short] (12.75,17) -- (15.75,17);
		\draw [short] (12.75,15.5) -- (15.75,15.5);
		\draw [short] (12.75,17) -- (12.75,15.5);
		\draw [ color={rgb,255:red,0; green,128; blue,255}, short] (14.75,17) -- (14.75,15.5);
		\draw [->, >=Stealth] (14.75,16.75) -- (15,16.75);
		\draw [->, >=Stealth] (14.75,16.5) -- (15.25,16.5);
		\draw [->, >=Stealth] (14.75,16.25) -- (15.5,16.25);
		\draw [->, >=Stealth] (14.75,16) -- (15.25,16);
		\draw [->, >=Stealth] (14.75,15.75) -- (15,15.75);
		\draw [->, >=Stealth] (12.75,16.75) -- (13,16.75);
		\draw [->, >=Stealth] (12.75,16.5) -- (13,16.5);
		\draw [->, >=Stealth] (12.75,16.25) -- (13,16.25);
		\draw [->, >=Stealth] (12.75,16) -- (13,16);
		\draw [->, >=Stealth] (12.75,15.75) -- (13,15.75);
		\draw [ color={rgb,255:red,0; green,128; blue,255}, ->, >=Stealth] (13.5,17) -- (13.5,16.5)node[pos=0.5,right]{$\delta$};
		\draw [ color={rgb,255:red,0; green,128; blue,255}, ->, >=Stealth] (13.5,15.5) -- (13.5,16)node[pos=0.5,right]{$\delta$};
		
		\draw [ color={rgb,255:red,0; green,128; blue,255}, short] (13,15.5) .. controls (13.25,16.25) and (14.75,16.25) .. (14.75,16.25);
		
		\draw [ color={rgb,255:red,0; green,128; blue,255}, short] (13,17) .. controls (13.25,16.25) and (14.75,16.25) .. (14.75,16.25);
		\draw [ color={rgb,255:red,255; green,0; blue,0}, ->, >=Stealth] (10.5,15.5) -- (12.5,16.25);
		\draw [ color={rgb,255:red,0; green,128; blue,255}, ->, >=Stealth] (15.75,16.25) -- (16.5,16.25)node[pos=1,right]{$Q_{H-P}$};
	\end{circuitikz}
\end{figure}
		\chapter{Centrales térmicas de carbón.}
\section{Esquema funcional de una central de carbón.}
Todas las centrales de combustible fósil clásicas tienen un esquema de funcionamiento prácticamente idéntico con las únicas diferencias siendo:
\begin{itemize}
	\item [-] El tratamiento previo del combustible.
	\item [-] El diseño de los quemadores.
	\item [-] El sistema de limpieza de humos y evacuación de las cenizas.
\end{itemize} 


Pese a ello, en la mayoría de centrales se pueden distinguir los siguientes circuitos básicos:
\begin{itemize}
	\item [-] Circuito de combustión.
	\item [-] Circuito aire-gases.
	\item [-] Circuito agua-vapor.
	\item [-] Circuito de agua de circulación.
	\item [-] Circuitos eléctricos.
	\item [-] Circuitos auxiliares.
\end{itemize}


\begin{figure}[H]
	\centering
	\includegraphics[width=0.7\linewidth]{res/tema10/esquemaFuncional}
	\label{fig:esquemafuncional}
\end{figure}

\section{Circuito aire-combustible-gases-ceniza.}
Este circuito se encarga de:
\begin{itemize}
	\item [-] Recibir y almacenar el combustible.
	\item [-] Preparar el combustible para ser quemado.
	\item [-] Transportar el combustible hasta el hogar.
	\item [-] Evacuación y filtrado de gases.
\end{itemize}
\section{Almacenamiento y preparación del combustible.}
\subsection{Almacenamiento del combustible.}
El almacenamiento se realiza en dos etapas. La primera etapa es el parque de combustible que suele tener una capacidad de almacenamiento de algunos meses de funcionamiento mientras que la segunda se compone por unos depósitos con capacidad para menos de un día.



El carbón se almacena en la cercanía de la central en parques a la intemperie y se maneja mediante rotopalas. Como se debe garantizar que haya un suministro esta la capacidad de almacenamiento suele ser elevada (hasta 150 días). Y si el carbón debe almacenarse más de un año se le proporciona un recubrimiento asfáltico.



No obstante, el almacenamiento de carbón presenta tres problemas principales:
\begin{itemize}
	\item [-] Combustión espontánea: debido al contacto con el aire el carbón se oxida a una velocidad proporcional a la temperatura (se duplica cada 8\grado). A los 65\grado $\ $ empieza a ser peligrosa y por tanto, el apilamiento debe hacerse de manera cuidadosa.
	\item [-] Pérdida de poder calorífica.
	\item [-] Degradación del tamaño del grano.
\end{itemize}


Del parque de almacenamiento se lleva el carbón a una torre de almacenamiento donde se separan las partículas férricas y trozos de piedras que pudiesen dañar los molinos. Tras pasar por la torre el carbón cae a través de las tolvas a unos alimentadores que dosifican la carga a los molinos.
\subsection{Preparación del combustible}
Una parte fundamental de la preparación del combustible consiste en pulverizar el carbón. 



Ventajas de la pulverización:
\begin{itemize}
	\item [-] Rendimiento de la combustión máximo.
	\item [-] Se pueden utilizar carbones de peor calidad.
	\item [-] Las cenizas y escorias no son pastosas (mejor manejo).
	\item [-] Mayor potencia calorífica por unidad de volumen del hogar. 
	\item [-] Menor costo de mano de obra.
 	\item [-] Fácil control del aire y combustible.
\end{itemize}





Desventajas de la pulverización:
\begin{itemize}
	\item [-] Elevado costo inicial de la instalación.
	\item [-] Costo de preparación del combustible.
	\item [-] Posibilidad de crear cenizas volantes (escapen por la chimenea).
\end{itemize}




Además, si el contenido de humedad es muy elevado el carbón antes de entrar al molino se mezcla con gas caliente para evaporar el agua.



Los molinos suelen transformar el carbón desde una granulometría menor a 150 mm hasta un grado de finura de aproximadamente 200$\mu m$ que depende del contenido en volátiles (a mayor contenido en volátiles mayor finura se requiere).



Una vez pulverizado la inyección puede ser:
\begin{itemize}
	\item [-] \textbf{Directa}: el carbón que sale de los molinos se lleva directamente a los quemadores. Es el método \textbf{más utilizado}.
	\item [-] \textbf{Indirecta}: el carbón se hace llegar a los molinos donde es transportado a unos silos de carbón pulverizado donde se almacena hasta que es inyectado en los quemadores.
\end{itemize}
\section{Tipos de molinos.}

\subsection{Molino de anillo de bolas.}
Es un tipo de molino donde un collar de esferas macizas de acero es arrastrado rondado entre dos anillos en los que hay talladas pistas troncotoroidales que guían a las bolas. La presión entre las bolas y anillos se mantiene mediante muelles de acero con una presión regulable.
\begin{figure}[H]
	\centering
	\includegraphics[width=0.4\linewidth]{res/tema10/molinoBolas}
	\label{fig:molinobolas}
\end{figure}

\subsection{Molino tubular de bolas tipo Hardinge.}
Es un molino que consta de un cilindro horizontal con bolas de acero en su interior que gira a velocidad constante. Se emplea aire caliente para secar el carbón y arrastrar el carbón pulverizado a un clasificador espiral.



Como las bolas se van desgastando con el tiempo ne van añadiendo en proporción 0,04-0,23 kg por tonelada de carbón pulverizado. Es un molino adecuado para antracitas aunque es ruidoso y es difícil controlar la finura del polvo. 



Estas calderas suelen absorber de 11 a 30$\frac{kWh}{ton}$ y pueden almacenar grandes cantidades de carbón para seguir suministrando carbón de 6 a 10 minutos.
\begin{figure}[H]
	\centering
	\includegraphics[width=0.6\linewidth]{res/tema10/molinoHardinge}
	\label{fig:molinohardinge}
\end{figure}

\subsection{Molino tubular de bolas Foster Wheeler.}
Es un molino muy simple, adecuado para antracitas. Es ruidoso y de velocidad limitada, pero permite controlar muy bien la finura de polvo. Puede almacenar gran cantidad de carbón.
\begin{figure}[H]
	\centering
	\includegraphics[width=0.4\linewidth]{res/tema10/fosterWheeler}
	\label{fig:fosterwheeler}
\end{figure}

\subsection{Molino de rodillos Babcock-Wilcox.}
Las bolas tienen un diámetro de 51 mm y su velocidad lineal es de 6$\frac{m}{s}$. El consumo de energía es de 8 a 12$\frac{kWh}{ton}$.
\begin{figure}[H]
	\centering
	\includegraphics[width=0.3\linewidth]{res/tema10/babcock}
	\label{fig:babcock}
\end{figure}

\subsection{Molino de rodillos de Raymond.}
Este molino consta de tres rodillos que giran sobre un camino de rodadura. La presión correcta se consigue mediante unos muelles ajustables.


Tiene un bajo coste de mantenimiento y es silencioso. Puede triturar 70$\frac{t}{h}$ de carbón pulverizado. Absorbe entre 11 y 16$\frac{kWh}{ton}$.
\begin{figure}[H]
	\centering
	\includegraphics[width=0.5 \linewidth]{res/tema10/raymond}
	\label{fig:raymond}
\end{figure}

\subsection{Molino de batea móvil.}
Es un molino  con una batea móvil con forma troncocónica que gira alrededor de un eje vertical a una velocidad entre 65 y 100 rpm. En su interior se alojan tres muelas cónicas tangentes a la superficie de la batea que giran solidariamente entre sí.




El carbón entra por el centro en el espacio que dejan libres las muelas que a través de la fuerza centrífuga es lanzado contra las paredes donde es triturado por las muelas.


Para extraer este carbón, se emplea aire precalentado que arrastra el carbón pulverizado hacia un separador y las partículas gruesas vuelven al molino.


Este molino tiene bajo costo de mantenimiento y bajo consumo de energía.

\begin{figure}[H]
	\centering
	\includegraphics[width=0.5\linewidth]{res/tema10/bateaMovil}
	\label{fig:bateamovil}
\end{figure}


\subsection{Molinos de tipo impacto.}
Este tipo de molinos emplea anillos cilíndricos que giran a una velocidad entre 1000 y 2000 rpm. Dos de los anillos son trituradores y otros dos son lisos para arrastrar el carbón.


A diferencia de anteriores molinos, el carbón descarga directamente por la parte inferior a través de una parrilla que asegura la granulometría deseada.
\begin{figure}[H]
	\centering
	\includegraphics[width=0.5\linewidth]{res/tema10/impacto}
	\label{fig:impacto}
\end{figure}

\section{Circuito aire-gases.}
El aire tomado de la atmósfera se envía mediante ventiladores de tiro forzado a través de precalentadores.

El objetivo de los quemadores:
\begin{itemize}
	\item [-] Recuperar el calor contenido en los gases a la salida de los intercambiadores de agua y vapor.
	\item [-] Elevar la temperatura del aire empleado en la combustión para mejorarla y, para secar el carbón.
\end{itemize}


A la salida de los precalentadores, el aire se envía a la cámara de combustión de diferentes manera:
\begin{itemize}
	\item [-] A través de los quemadores como aire primario mezclado con el combustible.
	\item [-] Alrededor de los quemadores como aire secundario.
	\item [-] A lo largo del recorrido de la llama como aire terciario.
\end{itemize}

\begin{figure}[H]
	\centering
	\includegraphics[width=0.7\linewidth]{res/tema10/circuitoAwaGas}
	\label{fig:circuitoawagas}
\end{figure}

\section{Quemadores.}
f
\section{Hogar.}
f
\subsection{Pantallas evaporadoras.}
f
\section{Ventiladores.}
f
\section{Precalentadores.}
f
\section{Productos de desecho.}
f
\section{Ceniceros.}
f
\section{Retención de cenizas.}
f
\section{Chimenea.}
f
\section{Circuito agua vapor.}
f
\section{Ciclo de Rankine.}
f
\section{Caldera acuotubular.}
f
\section{Calderín.}
f
\section{Sobrecalentadores.}
f
\section{Economizador.}
f
\section{Líneas de vapor.}
f
\section{Turbinas de vapor.}
f
\section{Condensadores.}
f
\section{Desgasificador.}
f
\section{Sistema de refrigeración.}
f
\section{Control.}
f
\end{document}