\section{Tema 2: Verificación de equipos de medida eléctricos}
\subsection{Definiciones}
\begin{enumerate}
	\item \underline{\textbf{Patrón}}: 
	\item \underline{\textbf{Patrón primario}}: 
	\item \underline{\textbf{Patrón secundario}}: 
	\item \underline{\textbf{Patrón industrial}}: 
	\item \underline{\textbf{Trazabilidad}}: 
	\item \underline{\textbf{Calibración}}: 
	\item \underline{\textbf{Verificación}}: 
	\item \underline{\textbf{Ajuste}}: 
	\item \underline{\textbf{}}: 
	\item \underline{\textbf{}}: 
	\item \underline{\textbf{}}: 
	\item \underline{\textbf{}}: 
	\item \underline{\textbf{}}: 
	\item \underline{\textbf{}}: 
	\item \underline{\textbf{}}: 
	\item \underline{\textbf{}}: 
	\item \underline{\textbf{}}: 
	\item \underline{\textbf{}}: 
	\item \underline{\textbf{}}: 
	\item \underline{\textbf{}}: 
\end{enumerate}
\subsection{Cualidades de los aparatos de medida patrones}
\subsection{Verificación de un aparato de medida}
\subsection{Criterio de rechazo de Chauvenet}
\subsection{Relación tolerancia incertidumbre}
\subsection{Relación incertidumbre resolución}