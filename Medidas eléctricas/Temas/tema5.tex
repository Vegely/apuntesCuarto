\section{Tema 5. Medidas de impedancia, capacidad y autoinducción en corriente alterna.}
	\subsection{Modelado de elementos pasivos.}
		\subsubsection{Modelo de un condensador real.}
			Un condensador real se puede modelar con un \textbf{condensador ideal en serie o en paralelo} con una \textbf{resistencia representativa de las pérdidas} en los terminales y en el dieléctrico.
			\begin{itemize}
				\item En el modelo serie $R_S$ es muy pequeña.
				\item En el modelo paralelo $R_P$ es muy grande.
			\end{itemize}
			
			Al ensayar un condensador se busca determinar estas dos magnitudes.
			
			\begin{figure}[H]
				\begin{minipage}{0.5\textwidth}
					\subsubsubsection{Modelo en paralelo:}
						En régimen senoidal, debido a $R_P$, el ángulo de desfase $\varphi_C$ es $<90^\circ$. Se llama \textbf{ángulo de pérdidas} a
						\[\delta = 90^\circ - \varphi_C\]
						
						
						A mayor $R_P$ menor $\delta$, y por lo tanto menores pérdidas. Se denomina \textbf{factor de pérdidas} a 
						\[\tan \delta = \dfrac{I_R}{I_C} = \dfrac{G_P}{\omega \cdot C} = \dfrac{1}{R_P\cdot \omega \cdot C}\]
				\end{minipage}%
				\begin{minipage}{0.1\textwidth}
					\textbf{ }
				\end{minipage}%
				\begin{minipage}{0.2\textwidth}
					\begin{figure}[H]
						\centering
						\begin{circuitikz}
							\tikzstyle{every node}=[font=\normalsize]
							\draw (5.25,18.5) to[C,l={ \normalsize $C$}] (5.25,16.25);
							\draw (6.5,18.5) to[R,l={ \normalsize $R_P$}] (6.5,16.25);
							\draw[] (6.5,16.25) to[short] (5.25,16.25);
							\draw[] (6.5,18.5) to[short] (5.25,18.5);
							\draw [](5.25,18.5) to[short, -o] (5.25,19.25) ;
							\draw [](5.25,15.75) to[short, o-] (5.25,16.25) ;
							\node at (5.25,18.5) [circ] {};
							\node at (5.25,16.25) [circ] {};
							\draw [-latex] (5.25,19) -- (5.25,18.75)node[pos=0.5,left]{$I$};
						\end{circuitikz}
					\end{figure}
				\end{minipage}%
				\begin{minipage}{0.2\textwidth}
					\begin{figure}[H]
						\centering
						\begin{circuitikz}
							\begin{scope}[xscale = 1, yscale = 1]
								\draw [-latex] (5.25,12.25) -- (5.25,14.75)node[pos=0.5,left]{$I_C$};
								\draw [-latex] (5.25,12.25) -- (6.25,14.75)node[pos=0.85,left]{$I$};
								\draw [-latex] (5.25,12.25) -- (6.25,12.25)node[pos=1,right]{$I_R$};
								
								\draw [latex-latex] (5.75,12.25) arc (0:70:0.5)node[pos=0.5,right]{$\varphi_C$};
								\draw [latex-latex] (5.75,13.5) arc (70:90:1.5)node[pos=0.5,above]{$\delta$};
							\end{scope}
						\end{circuitikz}
					\end{figure}
				\end{minipage}
			\end{figure}
					
			\begin{figure}[H]
				\begin{minipage}{0.2\textwidth}
					\begin{figure}[H]
						\centering
						\begin{circuitikz}
							\tikzstyle{every node}=[font=\normalsize]
							\draw (5.25,17) to[C,l={ \normalsize C}] (5.25,15.5);
							\draw (5.25,18.5) to[R,l={ \normalsize $R_P$}] (5.25,17);
							\draw [](5.25,18.5) to[short, -o] (5.25,19.25) ;
							\draw [](5.25,15.25) to[short, o-] (5.25,15.5) ;
							\draw [-latex] (5.25,19) -- (5.25,18.75)node[pos=0.5,left]{$I$};						
						\end{circuitikz}
					\end{figure}
				\end{minipage}%
				\begin{minipage}{0.2\textwidth}
					\begin{figure}[H]
						\begin{circuitikz}
							\tikzstyle{every node}=[font=\normalsize]
							\begin{scope}[xscale = 1, yscale = -1]
								\draw [-latex] (5.25,12.25) -- (5.25,14.75)node[pos=0.5,left]{$U_C$};
								\draw [-latex] (5.25,12.25) -- (6.25,14.75)node[pos=0.85,left]{$U$};
								\draw [-latex] (5.25,12.25) -- (6.25,12.25)node[pos=1,right]{$U_R$};
								\draw [latex-latex] (5.75,12.25) arc (0:70:0.5)node[pos=0.5,right]{$\varphi_C$};
								\draw [latex-latex] (5.75,13.5) arc (70:90:1.5)node[pos=0.5,above]{$\delta$};
							\end{scope}
						\end{circuitikz}
					\end{figure}
				\end{minipage}%
				\begin{minipage}{0.1\textwidth}
					\textbf{ }
				\end{minipage}%
				\begin{minipage}{0.5\textwidth}
					\subsubsubsection{Modelo en serie:}
						A veces interesa conocer $R_S$ para luego obtener $R_P$. Esto se hace mediante el ángulo de pérdidas:
						\[R_S \approx X_C \cdot \tan \delta = \dfrac{X_C^2}{R_P}\]
						\[\tan \delta = \dfrac{1}{\omega \cdot C \cdot R_P} \approx \omega \cdot C \cdot R_S\]
						
						
						Estas aproximaciones requieren que $R_P$ sea muy elevada.
				\end{minipage}
			\end{figure}

			\newpage
			\subsubsubsection{Limitaciones de los modelos:}
				En \textbf{baja frecuencia} el modelo paralelo suele ser más adecuado: al tomar $X_C$ valores elevados es más fácil evaluar si el efecto de $R_P$ es importante.
				
				
				En \textbf{alta frecuencia} el modelo serie puede ser más conveniente: como $X_C$ aumenta con la frecuencia, el efecto de $R_S$ puede observarse mejor.
				
				
				Aunque la \textbf{autoinducción} sea baja, en \textbf{alta frecuencia} sus efectos pueden ser importantes.
				
		\subsubsection{Modelo de una bobina real.}
			Todas las bobinas reales llevan asociados fenómenos disipativos de potencia. Las principales causas son:
			\begin{itemize}
				\item $R_\varOmega$: resistencia óhmica de los arrollamientos.
				\item $R_H$: histéresis en los núcleos magnéticos sometidos a magnetizaciones variables.
				\item $R_F$: corrientes parásitas de Foucault inducidas en las partes metálicas.
				\item $R_C$: pérdidas dieléctricas debidas a las capacidades parásitas que forman espiras contiguas.
			\end{itemize}
			
			\begin{figure}[H]
				\centering
					\begin{circuitikz}
						\tikzstyle{every node}=[font=\normalsize]
						\draw (3,15.75) to[R,l={ \normalsize $R_\varOmega$}] (4.5,15.75);
						\draw (4.5,15.75) to[R,l={ \normalsize $R_H$}] (6,15.75);
						\draw (6,15.75) to[R,l={ \normalsize $R_F$}] (7.5,15.75);
						\draw (7.5,15.75) to[R,l={ \normalsize $R_C$}] (9,15.75);
						\draw (9,15.75) to[L,l={ \normalsize $L$} ] (10.5,15.75);
						\draw (10.5,15) to[C,l={ \normalsize $C$}] (9,15);
						\node at (10.5,15.75) [circ] {};
						\node at (9,15.75) [circ] {};
						\draw [](9,15.75) to[short] (9,15);
						\draw [](10.5,15.75) to[short] (10.5,15);
						\draw [](10.5,15.75) to[short, -o] (11,15.75) ;
						\draw [](3,15.75) to[short, -o] (2.5,15.75) ;
					\end{circuitikz}
			\end{figure}
			
			$R_C$ y $C$ son despreciables en baja frecuencia, y $R_H + R_F = R_{Fe}$, denominada \textbf{resistencia de pérdidas en el hierro}. 
			
			
			Ninguna de las resistencias es lineal, dependen de la tensión y la frecuencia. Además, en alta frecuencia hay que tener en cuenta el \textbf{efecto skin}.
			
			
			\begin{figure}[H]
				\begin{minipage}{0.75\textwidth}
					\subsubsubsection{Ensayo de una bobina real.}
						Ensayar una bobina supone determinar los parámetros ideales que la componen:
						\[\tan \delta = \dfrac{U_R}{U_L} = \dfrac{R_{ef}}{\omega \cdot L}\]
						
						
						Donde $R_{ef}$ es la resistencia equivalente de todos los fenómenos disipativos, que generan un ángulo de pérdidas $\delta$.
				\end{minipage}%
				\begin{minipage}{0.25\textwidth}
					\begin{figure}[H]
						\centering
						\begin{circuitikz}
							\begin{scope}[xscale = 1, yscale = 1]
								\draw [-latex] (5.25,12.25) -- (5.25,14.75)node[pos=0.5,left]{$U_L$};
								\draw [-latex] (5.25,12.25) -- (6.25,14.75)node[pos=0.85,left]{$U$};
								\draw [-latex] (5.25,12.25) -- (6.25,12.25)node[pos=1,right]{$U_{R_{ef}}$};
								
								\draw [latex-latex] (5.75,12.25) arc (0:70:0.5)node[pos=0.5,right]{$\varphi_C$};
								\draw [latex-latex] (5.75,13.5) arc (70:90:1.5)node[pos=0.5,above]{$\delta$};
							\end{scope}
						\end{circuitikz}
					\end{figure}
				\end{minipage}
			\end{figure}
		
	\subsection{Medida industrial de capacidades y autoinducciones.}
		\subsubsection{Con voltímetro y amperímetro.}
			\begin{figure}[H]
				\begin{minipage}{0.6\textwidth}
					\begin{figure}[H]
						\scalebox{.9}{
						\centering
						\begin{circuitikz}
							\tikzstyle{every node}=[font=\normalsize]
							\draw [, dashed] (12.75,16) circle (0.5cm) node {\large Hz} ;
							\draw  (14.25,17.5) circle (0.5cm) node {\large A} ;
							\draw  (15.75,16) circle (0.5cm) node {\large V} ;
							\draw (17.25,17.5) to[C,l={ \normalsize $C_X$}] (17.25,14.5);
							\draw[] (12.75,14.5) to[short] (11,14.5);
							\draw[] (15.75,14.5) to[short] (17.25,14.5);
							\draw[] (17.25,17.5) to[short] (14.75,17.5);
							\draw[] (13.75,17.5) to[short] (11,17.5);
							\draw [dashed] (12.75,16.5) -- (12.75,17.5);
							\draw [dashed] (12.75,15.5) -- (12.75,14.5);
							\draw (11,14.5) to[sinusoidal voltage source, sources/symbol/rotate=auto,l={ \normalsize $V_{sin}$}] (11,17.5);
							\draw [](15.75,17.5) to[short] (15.75,16.5);
							\draw [](15.75,15.5) to[short] (15.75,14.5);
							\draw (12.75,17.5) to[short, -*] (12.75,17.5);
							\draw (12.75,14.5) to[short, -*] (12.75,14.5);
							\draw (12.75,14.5) to[R, l=$R_{lim}$] (15.75,14.5);
							\draw (15.75,17.5) to[short, -*] (15.75,17.5);
							\draw (15.75,14.5) to[short, -*] (15.75,14.5);
						\end{circuitikz}
					}
					\end{figure}
				\end{minipage}%
				\begin{minipage}{0.1\textwidth}
					\[\,\]
				\end{minipage}%
				\begin{minipage}{0.3\textwidth}
					No se consideran pérdidas en el condensador. Es útil para determinar capacidades del orden de $>pF$.
					
					
					Se aplica una corriente alterna de \textbf{frecuencia conocida} y se mide la tensión e intensidad:
					\[C_X = \dfrac{I_A}{2\pi f\cdot V_V}\]
				\end{minipage}
			\end{figure}
			
			En función de $R_V$ (resistencia del voltímetro), para valores bajos de capacidad ($X_C \approx R_V$) puede ser más conveniente un montaje largo.
			
			
		\subsubsection{Por comparación de parámetros.}
			
			\begin{figure}[H]
				\begin{minipage}{0.6\textwidth}
					\subsubsubsection{Comparación de tensiones o corrientes.}
						Necesita voltímetro, amperímetro y un condensador patrón, pero \textbf{no hace falta conocer la frecuencia.} Las pérdidas son despreciables.
						
						
						\[V_P\approx \dfrac{I_P}{\omega C_P};\quad V_X\approx\dfrac{I_X}{\omega C_X}\]
				\end{minipage}
				\begin{minipage}{0.4\textwidth}
					\begin{figure}[H]
						\scalebox{.9}{
						\centering
						\begin{circuitikz}
							\tikzstyle{every node}=[font=\normalsize]
							\draw (3,12) to[sinusoidal voltage source, sources/symbol/rotate=auto,l={ \normalsize $V_{sin}$}] (3,14.5);
							\draw (3,14.5) to[rmeter, t=A] (5.25,14.5);
							\draw (5.25,14.5) to[rmeter, t=V] (5.25,12);
							\draw (3,12) to[R] (5.25,12);
							\draw [-latex] (3.75,11.5) -- (4.5,12.5);
							\draw [](5.25,14.5) to[short] (7,14.5);
							\draw (6.5,12) to[C,l={ \normalsize $C_P$}] (6.5,13.25);
							\draw (7.5,13.25) to[C,l={ \normalsize $C_X$}] (7.5,12);
							\draw[] (7.5,12) to[short] (5.25,12);
							\node at (6.5,12) [circ] {};
							\node at (5.25,12) [circ] {};
							\draw [](6.5,13.25) to[short, -o] (6.5,13.5) ;
							\draw [](7.5,13.25) to[short, -o] (7.5,13.5) ;
							\draw [](7,14.25) to[short, o-] (7,14.5) ;
							\draw [-latex] (7,14.25) -- (7.5,13.5);
							\node [font=\normalsize] at (7.75,13.5) {X};
							\node [font=\normalsize] at (6.15,13.5) {P};
							\node at (5.25,14.5) [circ] {};
						\end{circuitikz}
					}
					\end{figure}
				\end{minipage}
			\end{figure}
			
			Operando se llega a:
			\[C_X\approx \dfrac{I_X\cdot V_P}{I_P\cdot V_X}\cdot C_P\]
			
			
			Si $I_P\approx I_X \Rightarrow C_X\approx \dfrac{V_P}{V_X}\cdot C_P$. Si $V_P \approx V_X \Rightarrow C_X\approx \dfrac{I_X}{I_P}\cdot C_P$
			
			
			Si se ajusta $C_P = C_X$ se evita el error sistemático y sólo 3 términos introducen incertidumbre.
				
			\newpage
			\subsubsubsection{Comparación de tensiones.}
				Es necesario determinar o medir la frecuencia. Tampoco se consideran las pérdidas.
				
				
				Si la frecuencia es conocida basta con un voltímetro y una resistencia patrón.
				
				\begin{figure}[H]
					\begin{minipage}{0.5\textwidth}
						\[Z_C \approx X_C = \dfrac{1}{2\pi f \cdot C_X}\]
						\[\dfrac{V_C}{V_R}\approx\dfrac{X_C}{R_P} = \dfrac{1}{2\pi f \cdot C_X \cdot R_P}\]
						\[C_X \approx \dfrac{1}{2\pi f \cdot R_P}\cdot \dfrac{V_R}{V_C}\]
					\end{minipage}
					\begin{minipage}{0.5\textwidth}
						\begin{figure}[H]
							\centering
							\begin{circuitikz}
								\tikzstyle{every node}=[font=\normalsize]
								\draw (4,19.25) to[sinusoidal voltage source, sources/symbol/rotate=auto,l={ \normalsize $V_{sin}$}] (4,15.25);
								\draw [, dashed] (6,17.25) circle (0.5cm) node {\normalsize Hz} ;
								\draw [dashed] (6,17.75) -- (6,19.25);
								\draw [dashed] (6,16.75) -- (6,15.25);
								\draw (6,19.25) to[R] (8.25,19.25);
								\draw [-latex] (6.75,18.75) -- (7.5,19.75);
								\draw  (8.25,18.25) circle (0.5cm) node {\normalsize $V_R$} ;
								\draw  (8.25,16.25) circle (0.5cm) node {\normalsize $V_C$} ;
								\draw [](8.25,17.75) to[short] (8.25,17.25);
								\draw [](8.25,18.75) to[short] (8.25,19.25);
								\draw (9.5,19.25) to[R,l={ \normalsize $R_P$}] (9.5,17.25);
								\draw (9.5,17.25) to[C,l={ \normalsize $C_X$}] (9.5,15.25);
								\draw [](8.25,15.75) to[short] (8.25,15.25);
								\draw [](8.25,17.25) to[short] (9.5,17.25);
								\draw[] (9.5,19.25) to[short] (8.25,19.25);
								\draw[] (9.5,15.25) to[short] (4,15.25);
								\draw[] (6,19.25) to[short] (4,19.25);
								\node at (6,19.25) [circ] {};
								\node at (6,15.25) [circ] {};
								\node at (8.25,15.25) [circ] {};
								\node at (8.25,17.25) [circ] {};
								\node at (9.5,17.25) [circ] {};
								\node at (8.25,19.25) [circ] {};
								\draw [](8.25,17.25) to[short] (8.25,16.75);
							\end{circuitikz}
						\end{figure}
					\end{minipage}
				\end{figure}
				
				Hay 4 términos que introducen incertidumbre. En este caso $V_R = V_C$ \textbf{no evita el error sistemático.}
				
				
				Sustituyendo $R_P$ por un $C_P$ no es necesario medir la frecuencia.				
			
		\subsubsection{Método de las tres tensiones.}
			\begin{figure}[H]
				\centering
					\begin{circuitikz}
						\tikzstyle{every node}=[font=\normalsize]
						\draw (4,19.25) to[sinusoidal voltage source, sources/symbol/rotate=auto,l={ \normalsize $V_{sin}$}] (4,15.25);
						\draw  (8.25,18.25) circle (0.5cm) node {\normalsize $V_R$} ;
						\draw  (8.25,16.25) circle (0.5cm) node {\normalsize $V_Z$} ;
						\draw [](8.25,17.75) to[short] (8.25,17.25);
						\draw [](8.25,18.75) to[short] (8.25,19.25);
						\draw (9.5,19.25) to[R,l={ \normalsize $R_P$}] (9.5,17.25);
						\draw [](8.25,15.75) to[short] (8.25,15.25);
						\draw [](8.25,17.25) to[short] (9.5,17.25);
						\draw[] (9.5,19.25) to[short] (8.25,19.25);
						\draw[] (9.5,15.25) to[short] (4,15.25);
						\node at (8.25,15.25) [circ] {};
						\node at (8.25,17.25) [circ] {};
						\node at (9.5,17.25) [circ] {};
						\node at (8.25,19.25) [circ] {};
						\draw [](8.25,17.25) to[short] (8.25,16.75);
						\ctikzset{resistor = european}
						\draw (9.5,17.25) to[R,l={ \normalsize $Z_X$}] (9.5,15.25);
						\draw [, dashed] (5.25,19.25) circle (0.5cm) node {\normalsize A} ;
						\draw  (6.75,17.25) circle (0.5cm) node {\normalsize $V_T$} ;
						\draw [](6.75,17.75) to[short] (6.75,19.25);
						\draw [](5.75,19.25) to[short] (8.25,19.25);
						\draw [](4,19.25) to[short] (4.75,19.25);
						\draw [](6.75,16.75) to[short] (6.75,15.25);
						\node at (6.75,15.25) [circ] {};
						\node at (6.75,19.25) [circ] {};
					\end{circuitikz}
			\end{figure}
			
			Sirve para cualquier tipo de impedancia. Se necesita una resistencia patrón $R_P$ en serie o bien medir la intensidad.
			
			
			Hay que medir las tensiones en la $R_P$, $Z_X$ y la total, correspondientes a los voltímetros $V_R$, $V_Z$ y $V_T$, respectivamente.
			
			
			El diagrama fasorial permite deducir el valor de $L$ ó $C$ y la $R$ serie efectiva en c.a.
			
			
			En la práctica sus resultados son poco fiables si las medidas no son de alta calidad (baja incertidumbre de los aparatos.)
			
			
			\subsubsubsection{Caso de $\mathbf{Z_X}\,$inductiva.}
				Del triángulo de tensiones deducimos:
				\[V_T^2 = (V_R + V_{R_Z})^2 + V_{X_Z}^2 = V_R^2 + V_Z^2 + 2V_RV_Z\cos \varphi\]
				
				
				Esta ecuación permite obtener el factor de potencia:
				\[\cos \varphi = \dfrac{V_T^2 - V_R^2 - V_Z^2}{2V_RV_Z}\]
				
				\begin{figure}[H]
					\begin{minipage}{0.6\textwidth}
						\begin{figure}[H]
							\centering
							\begin{circuitikz}
								\tikzstyle{every node}=[font=\normalsize]
								\draw [-latex] (1.5,13) -- (5.25,13)node[pos=0.5,below]{$V_R = I\cdot R_P$};
								\draw [-latex] (5.25,13) -- (7.5,13)node[pos=0.5,below]{$V_{R_Z}=I\cdot R_Z$};
								\draw [-latex] (7.5,13) -- (7.5,16.25)node[pos=0.5,below,sloped]{$V_{X_Z}=X_L\cdot I$};
								\draw [-latex] (1.5,13) -- (7.5,16.25)node[pos=0.5,above]{$V_T$};
								\draw [-latex] (5.25,13) -- (7.5,16.25)node[pos=0.5,left]{$V_Z$};
								\draw (5.75, 13) arc (0:55:0.5)node[pos=0.5,right]{$\varphi$};
								\draw (7.5,15.75) arc (270:235:0.5)node[pos=0.5,below]{$\delta$};
							\end{circuitikz}
						\end{figure}
					\end{minipage}
					\begin{minipage}{0.4\textwidth}
						\[
						\left.
						\begin{matrix}
							Z = \dfrac{V_Z}{I}\\
							\,\\
							I = \dfrac{V_R}{R_P}
						\end{matrix}
						\right\} \Rightarrow
						Z = \dfrac{R_P\cdot V_Z}{V_R}
						\]
						\[R_Z = Z\cdot \cosphi\]
						\[X_L = \sqrt{Z_X^2 - R_Z^2};\quad L=\dfrac{X_L}{2\pi f}\]
					\end{minipage}
				\end{figure}
				
		\subsubsection{Método de las tres intensidades.}
			Es dual al método de las tres tensiones.
			
			
			\subsubsubsection{Caso de $\mathbf{Z_X}\,$inductiva.}
			\begin{figure}[H]
				\begin{minipage}{0.6\textwidth}
					\begin{figure}[H]
						\centering
						\begin{circuitikz}[scale=0.7]
							\tikzstyle{every node}=[font=\normalsize]
							\begin{scope}[xscale = 1, yscale = -1]
								\draw [-latex] (1.5,13) -- (5.25,13)node[pos=0.5,above]{$I_R$};
								\draw [dashed] (5.25,13) -- (7.5,13);
								\draw [dashed] (7.5,13) -- (7.5,16.25);
								\draw [-latex] (1.5,13) -- (7.5,16.25)node[pos=0.5,below]{$I_T$};
								\draw [-latex] (5.25,13) -- (7.5,16.25)node[pos=0.4,right]{$I_Z$};
								\draw (5.75, 13) arc (0:55:0.5)node[pos=0.75,right]{$\varphi$};
								\draw (7.5,15.75) arc (270:235:0.5)node[pos=0.75,above]{$\delta$};
							\end{scope}
						\end{circuitikz}
						\vspace{0.25cm}
						\centering
						\begin{circuitikz}
							\tikzstyle{every node}=[font=\normalsize]
							\draw (4,19.25) to[sinusoidal voltage source, sources/symbol/rotate=auto,l={ \normalsize $V_{sin}$}] (4,15.25);
							\draw  (8.5,18) circle (0.5cm) node {\normalsize $A_R$} ;
							\draw  (10,18) circle (0.5cm) node {\normalsize $A_Z$} ;
							\draw (8.5,17.5) to[R,l={ \normalsize $R_P$}] (8.5,15.25);
							\draw[] (10,15.25) to[short] (4,15.25);
							\ctikzset{resistor=european}
							\draw (10,17.5) to[R,l={ \normalsize $Z_X$}] (10,15.25);
							\draw  (6,17.25) circle (0.5cm) node {\normalsize V} ;
							\draw [](6,17.75) to[short] (6,19.25);
							\draw [](4,19.25) to[short] (6.75,19.25);
							\draw [](6,16.75) to[short] (6,15.25);
							\draw  (7.25,19.25) circle (0.5cm) node {\normalsize A} ;
							\draw [](7.75,19.25) to[short] (10,19.25);
							\draw [](10,19.25) to[short] (10,18.5);
							\draw [](8.5,19.25) to[short] (8.5,18.5);
							\node at (6,19.25) [circ] {};
							\node at (8.5,19.25) [circ] {};
							\node at (6,15.25) [circ] {};
							\node at (8.5,15.25) [circ] {};
						\end{circuitikz}
					\end{figure}
				\end{minipage}
				\begin{minipage}{0.4\textwidth}
					Del triángulo de intensidades obtenemos:
					\[I_T^2 = I_R^2+I_Z^2+2I_RI_Z\cosphi\]
					\[\cosphi = \dfrac{I_T^2-I_R^2-I_Z^2}{2I_RI_Z}\]
				\end{minipage}
			\end{figure}
		
		\subsubsection{Método de Joubert.}
			Supone que la $R_{ef}$ de c.a. es igual a la de c.c. ($R_{ef} \approx R_{cc}$), simplificación válida para \textbf{bobinas sin núcleo o trabajando a muy baja frecuencia.}
			
			
			Hay que medir en c.a. la $Z_X$ y en c.c. la $R_{cc}$.
			\[R_{cc} = \dfrac{V_{cc}}{I_{cc}};\quad Z_X = \dfrac{V_{ca}}{I_{ca}} = \sqrt{R_{cc}^2 + L_X^2\omega^2}\]
			
			\[L_X = \dfrac{1}{2\pi f}\sqrt{\dfrac{V_{ca}^2}{I_{ca}^2} - \dfrac{V_{cc}^2}{I_{cc}^2}}\]
			
			\begin{figure}[H]
				\centering
				\begin{circuitikz}
					\tikzstyle{every node}=[font=\large]
					\draw [, dashed] (12.75,16) circle (0.5cm) node {\large Hz} ;
					\draw  (14.25,17.5) circle (0.5cm) node {\large A} ;
					\draw  (15.75,16) circle (0.5cm) node {\large V} ;
					\ctikzset{resistor = european}
					\draw (17.25,17.5) to[R,l={ \large $Z_X$}] (17.25,14.5);
					\draw[] (17.25,14.5) to[short] (11,14.5);
					\draw[] (17.25,17.5) to[short] (14.75,17.5);
					\draw[] (13.75,17.5) to[short] (11,17.5);
					\draw [dashed] (12.75,16.5) -- (12.75,17.5);
					\draw [dashed] (12.75,15.5) -- (12.75,14.5);
					\draw (11,14.5) to[sinusoidal voltage source, sources/symbol/rotate=auto,l={ \normalsize $V_{cc/ca}$}] (11,17.5);
					\draw [](15.75,17.5) to[short] (15.75,16.5);
					\draw [](15.75,15.5) to[short] (15.75,14.5);
					\draw (12.75,17.5) to[short, -*] (12.75,17.5);
					\draw (12.75,14.5) to[short, -*] (12.75,14.5);
					\draw (15.75,17.5) to[short, -*] (15.75,17.5);
					\draw (15.75,14.5) to[short, -*] (15.75,14.5);
				\end{circuitikz}
			\end{figure}
		
		\subsubsection{Con voltímetro, amperímetro y vatímetro.}
			Adecuado para bobinas con núcleo de hierro trabajando a tensión y frecuencia constante (resultados válidos sólo para la $U$ y $f$ del ensayo). Se basa en suponer que toda la potencia es disipada por la resistencia efectiva en c.a. de la bobina, por lo que \textbf{la resistencia que se obtiene está sobreestimada}.
			
			
			Los aparatos deben conectarse minimizando sus errores de inserción (montaje largo o corto). Es muy importante que el campo de medida del vatímetro sea muy próximo al valor de la medida.
			
			
			Se suele utilizar para determinar los parámetros de máquinas eléctricas.
			
			\subsubsubsection{$\mathbf{Z_X}$ y $\mathbf{R_X}$ altas.}
				\[\vec U_Z = \vec U_X + \vec I_A \cdot R_A\]
				
				
				Si $R_A <\!< Z_X \Rightarrow U_Z \approx V_X$
				
				\begin{figure}[H]
					\centering
					\begin{circuitikz}
						\tikzstyle{every node}=[font=\normalsize]
						\draw [, dashed] (12.75,16) circle (0.5cm) node {\large Hz} ;
						\draw  (17.75,17.5) circle (0.5cm) node {\large A} ;
						\draw  (16.25,16) circle (0.5cm) node {\large V} ;
						\ctikzset{resistor = european}
						\draw (19.25,17.5) to[R,l={ \large $Z_X$}] (19.25,14.5);
						\draw[] (19.25,14.5) to[short] (11,14.5);
						\draw[] (17.25,17.5) to[short] (16.25,17.5);
						\draw[] (13.75,17.5) to[short] (11,17.5);
						\draw [dashed] (12.75,16.5) -- (12.75,17.5);
						\draw [dashed] (12.75,15.5) -- (12.75,14.5);
						\draw (11,14.5) to[sinusoidal voltage source, sources/symbol/rotate=auto,l={ \normalsize $V_{sin}$}] (11,17.5);
						\draw [](16.25,17.5) to[short] (16.25,16.5);
						\draw [](16.25,15.5) to[short] (16.25,14.5);
						\draw (12.75,17.5) to[short, -*] (12.75,17.5);
						\draw (12.75,14.5) to[short, -*] (12.75,14.5);
						\draw (16.25,17.5) to[short, -*] (16.25,17.5);
						\draw (16.25,14.5) to[short, -*] (16.25,14.5);
						\draw  (14.25,17.5) circle (0.5cm) node {\large W} ;
						\draw [](14.75,17.5) to[short] (16.25,17.5);
						\draw [](14.25,18) to[short] (14.25,18.25);
						\draw [](13.5,18.25) to[short] (14.25,18.25);
						\draw [](13.5,18.25) to[short] (13.5,17.5);
						\draw [](14.25,17) to[short] (14.25,14.5);
						\draw (13.5,17.5) to[short, -*] (13.5,17.5);
						\draw (14.25,14.5) to[short, -*] (14.25,14.5);
						\draw [-latex] (19,17.5) -- (19.1,17.5)node[pos=0.5,above]{$I_A=I_X$};
						\draw [](18.25,17.5) to[short] (19.25,17.5);
						\draw [-latex] (15.5,16.5) -- (15.5,15.5)node[pos=0.5,left]{$U$};
						\draw [-latex] (18.75,16.5) -- (18.75,15.5)node[pos=0.5,left]{$U_X$};
					\end{circuitikz}
				\end{figure}
				
			\subsubsubsection{$\mathbf{Z_X}$ y $\mathbf{R_X}$ bajas.}
				\[\vec I_A = \vec I_X + \vec U_A \cdot R_V\]
				
				
				Si $R_V >\!> Z_X \Rightarrow I_A \approx I_X$
				
				\begin{figure}[H]
					\centering
					\begin{circuitikz}
						\tikzstyle{every node}=[font=\large]
						\draw [, dashed] (12.75,16) circle (0.5cm) node {\large Hz} ;
						\draw  (16.25,17.5) circle (0.5cm) node {\large A} ;
						\draw  (17.75,16) circle (0.5cm) node {\large V} ;
						\ctikzset{resistor = european}
						\draw (19.5,17.5) to[R,l={ \large $Z_X$}] (19.5,14.5);
						\draw[] (19.5,14.5) to[short] (11,14.5);
						\draw[] (19.5,17.5) to[short] (16.75,17.5);
						\draw[] (13.75,17.5) to[short] (11,17.5);
						\draw [dashed] (12.75,16.5) -- (12.75,17.5);
						\draw [dashed] (12.75,15.5) -- (12.75,14.5);
						\draw (11,14.5) to[sinusoidal voltage source, sources/symbol/rotate=auto,l={ \normalsize $V_{sin}$}] (11,17.5);
						\draw [](17.75,17.5) to[short] (17.75,16.5);
						\draw [](17.75,15.5) to[short] (17.75,14.5);
						\draw (12.75,17.5) to[short, -*] (12.75,17.5);
						\draw (12.75,14.5) to[short, -*] (12.75,14.5);
						\draw (17.75,17.5) to[short, -*] (17.75,17.5);
						\draw (17.75,14.5) to[short, -*] (17.75,14.5);
						\draw  (14.25,17.5) circle (0.5cm) node {\large W} ;
						\draw [](14.75,17.5) to[short] (15.75,17.5);
						\draw [](14.25,18) to[short] (14.25,18.25);
						\draw [](14.25,18.25) to[short] (15,18.25);
						\draw [](15,18.25) to[short] (15,17.5);
						\draw [](14.25,17) to[short] (14.25,14.5);
						\draw (15,17.5) to[short, -*] (15,17.5);
						\draw (14.25,14.5) to[short, -*] (14.25,14.5);
						\draw [-latex] (17.25,17.5) -- (17.3,17.5)node[pos=0.5,above]{$I_A$};
						\draw [-latex] (17.75,15.25) -- (17.75,15)node[pos=0.5,left]{$I_V$};
						\draw [-latex] (18.5,17.5) -- (18.75,17.5)node[pos=0.5,above]{$I_X$};
					\end{circuitikz}
				\end{figure}
				
			\[Z_X \approx \dfrac{V_V}{I_A};\quad P_Z \approx P_W \approx I_A^2 \cdot R_X\]
			\[R_X \approx \dfrac{P_W}{I_A^2}; \quad X_X = \sqrt{Z_X^2 - R_X^2}\]
		
	\subsection{Métodos de laboratorio para la medida de L y C.}
		\subsubsection{Puentes de corriente alterna.}
			Al estar formados por impedancias su equilibrado es mucho más difícil que el de Wheatstone.
			\[\vec Z_i = R_i + jX_i\]
			
			
			En el equilibrio, $V_V = 0$:
			\[\vec I_A\cdot \vec Z_A = \vec I_B\cdot \vec Z_B; \qquad
			\vec I_A\cdot \vec Z_C = \vec I_B\cdot \vec Z_D \]
			
			Estas ecuaciones implican que deben cumplirse simultáneamente:
			\[\dfrac{\vec Z_A}{\vec Z_C} = \dfrac{\vec Z_B}{\vec Z_D} \Rightarrow Z_A\cdot Z_D = Z_B \cdot Z_C\]
			\[\varphi_A + \varphi_D = \varphi_B + \varphi_C\]
			
			\begin{figure}[H]
				\begin{minipage}{0.6\textwidth}
					\begin{figure}[H]
						\centering
						\begin{circuitikz}
							\tikzstyle{every node}=[font=\normalsize]
							\ctikzset{resistor=european}
							\draw (2.5,19.75) to[sinusoidal voltage source, sources/symbol/rotate=auto,l={ \normalsize $V_{sin}$}] (2.5,14);
							\draw (4.75,19.75) to[R,l={ \normalsize $Z_A$}] (4.75,17);
							\draw (4.75,17) to[R,l={ \normalsize $Z_C$}] (4.75,14);
							\draw (7,19.75) to[R,l={ \normalsize $Z_B$}] (7,17);
							\draw (7,17) to[R,l={ \normalsize $Z_X$}] (7,14);
							\draw (4.75,17) to[rmeter, t=V] (7,17);
							\node at (4.75,17) [circ] {};
							\node at (7,17) [circ] {};
							\node at (4.75,19.75) [circ] {};
							\node at (4.75,14) [circ] {};
							\draw[] (7,14) to[short] (2.5,14);
							\draw [](2.5,19.75) to[short] (7,19.75);
							\draw [-latex] (4.25,18) -- (5.25,19);
							\draw [-latex] (6.5,18) -- (7.5,19);
							\draw [-latex] (4.25,15) -- (5.25,16);
							\draw [-latex] (6.5,15) -- (7.5,16);
							\node [font=\normalsize] at (4.75,20) {1};
							\node [font=\normalsize] at (4.75,13.75) {2};
							\node [font=\normalsize] at (4.5,17) {3};
							\node [font=\normalsize] at (7.25,17) {4};
						\end{circuitikz}
					\end{figure}
				\end{minipage}
				\begin{minipage}{0.4\textwidth}
					Equilibrar este puente puede requerir el ajuste de hasta 6 variables, haciéndolo complejo. Para facilitar esta tarea se suelen eliminar variables, haciendo que algunas impedancias sólo tengan parte real o parte imaginaria (sólo $R$ o sólo $L/C$), o sean del mismo tipo (inductivas o capacitivas).
					
					
					Como detectores de cero se emplean instrumentos de alta resolución e impedancia, como osciloscopios.
				\end{minipage}
			\end{figure}
		
		\subsubsection{Puente de Maxwell.}
			\begin{figure}[H]
				\begin{minipage}{0.6\textwidth}
					\begin{figure}[H]
						\centering
						\begin{circuitikz}
							\tikzstyle{every node}=[font=\normalsize]
							\draw (10,20) to[sinusoidal voltage source, sources/symbol/rotate=auto,l={ \normalsize $V_{sin}$}] (10,13);
							\draw (12,20) to[R] (12,16.5);
							\draw (12,16.5) to[R] (12,13);
							\draw (14.5,20) to[L ] (14.5,18.25);
							\draw (14.5,18.25) to[R] (14.5,16.5);
							\draw (14.5,16.5) to[R] (14.5,14.75);
							\draw (14.5,14.75) to[L ] (14.5,13);
							\draw (12,16.5) to[rmeter, t=V] (14.5,16.5);
							\draw (14.5,13) to[R] (12,13);
							\draw (12,20) to[R] (14.5,20);
							\draw [](10,20) to[short] (12,20);
							\draw [](10,13) to[short] (12,13);
							\node at (12,13) [circ] {};
							\node at (12,16.5) [circ] {};
							\node at (12,20) [circ] {};
							\node at (14.5,16.5) [circ] {};
							\draw [, dashed] (12.5,20.75) rectangle  (15,18.25);
							\draw [, dashed] (12.5,14.75) rectangle  (15,12.5);
							\draw [-latex] (11.5,17.75) -- (12.5,18.75);
							\draw [-latex] (11.5,14.25) -- (12.5,15.25);
							\draw [-latex] (14,15) -- (15,16.25);
							\draw [-latex] (14,16.75) -- (15,18);
							\node [font=\normalsize] at (11.5,18.25) {$R_A$};
							\node [font=\normalsize] at (11.5,14.75) {$R_C$};
							\node [font=\normalsize] at (13.25,19.5) {$R_P$};
							\node [font=\normalsize] at (14,19) {$L_P$};
							\node [font=\normalsize] at (15.25,17.5) {$\Delta R_P$};
							\node [font=\normalsize] at (15.25,15.5) {$\Delta R_X$};
							\node [font=\normalsize] at (14,14) {$L_X$};
							\node [font=\normalsize] at (13.25,13.5) {$R_X$};
						\end{circuitikz}
					\end{figure}
				\end{minipage}
				\begin{minipage}{0.4\textwidth}
					En teoría permite medir inductancias de bobinas con cualquier factor de calidad:
					\[Q = \dfrac{X_L}{R}\]
					
					
					En el equilibrio, $V_V = 0$, luego
					\begin{equation}\label{eq:1}
						\vec I_A\cdot R_A = \vec I_B \cdot (R'_P + j\omega L_P)
					\end{equation}
					\begin{equation}\label{eq:2}
						\vec I_A\cdot R_C = \vec I_B \cdot (R'_X + j\omega L_X)
					\end{equation}
					\[R'_P = R_P + \Delta R_P\]
					\[R'_X = R_X + \Delta R_X\]
					
					
					Dividiendo \eqref{eq:1} entre \eqref{eq:2}:
					\[R'_X = \dfrac{R_C}{R_A}\cdot R'_P; \quad L_X = \dfrac{R_C}{R_A}\cdot L_P\]
					
					
					Se observa que los resultados no dependen de la frecuencia. El ajuste es lento e iterativo.
				\end{minipage}
			\end{figure}

		\newpage	
		\subsubsubsection{Proceso de ajuste.}
			\begin{enumerate}
				\item Se reduce $R_C$ o aumenta $R_A$ hasta minimizar $V_V$.
				\item Se aumenta $\Delta R_P$ hasta minimizar de nuevo $V_V$, y así hasta conseguir que $\varphi_{P1} < \varphi_P$ y que $\varphi_{P1}\rightarrow\varphi_X$.
				\item Se vuelve a reducir $R_C$ hasta minimizar de nuevo $V_V$.
				\item Se aumenta de nuevo $\Delta R_P$ hasta minimizar $V_V$ y hacer que $\varphi_{P2} < \varphi_{P1}$ y que $\varphi_{P2}\rightarrow\varphi_X$
			\end{enumerate}
			
			Se repite el proceso de ajuste hasta que $V_V = 0$.	
			
			
			\begin{figure}[H]
				\begin{minipage}{0.4\textwidth}
					\begin{figure}[H]
						\centering
						\begin{circuitikz}
							\tikzstyle{every node}=[font=\normalsize]
							\draw [ color={rgb,255:red,0; green,128; blue,255}, -latex] (1.25,11.5) -- (3.5,11.5)node[pos=1,right]{$\vec U$};
							\draw [ color={rgb,255:red,0; green,128; blue,255}, -latex] (1.25,11.5) -- (3.5,12.5)node[pos=0.6,above, sloped]{$I_X \cdot Z_X$};
							\draw [-latex] (1.25,11.5) -- (2.5,13.5)node[pos=0.5,above, sloped]{$I_C \cdot R_C$};
							\draw [ color={rgb,255:red,0; green,128; blue,0}, -latex] (2.5,13.5) -- (4.75,17)node[pos=0.5,above, sloped]{$I_A \cdot R_A$};
							\draw [ color={rgb,255:red,255; green,0; blue,0}, -latex] (3.5,12.5) -- (4.75,12.5)node[pos=1,right]{$\vec U$};
							\draw [ color={rgb,255:red,255; green,0; blue,0}, short] (4.75,12.5) -- (4.75,17);
							\draw [ color={rgb,255:red,255; green,0; blue,0}, -latex] (3.5,12.5) -- (4.75,17)node[pos=0.5,above,sloped]{$I_P \cdot Z_P$};
							\draw [ color={rgb,255:red,0; green,128; blue,255}, short] (3.5,11.5) -- (3.5,12.5);
							\draw [ color={rgb,255:red,128; green,0; blue,255}, latex-latex] (3.5,12.5) -- (2.5,13.5)node[pos=0.5,above,sloped]{$V_V$};
							
						\end{circuitikz}
					\end{figure}
				\end{minipage}%
				\begin{minipage}{0.1\textwidth}
					\centering \[\Rightarrow\]
				\end{minipage}%
				\begin{minipage}{0.4\textwidth}
					\begin{figure}[H]
						\centering
						\begin{circuitikz}
							\tikzstyle{every node}=[font=\normalsize]
							\draw [ color={rgb,255:red,0; green,128; blue,255}, -latex] (1.25,11.5) -- (3.5,11.5)node[pos=1,right]{$\vec U$};
							\draw [ color={rgb,255:red,0; green,128; blue,255}, -latex] (1.25,11.5) -- (3.5,12.5)node[pos=0.6,below, sloped]{$I_X \cdot Z_X$};
							\draw [-latex] (1.25,11.5) -- (3,13)node[pos=0.5,above, sloped]{$I_C \cdot R_C$};
							\draw [ color={rgb,255:red,0; green,128; blue,0}, -latex] (3,13) -- (6,15.75)node[pos=0.5,above, sloped]{$I_A \cdot R_A$};
							\draw [ color={rgb,255:red,255; green,0; blue,0}, -latex] (3.5,12.5) -- (6,12.5)node[pos=1,right]{$\vec U$};
							\draw [ color={rgb,255:red,255; green,0; blue,0}, short] (6,12.5) -- (6,15.75);
							\draw [ color={rgb,255:red,255; green,0; blue,0}, -latex] (3.5,12.5) -- (6,15.75)node[pos=0.2,above,sloped]{$I_P \cdot Z_P$};
							\draw [ color={rgb,255:red,0; green,128; blue,255}, short] (3.5,11.5) -- (3.5,12.5);
							\draw [ color={rgb,255:red,128; green,0; blue,255}, latex-latex] (3.5,12.5) -- (3,13)node[pos=0.5,below,sloped]{$V_V$};
						\end{circuitikz}
					\end{figure}
				\end{minipage}%
				\begin{minipage}{0.1\textwidth}
					\centering \[\Rightarrow\]
				\end{minipage}%
			\end{figure}
			
			\begin{figure}[H]
				\centering
				\begin{minipage}{0.2\textwidth}
				\centering \[\Rightarrow\]
				\end{minipage}%
				\begin{minipage}{0.4\textwidth}
					\begin{figure}[H]
						\centering
						\begin{circuitikz}
							\tikzstyle{every node}=[font=\normalsize]
							\draw [ color={rgb,255:red,0; green,128; blue,255}, -latex] (1.25,11.5) -- (3.5,11.5)node[pos=1,right]{$\vec U$};
							\draw [ color={rgb,255:red,0; green,128; blue,255}, -latex] (1.25,11.5) -- (3.5,12.5)node[pos=0.6,below, sloped]{$I_X \cdot Z_X$};
							\draw [-latex] (1.25,11.5) -- (3.25,13)node[pos=0.5,above, sloped]{$I_C \cdot R_C$};
							\draw [ color={rgb,255:red,0; green,128; blue,0}, -latex] (3.25,13) -- (6,14.25)node[pos=0.5,above, sloped]{$I_A \cdot R_A$};
							\draw [ color={rgb,255:red,255; green,0; blue,0}, -latex] (3.5,12.5) -- (6,12.5)node[pos=1,right]{$\vec U$};
							\draw [ color={rgb,255:red,255; green,0; blue,0}, short] (6,12.5) -- (6,14.25);
							\draw [ color={rgb,255:red,255; green,0; blue,0}, -latex] (3.5,12.5) -- (6,14.25)node[pos=0.5,below,sloped]{$I_P \cdot Z_P$};
							\draw [ color={rgb,255:red,0; green,128; blue,255}, short] (3.5,11.5) -- (3.5,12.5);
							\draw [ color={rgb,255:red,128; green,0; blue,255}, latex-latex] (3.5,12.5) -- (3.25,13)node[pos=0.5,above,sloped]{$V_V$};
						\end{circuitikz}
					\end{figure}
				\end{minipage}
			\end{figure}
		
		\newpage
		\subsubsection{Puente de Wien.}
			Aunque permite hallar capacidades desconocidas, su uso principal es \textbf{determinar frecuencias}. También se utiliza como \textbf{filtro paso banda}, al separar fácilmente las frecuencias dadas por el ajuste de sus parámetros.
			
			
			Para facilitar los ajustes las resistencias $R_A$ y $R_C$ se eligen de forma que su relación sea un número entero, generalmente 2. En el caso de emplearse para obtener capacidades desconocidas el condensador y la resistencia patrón, deben ser $C_D$ y $R_D$.
			
			\begin{figure}[H]
				\centering
					\begin{circuitikz}
						\tikzstyle{every node}=[font=\normalsize]
						\draw (2.5,19.75) to[sinusoidal voltage source, sources/symbol/rotate=auto,l={ \normalsize $V_{sin}$}] (2.5,14);
						\draw (4.75,19.75) to[R,l={ \normalsize $R_A$}] (4.75,17);
						\draw (4.75,17) to[R,l={ \normalsize $R_C$}] (4.75,14);
						\draw (7,19.75) to[R,l={ \normalsize $R_B$}] (7,18.25);
						\draw (8,17) to[R,l={ \normalsize $R_D$}] (8,14);
						\draw (4.75,17) to[rmeter, t=V] (7,17);
						\node at (4.75,17) [circ] {};
						\node at (7,17) [circ] {};
						\node at (4.75,19.75) [circ] {};
						\node at (4.75,14) [circ] {};
						\draw[] (7,14) to[short] (2.5,14);
						\draw [](2.5,19.75) to[short] (7,19.75);
						\draw [-latex] (4.25,18) -- (5.25,19);
						\draw [-latex] (6.5,18.5) -- (7.5,19.5);
						\draw [-latex] (4.25,15) -- (5.25,16);
						\draw [-latex] (7.5,15) -- (8.5,16);
						\node [font=\normalsize] at (4.75,20) {1};
						\node [font=\normalsize] at (4.75,13.75) {2};
						\node [font=\normalsize] at (4.5,17) {3};
						\node [font=\normalsize] at (7.25,16.75) {4};
						\draw (7,18.25) to[C,l={ \normalsize $C_B$}] (7,17);
						\draw (7,14) to[C,l={ \normalsize $C_D$}] (7,17);
						\draw[] (8,17) to[short] (7,17);
						\draw[] (8,14) to[short] (7,14);
						\node at (7,14) [circ] {};
					\end{circuitikz}
			\end{figure}
			
			\[\vec I_A\cdot R_A = \vec I_B \cdot \left(R_B + \dfrac{1}{j\omega C_B}\right) = \dfrac{\vec I_B\cdot (1 + j\omega R_BC_B)}{j\omega C_B}\]
			
			\[\vec I_A\cdot R_C = \dfrac{\vec I_B}{\dfrac{1}{R_B} + \dfrac{1}{j\omega C_D}} = \dfrac{\vec I_B\cdot R_D)}{1+j\omega R_D C_D}\]
			
			
			Operando se llega a:
			\[\dfrac{R_A}{R_C} = \dfrac{R_B}{R_D} + \dfrac{C_D}{C_B}\]
			\[\omega R_B C_D = \dfrac{1}{\omega R_D C_B} \Rightarrow f = \dfrac{1}{2\pi \sqrt{R_BC_BR_DC_D}}\]
		
		\newpage
		\subsubsection{Puente de Maxwell-Wien.}
			Se emplea para \textbf{medir inductancias.} Se sustituye la inductancia patrón por una capacidad patrón de bajas pérdidas. La resistencia ajustable en paralelo con el condensador patrón limita su uso, pues para $Q$ muy altos debe ser muy elevada.
			
			
			En la práctica ofrece buen resultado para bobinas con $1 < Q < 10$. Para frecuencias bajas sus parámetros pueden considerarse independientes de esa variable.
			
			
			Una ventaja de estos dos puentes es que los ajustes son independientes de la frecuencia.
			
			\begin{figure}[H]
				\centering
				\begin{circuitikz}
					\tikzstyle{every node}=[font=\LARGE]
					\draw (10,20) to[sinusoidal voltage source, sources/symbol/rotate=auto,l={ \normalsize $V_{sin}$}] (10,13);
					\draw (12,20) to[R] (12,16.5);
					\draw (12.5,16) to[R] (12.5,13.5);
					\draw (14.5,18.25) to[L ] (14.5,16.5);
					\draw (12,16.5) to[rmeter, t=V] (14.5,16.5);
					\draw [](10,20) to[short] (12,20);
					\draw [](10,13) to[short] (12,13);
					\node at (12,13) [circ] {};
					\node at (12,16.5) [circ] {};
					\node at (12,20) [circ] {};
					\node at (14.5,16.5) [circ] {};
					\draw [, dashed] (13.75,20.25) rectangle  (15.25,16.75);
					\draw [-latex] (11.5,17.75) -- (12.5,18.75);
					\draw [-latex] (12,14.25) -- (13,15.25);
					\node [font=\normalsize] at (11.5,18.25) {$R_A$};
					\node [font=\normalsize] at (13,14.75) {$R_P$};
					\node [font=\normalsize] at (14,19.25) {$R_X$};
					\node [font=\normalsize] at (14,17.25) {$L_X$};
					\draw [-latex] (14,14.25) -- (15,15.25);
					\draw (14.5,16.5) to[R] (14.5,13);
					\node [font=\normalsize] at (14,14.75) {$R_D$};
					\draw (11.5,16) to[C] (11.5,13.5);
					\draw [-latex] (11,14.25) -- (12,15.25);
					\node [font=\normalsize] at (10.75,14.75) {$C_P$};
					\draw [](11.5,16) to[short] (12.5,16);
					\draw [](11.5,13.5) to[short] (12.5,13.5);
					\draw [](12,13) to[short] (12,13.5);
					\draw [](12,16.5) to[short] (12,16);
					\draw [](12,13) to[short] (14.5,13);
					\node at (12,16) [circ] {};
					\node at (12,13.5) [circ] {};
					\draw (14.5,18.25) to[R] (14.5,20);
					\draw[] (14.5,20) to[short] (12,20);
				\end{circuitikz}
			\end{figure}
			
			\[\vec I_A\cdot R_A = \vec I_B\cdot (R_X + j\omega L_X)\]
			\[\vec I_A\cdot (R_P || \vec X_P) = \vec I_B\cdot R_D \Rightarrow \vec I_A \cdot \dfrac{R_P}{1 + j\omega R_P C_P} = \vec I_B \cdot R_D\]
			
			\[R_X + j\omega L_X = \dfrac{R_AR_D}{\dfrac{R_P}{1 + j\omega R_P C_P}}\]
			\[R_X + j\omega L_X = \dfrac{R_AR_D(1 + j\omega R_P C_P)}{R_P}\]
			
			\[R_X = \dfrac{R_AR_D}{R_P} \qquad L_X = R_AR_DC_P\]
		
		\subsubsection{Puente de Hay.}
			Se utlilza para \textbf{medir inductancias.} El condensador patrón se considera ideal, por lo que debe ser de bajas pérdidas. En serie con él se coloca una resistencia ajustable $R_A$.
			
			
			Ofrece buenas características para la medida de bobinas con altos factores de calidad ($Q > 10$). El ajuste se simplifica si se deja fija alguna de las resistencias de los brazos. Su inconveniente es que los resultados dependen de la frecuencia, por lo que hay que conocerla.
			
			\begin{figure}[H]
				\centering
					\begin{circuitikz}
						\tikzstyle{every node}=[font=\normalsize]
						\draw (2.5,20) to[sinusoidal voltage source, sources/symbol/rotate=auto,l={ \normalsize $V_{sin}$}] (2.5,14);
						\draw (4.75,20) to[R,l={ \normalsize $R_A$}] (4.75,17);
						\draw (4.75,15.75) to[R,l={ \normalsize $R_P$}] (4.75,14);
						\draw (7,20) to[R,l={ \normalsize $R_X$}] (7,18.5);
						\draw (7,17) to[R,l={ \normalsize $R_D$}] (7,14);
						\draw (4.75,17) to[rmeter, t=V] (7,17);
						\node at (4.75,17) [circ] {};
						\node at (7,17) [circ] {};
						\node at (4.75,20) [circ] {};
						\node at (4.75,14) [circ] {};
						\draw[] (7,14) to[short] (2.5,14);
						\draw [](2.5,20) to[short] (7,20);
						\draw [-latex] (4.25,18) -- (5.25,19);
						\draw [-latex] (6.5,18.75) -- (7.5,19.75);
						\draw [-latex] (4.25,14.25) -- (5.25,15.25);
						\draw [-latex] (6.5,15) -- (7.5,16);
						\node [font=\normalsize] at (4.75,20.25) {1};
						\node [font=\normalsize] at (4.75,13.75) {2};
						\node [font=\normalsize] at (4.5,17) {3};
						\node [font=\normalsize] at (7.25,16.75) {4};
						\draw (7,18.5) to[L,l={ \normalsize $L_X$} ] (7,17);
						\draw (4.75,17) to[C,l={ \normalsize $C_P$}] (4.75,15.75);
						\draw  (1.25,17) circle (0.5cm) node {\normalsize Hz} ;
						\draw [](1.25,17.5) to[short] (1.25,20);
						\draw [](1.25,20) to[short] (2.5,20);
						\draw [](1.25,16.5) to[short] (1.25,14);
						\draw [](1.25,14) to[short] (2.5,14);
						\node at (2.5,14) [circ] {};
						\node at (2.5,20) [circ] {};
					\end{circuitikz}
			\end{figure}
			
			En el equilibrio:
			
			\[\vec I_A \cdot R_A = \vec I_B \cdot (R_X + j\omega L_X)\]
			\[\vec I_A \cdot (R_P + \dfrac{1}{j\omega C_P}) = \vec I_B \cdot R_D\]
			
			
			Luego:
			\[\dfrac{R_A}{R_P + \dfrac{1}{j\omega C_P}} = \dfrac{R_X + j\omega L_X}{R_D}\]
			\[L_X = \dfrac{C_PR_AR_D}{1+(\omega C_PR_P)^2} \qquad R_X = \dfrac{C_P^2\omega^2R_PR_AR_D}{1+(\omega C_PR_P)^2} = \omega^2 L_X C_P R_P\]
		
		\newpage
		\subsubsection{Puente de Sauty.}
			Se utiliza para \textbf{determinar capacidades de muy alto valor} ($nF\sim \mu F$). Es fundamental que las pérdidas sean muy reducidas para considerarse despreciables. Como impedancia de referencia se un condensador patrón.
			
			
			Sus resultados no son demasiado buenos, pero tiene la ventaja de que el ajuste es sencillo y rápido. 
			
			\begin{figure}[H]
				\centering
				\begin{circuitikz}
					\tikzstyle{every node}=[font=\normalsize]
					\draw (3.5,19.75) to[sinusoidal voltage source, sources/symbol/rotate=auto,l={ \normalsize $V_{sin}$}] (3.5,15.75);
					\draw (5.5,19.75) to[R,l={ \normalsize $R_A$}] (5.5,17.75);
					\draw (7.5,19.75) to[R,l={ \normalsize $R_B$}] (7.5,17.75);
					\draw (5.5,17.75) to[C,l={ \normalsize $C_P$}] (5.5,15.75);
					\draw (7.5,17.75) to[C,l={ \normalsize $C_X$}] (7.5,15.75);
					\draw (5.5,17.5) to[rmeter, t=V] (7.5,17.5);
					\node at (5.5,17.5) [circ] {};
					\node at (7.5,17.5) [circ] {};
					\node at (5.5,19.75) [circ] {};
					\node at (5.5,15.75) [circ] {};
					\draw[] (7.5,19.75) to[short] (3.5,19.75);
					\draw[] (7.5,15.75) to[short] (3.5,15.75);
					\draw [-latex] (5,18.25) -- (6,19.25);
					\draw [-latex] (7,18.25) -- (8,19.25);
				\end{circuitikz}
			\end{figure}
			
			\[\vec I_A\cdot R_A = \vec I_B\cdot R_B\]
			\[\dfrac{\vec I_A}{j\omega C_P} = \dfrac{\vec I_B}{j\omega C_X}\]
			\[\dfrac{R_A}{j\omega C_P} = \dfrac{R_B}{j\omega C_X}\]
			\[C_X = C_P\cdot \dfrac{R_A}{R_B}\]

		\newpage
		\subsubsection{Puente de Sauty-Wien.}
			Es el equivalente al puente de Maxwell para capacidades. Supone que la capacidad incógnita está formada por un condensador ideal $C_X$ en serie con una resistencia de pérdidas $R_X$. El condensador patrón debe modelarse también con resistencia de pérdidas en serie.
			
			
			Al tener en cuenta las resistencias de pérdidas el ajuste que se consigue permite obterner resultados mejores que los del puente de Sauty.
			
			
			Si se quiere determinar la tangente de pérdidas es necesario conocer la frecuencia.
			
			
			El ajuste sólo es posible si $\tan\delta_{C_P} < \tan\delta_{C_X}$
			
			\[\vec I_A \cdot R_A = \vec I_B\cdot \left(R_B + R_P + \dfrac{1}{j\omega C_P}\right)\]
			\[\vec I_A \cdot R_C = \vec I_B\cdot \left(R_X + R_P + \dfrac{1}{j\omega C_X}\right)\]
			
			Operando se obtiene:
			\[R_X = \dfrac{R_C}{R_A}(R_B+R_P) \qquad C_X = \dfrac{R_A}{R_C}C_P\]
			\[\tan \delta = \omega R_XC_X\]
			\[\tan \delta = \omega (R_B + R_P)C_X\]
		
			\begin{figure}[H]
				\centering
					\begin{circuitikz}[scale=1.1]
						\tikzstyle{every node}=[font=\normalsize]
						\draw (2.5,21) to[sinusoidal voltage source, sources/symbol/rotate=auto,l={ \normalsize $V_{sin}$}] (2.5,14);
						\draw (4.75,21) to[R,l={ \normalsize $R_A$}] (4.75,17);
						\draw (4.75,17) to[R,l={ \normalsize $R_C$}] (4.75,14);
						\draw (7,18.5) to[R,l={ \normalsize $R_P$}] (7,17);
						\draw (7,15.75) to[R,l={ \normalsize $R_X$}] (7,14);
						\draw (4.75,17) to[rmeter, t=V] (7,17);
						\node at (4.75,17) [circ] {};
						\node at (7,17) [circ] {};
						\node at (4.75,21) [circ] {};
						\node at (4.75,14) [circ] {};
						\draw[] (7,14) to[short] (2.5,14);
						\draw [](2.5,21) to[short] (7,21);
						\draw [-latex] (4.25,18.5) -- (5.25,19.5);
						\draw [-latex] (6.5,19.75) -- (7.5,20.75);
						\draw [-latex] (6.5,14.25) -- (7.5,15.25);
						\draw [-latex] (4.25,15) -- (5.25,16);
						\node [font=\normalsize] at (4.75,21.25) {1};
						\node [font=\normalsize] at (4.75,13.75) {2};
						\node [font=\normalsize] at (4.5,17) {3};
						\node [font=\normalsize] at (7.25,17) {4};
						\draw (7,17) to[C,l={ \normalsize $C_X$}] (7,15.75);
						\draw  (1.25,17.5) circle (0.5cm) node {\normalsize Hz} ;
						\draw [](1.25,18) to[short] (1.25,21);
						\draw [](1.25,21) to[short] (2.5,21);
						\draw [](1.25,17) to[short] (1.25,14);
						\draw [](1.25,14) to[short] (2.5,14);
						\node at (2.5,14) [circ] {};
						\node at (2.5,21) [circ] {};
						\draw (7,19.5) to[C,l={ \normalsize $C_P$}] (7,18.5);
						\draw (7,21) to[R,l={ \normalsize $R_B$}] (7,19.5);
						\draw [, dashed] (6.5,19.5) rectangle  (8.25,17.125);
						\draw [, dashed] (6.5,16.75) rectangle  (8.25,14.125);
					\end{circuitikz}
			\end{figure}
		
		\newpage
		\subsubsection{Puente de Schering.}
			Se utiliza para medir la \textbf{capacidad entre conductores y pantallas de cables de a.t.}, por lo que se alimenta con tensiones muy elevadas. Incorpora un transformador que eleva la tensión de la fuente de alimentación.
			
			
			El condensador patrón $C_A$ debe estar diseñado para trabajar a las altas tensiones de los cables.
			
			
			Las ramas se configuran para que la tensión del condensador patrón ajustable sea muy reducida, con el fin de evitar accidentes al operador.
			
			\[\vec I_A\cdot \left(\dfrac{1}{j\omega C_A}\right) = \vec I_B\cdot \left(R_X + \dfrac{1}{j\omega C_X}\right)\]
			\[\vec I_A\cdot \left(\dfrac{R_P}{1+j\omega C_PR_P}\right) = \vec I_B\cdot R_D\]
			\[\left(R_X + \dfrac{1}{j\omega C_X}\right)\left(\dfrac{R_P}{1+j\omega C_PR_P}\right) = R_D\cdot \dfrac{1}{j\omega C_A}\]
			\[R_X = \dfrac{C_P}{C_A}\cdot R_D \qquad C_X = \dfrac{R_P}{R_D}\cdot C_A\]
			
			\begin{figure}[H]
				\centering
					\begin{circuitikz}
						\tikzstyle{every node}=[font=\normalsize]
						\draw (1.25,16.75) to[sinusoidal voltage source, sources/symbol/rotate=auto,l={ \normalsize $V_{sin}$}] (1.25,18.25);
						\draw (2.25,18.25) to[L ] (2.25,16.75);
						\draw (3.25,16.75) to[L ] (3.25,18.25);
						\draw [short] (2.625,18) -- (2.625,17);
						\draw [short] (2.825,18) -- (2.825,17);
						\draw [](1.25,18.25) to[short] (2.25,18.25);
						\draw [](1.25,16.75) to[short] (2.25,16.75);
						\draw (4.75,17.5) to[rmeter, t=V] (8,17.5);
						\draw (4.75,17.5) to[C,l={ \normalsize $C_A$}] (4.75,21);
						\draw (8,21) to[C,l={ \normalsize $C_X$}] (8,19.25);
						\draw (8,19.25) to[R,l={ \normalsize $R_X$}] (8,17.5);
						\draw (8,17.5) to[R,l={ \normalsize $R_D$}] (8,15);
						\draw [-latex] (7.5,15.75) -- (8.5,16.75);
						\draw (4.25,15.5) to[C,l={ \normalsize $C_P$}] (4.25,17);
						\draw (5.25,17) to[R,l={ \normalsize $R_P$}] (5.25,15.5);
						\draw[] (5.25,17) to[short] (4.5,17);
						\draw[] (4.5,17) to[short] (4.25,17);
						\draw [](4.75,17.5) to[short] (4.75,17);
						\draw [](4.25,15.5) to[short] (5.25,15.5);
						\draw [](4.75,15.5) to[short] (4.75,15);
						\draw [](4.75,15) to[short] (8,15);
						\draw [-latex] (4,16) -- (4.5,16.5);
						\draw [-latex] (4.75,15.75) -- (5.75,16.75);
						\draw [](4.75,21) to[short] (8,21);
						\draw [](3.25,18.25) to[short] (3.25,21);
						\draw [](3.5,21) to[short] (4.75,21);
						\draw[] (3.5,21) to[short] (3.25,21);
						\draw [](3.25,16.75) to[short] (3.25,15);
						\draw [](3.25,15) to[short] (4.75,15);
						\node at (4.75,15) [circ] {};
						\node at (4.75,15.5) [circ] {};
						\node at (4.75,17) [circ] {};
						\node at (4.75,17.5) [circ] {};
						\node at (8,17.5) [circ] {};
						\node at (4.75,21) [circ] {};
						\draw (4.75,15) to (4.75,14.75) node[ground]{};
						\draw [, dashed] (7.5,20.75) rectangle  (9.25,17.75);
					\end{circuitikz}
			\end{figure}