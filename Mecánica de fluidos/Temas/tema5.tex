\chapter{Flujos de viscosidad dominante.}
\section{Introducción.}
Se trabaja con líquidos newtonianos donde se cumple que:
\[\rho, \mu =cte\]
\[\vec{\nabla}\cdot\vec{v}=0\]
\[\rho\dfrac{\partial \vec{v}}{\partial t}
+
\red{\underbrace{\black \rho\left(\vec{v}\cdot\vec{\nabla}\right)\vec{v}}_{\text{Inercia convectiva}}} \black
=
-
\vec{\nabla}P
+
\red{\underbrace{\black \mu\vec{\nabla}^2\vec{v} }_{\text{Esfuerzos viscosos}}} \black
+
 \vec{f}_v\]


Para conocer si el flujo es de viscosidad dominante se emplea el número de Reynolds, que como se definió en temas anteriores:
\[Re=\dfrac{\text{Orden de magnitud de la inercia convectiva}}{\text{Orden de magnitud de fuerzas viscosas}}=\dfrac{\rho v_c L_c}{\mu}\]
\begin{itemize}
	\item Si $Re\uparrow\uparrow$ efectos viscosos despreciables.
	\item Si $Re\downarrow\downarrow$ efectos viscosos dominantes.
\end{itemize}

\section{Flujos unidireccionales.}
Se habla de flujo unidireccional cuando un flujo cuya expresión de la velocidad es de la forma:
\[\vec{v}=v_x(x,y,z,t)\vec{i}+v_y(x,y,z,t)\vec{j}+v_z(x,y,z,t)\vec{k}  \leftrightarrow v_y=v_z=0\]


Como además, ha de cumplirse la conservación de la masa:
\[\vec{\nabla}\cdot\vec{v}=0=\dfrac{\partial v_x(x,y,z,t)}{\partial x} \rightarrow v_x \ne f(x)\]


Por tanto, la expresión de la velocidad:
\[\vec{v}=v_x(y,z,t)\vec{i}\]
\newpage
\begin{figure}[H]
	\centering
	\begin{circuitikz}
		\tikzstyle{every node}=[font=\normalsize]
		\draw [short] (2.25,13) -- (9.5,13);
		\draw [short] (2.25,10) -- (9.5,10);
		\draw [->, >=Stealth] (2.25,10.25) -- (2.75,10.25);
		\draw [->, >=Stealth] (2.25,10.5) -- (3,10.5);
		\draw [->, >=Stealth] (2.25,10.75) -- (3.25,10.75);
		\draw [->, >=Stealth] (2.25,11) -- (3.5,11);
		\draw [->, >=Stealth] (2.25,12.75) -- (2.75,12.75);
		\draw [->, >=Stealth] (2.25,12.5) -- (3,12.5);
		\draw [->, >=Stealth] (2.25,12.25) -- (3.25,12.25);
		\draw [->, >=Stealth] (2.25,12) -- (3.5,12);
		\draw [->, >=Stealth] (2.25,11.75) -- (3.75,11.75);
		\draw [->, >=Stealth] (2.25,11.25) -- (3.75,11.25);
		\draw [->, >=Stealth] (2.25,11.5) -- (4,11.5);
		\node [font=\normalsize] at (4.75,11.5) {$\vec{v}$};
	\end{circuitikz}
	\caption{Representación de un posible campo de velocidades.}
	\label{fig:my_label}
\end{figure}

Una vez visto la expresión de la velocidad, desarrollando los términos de la ecuación de Navier-Stokes:
\[\rho\dfrac{\partial \vec{v}}{\partial t}
+
 \rho\left(\vec{v}\cdot\vec{\nabla}\right)\vec{v}
=
-
\vec{\nabla}P
+
 \mu\vec{\nabla}^2\vec{v} 
+
\vec{f}_v\]


\begin{itemize}
	\item Como solo se estudiaran movimientos cuasiestacionarios, el término local es nulo:
	\[\rho\dfrac{\partial \vec{v}}{\partial t}=0\]
	\item Desarrollando el término de inercia local:
	\[\rho\left(\vec{v}\cdot\vec{\nabla}\right)\vec{v}= \rho \left[v_x\vec{i}\cdot 
	\left(\dfrac{\partial}{\partial x}\vec{i}+\dfrac{\partial}{\partial y}\vec{j}+\dfrac{\partial}{\partial z}\vec{k}\right)\right]v_x\vec{i}=
	\rho v_x \dfrac{\partial}{\partial x} v_x \vec{i}=0\]
	\item El término de presiones no cambia.
	\item Desarrollando el término de viscosidad (v solo tiene componente en $\vec{i}$):
	\[\mu\vec{\nabla}^2\vec{v}=\mu\left[\dfrac{\partial^2 v_x}{\partial y^2}+\dfrac{\partial^2 v_x}{\partial z^2}\right]\vec{i}\]	
	\item El término de fuerzas volumétricas se relaciona con una función potencial:
	\[\vec{f}_v=-\rho\vec{\nabla}U\]
\end{itemize}


Por tanto, la ecuación queda como:
\[0=\mu\left[\dfrac{\partial^2 v_x}{\partial y^2}+\dfrac{\partial^2 v_x}{\partial z^2}\right]\vec{i}
-
\vec{\nabla}P
-\rho\vec{\nabla}U= \mu\left[\dfrac{\partial^2 v_x}{\partial y^2}+\dfrac{\partial^2 v_x}{\partial z^2}\right]\vec{i} -\vec{\nabla}\left(P+\rho U\right)\]

Particularizando por cada eje:
\[0=\mu\left[\dfrac{\partial^2 v_x}{\partial y^2}+\dfrac{\partial^2 v_x}{\partial z^2}\right]-\dfrac{\partial}{\partial x}\left(P+\rho U\right) \rightarrow \text{Como $v_x$ no depende de x, $v_x\ne f(x)$}\rightarrow  \dfrac{\partial}{\partial x}\left(P+\rho U\right) \ne f(x)\]
\[0=-\dfrac{\partial}{\partial y}\left(P+\rho U\right) \rightarrow P+\rho U \ne f(y)\]
\[0=-\dfrac{\partial}{\partial z}\left(P+\rho U\right) \rightarrow P+\rho U \ne f(z)\]


Por tanto, teniendo en cuenta los desarrollos anteriores:
\[P+\rho U = Ax+B\]


Se define como gradiente de presión reducida a la siguiente variable:
\[P_l=-\dfrac{\partial}{\partial x}\left(P+\rho U\right) \]
\section{Flujo de Covette.}
Teniendo en cuenta el esquema de la figura \ref{fig:Covette} y desarrollando las ecuaciones anteriores en x e y:
\[P_l=-\dfrac{\partial}{\partial x}\left(P+\rho U\right)=0 \leftarrow \text{En el eje x no se producen gradientes de presión ni de fuerzas volumétricas.} \]


Por tanto:
\[0=\mu\left[\dfrac{\partial^2 v_x}{\partial y^2}+\dfrac{\partial^2 v_x}{\partial z^2}\right]-\dfrac{\partial}{\partial x}\left(P+\rho U\right) =\mu \dfrac{\partial^2 v_x}{\partial y^2} \rightarrow v_x=Ay+B\]


Como la velocidad del líquido es solidaria a la de los sólidos con los que está en contacto:
\[v_x(y=0)=0\]
\[v_x(y=h_0)=v_p\]


Sustituyendo:
\[v_x=\dfrac{v_p}{h_0}y\]
\begin{figure}[H]
	\centering
		\begin{circuitikz}
			\tikzstyle{every node}=[font=\LARGE]
			\draw [short] (4.25,9) -- (12.75,9);
			\draw [short] (4.75,9) -- (4.5,8.25);
			\draw [short] (5,8.25) -- (5.25,9);
			\draw [short] (5.5,8.25) -- (5.75,9);
			\draw [short] (6,8.25) -- (6.25,9);
			\draw [short] (6.75,9) -- (6.5,8.25);
			\draw [short] (7,8.25) -- (7.25,9);
			\draw [short] (7.5,8.25) -- (7.75,9);
			\draw [short] (8,8.25) -- (8.25,9);
			\draw [short] (8.75,9) -- (8.5,8.25);
			\draw [short] (9,8.25) -- (9.25,9);
			\draw [short] (9.5,8.25) -- (9.75,9);
			\draw [short] (10,8.25) -- (10.25,9);
			\draw [short] (10.75,9) -- (10.5,8.25);
			\draw [short] (11,8.25) -- (11.25,9);
			\draw [short] (11.5,8.25) -- (11.75,9);
			\draw [short] (12,8.25) -- (12.25,9);
			\draw [dashed] (9,9) -- (10.5,10.5);
			\draw [short] (11,10.5) -- (6.25,10.5);
			\draw [short] (6.25,10.5) -- (6.25,11);
			\draw [short] (6.25,11) -- (11.5,11);
			\draw [short] (11.5,11) -- (11.5,10.5);
			\draw [short] (11,10.5) -- (11.5,10.5);
			\draw [->, >=Stealth] (11.5,10.75) -- (12.75,10.75);
			\draw [<->, >=Stealth, dashed] (6.25,10.5) -- (6.25,9);
			\node [font=\normalsize] at (5.75,9.75) {$h_0$};
			\node [font=\normalsize] at (13,10.75) {$v_p$};
			\draw [ color={rgb,255:red,187; green,0; blue,255}, ->, >=Stealth] (4.25,9) -- (4.75,9);
			\draw [ color={rgb,255:red,187; green,0; blue,255}, ->, >=Stealth] (4.25,9) -- (4.25,9.5);
			\node [font=\normalsize, color={rgb,255:red,187; green,0; blue,255}] at (4.75,9.25) {x};
			\node [font=\normalsize, color={rgb,255:red,187; green,0; blue,255}] at (4.25,9.75) {y};
			\draw [ color={rgb,255:red,0; green,255; blue,238}, ->, >=Stealth, dashed] (9,9.25) -- (9.25,9.25);
			\draw [ color={rgb,255:red,0; green,255; blue,238}, ->, >=Stealth, dashed] (9,9.5) -- (9.5,9.5);
			\draw [ color={rgb,255:red,0; green,255; blue,238}, ->, >=Stealth, dashed] (9,9.75) -- (9.75,9.75);
			\draw [ color={rgb,255:red,0; green,255; blue,238}, ->, >=Stealth, dashed] (9,10) -- (10,10);
			\draw [ color={rgb,255:red,0; green,255; blue,238}, ->, >=Stealth, dashed] (9,10.25) -- (10.25,10.25);
			\node [font=\normalsize, color={rgb,255:red,1; green,137; blue,128}] at (8.5,9.75) {v};
		\end{circuitikz}
	\caption{Representación del flujo de Covette.}
	\label{fig:Covette}
\end{figure}
\section{Flujo de Poiseulle.}
Teniendo en cuenta el esquema de la figura \ref{fig:Poiseuille} y desarrollando las ecuaciones anteriores en x e y:
\[0=P_l+\mu \dfrac{\partial^2 v_x}{\partial y^2} \leftrightarrow P_l \ne f(y) \rightarrow \dfrac{\partial^2 v_x}{\partial y^2} = -\dfrac{P_l}{\mu}\rightarrow v_x=-\dfrac{P_l}{2\mu}y^2+Ay+B\]


Sean las condiciones de contorno:
\[v_x(y=0)=0 \rightarrow B=0 \leftrightarrow v_x(y=h_0)=0\rightarrow A=\dfrac{P_l h_0}{2\mu} \rightarrow v_x=-\dfrac{P_l y}{2\mu}\left(y-h_0\right)\]

Para casos cuasiunidireccionales se asume:
\[P_l=-\dfrac{\Delta (P+\rho U)}{L}=\dfrac{(P+\rho U)_e-(P+\rho U)_s}{L}\]
\newpage
\begin{figure}[H]
	\centering
		\begin{circuitikz}
			\tikzstyle{every node}=[font=\normalsize]
			\draw [short] (4.25,9) -- (12.75,9);
			\draw [short] (4.75,9) -- (4.5,8.25);
			\draw [short] (5,8.25) -- (5.25,9);
			\draw [short] (5.5,8.25) -- (5.75,9);
			\draw [short] (6,8.25) -- (6.25,9);
			\draw [short] (6.75,9) -- (6.5,8.25);
			\draw [short] (7,8.25) -- (7.25,9);
			\draw [short] (7.5,8.25) -- (7.75,9);
			\draw [short] (8,8.25) -- (8.25,9);
			\draw [short] (8.75,9) -- (8.5,8.25);
			\draw [short] (9,8.25) -- (9.25,9);
			\draw [short] (9.5,8.25) -- (9.75,9);
			\draw [short] (10,8.25) -- (10.25,9);
			\draw [short] (10.75,9) -- (10.5,8.25);
			\draw [short] (11,8.25) -- (11.25,9);
			\draw [short] (11.5,8.25) -- (11.75,9);
			\draw [short] (12,8.25) -- (12.25,9);
			\draw [ color={rgb,255:red,187; green,0; blue,255}, ->, >=Stealth] (4.25,9) -- (4.75,9);
			\draw [ color={rgb,255:red,187; green,0; blue,255}, ->, >=Stealth] (4.25,9) -- (4.25,9.5);
			\node [font=\normalsize, color={rgb,255:red,187; green,0; blue,255}] at (4.75,9.25) {x};
			\node [font=\normalsize, color={rgb,255:red,187; green,0; blue,255}] at (4.25,9.75) {y};
			\draw [short] (4.5,11.75) -- (12.75,11.75);
			\draw [short] (5,12.5) -- (4.75,11.75);
			\draw [short] (5.25,11.75) -- (5.5,12.5);
			\draw [short] (5.75,11.75) -- (6,12.5);
			\draw [short] (6.25,11.75) -- (6.5,12.5);
			\draw [short] (7,12.5) -- (6.75,11.75);
			\draw [short] (7.25,11.75) -- (7.5,12.5);
			\draw [short] (7.75,11.75) -- (8,12.5);
			\draw [short] (8.25,11.75) -- (8.5,12.5);
			\draw [short] (9,12.5) -- (8.75,11.75);
			\draw [short] (9.25,11.75) -- (9.5,12.5);
			\draw [short] (9.75,11.75) -- (10,12.5);
			\draw [short] (10.25,11.75) -- (10.5,12.5);
			\draw [short] (11,12.5) -- (10.75,11.75);
			\draw [short] (11.25,11.75) -- (11.5,12.5);
			\draw [short] (11.75,11.75) -- (12,12.5);
			\draw [short] (12.25,11.75) -- (12.5,12.5);
			\draw [ color={rgb,255:red,0; green,255; blue,238}, ->, >=Stealth, dashed] (5.5,11.5) -- (5.75,11.5);
			\draw [ color={rgb,255:red,0; green,255; blue,238}, ->, >=Stealth, dashed] (5.5,11.25) -- (6,11.25);
			\draw [ color={rgb,255:red,0; green,255; blue,238}, ->, >=Stealth, dashed] (5.5,11) -- (6.25,11);
			\draw [ color={rgb,255:red,0; green,255; blue,238}, ->, >=Stealth, dashed] (5.5,10.75) -- (6.5,10.75);
			\draw [ color={rgb,255:red,0; green,255; blue,238}, ->, >=Stealth, dashed] (5.5,10.5) -- (6.75,10.5);
			\draw [ color={rgb,255:red,0; green,255; blue,238}, ->, >=Stealth, dashed] (5.5,10) -- (6.5,10);
			\draw [ color={rgb,255:red,0; green,255; blue,238}, ->, >=Stealth, dashed] (5.5,9.75) -- (6.25,9.75);
			\draw [ color={rgb,255:red,0; green,255; blue,238}, ->, >=Stealth, dashed] (5.5,9.5) -- (6,9.5);
			\draw [ color={rgb,255:red,0; green,255; blue,238}, ->, >=Stealth, dashed] (5.5,9.25) -- (5.75,9.25);
			\draw [ color={rgb,255:red,0; green,255; blue,238}, ->, >=Stealth, dashed] (5.5,10.25) -- (6.75,10.25);
			\draw [dashed] (5.5,9) .. controls (7.25,10.25) and (7.25,10.5) .. (5.5,11.75);
			\draw [<->, >=Stealth, dashed] (4.5,7.75) -- (12.75,7.75);
			\draw [<->, >=Stealth, dashed] (12.75,11.75) -- (12.75,9);
			\node [font=\normalsize] at (13.25,10.25) {$ $};
			\node [font=\normalsize] at (8.5,7.25) {L};
		\end{circuitikz}
	\caption{Representación del flujo de Poiseuille.}
	\label{fig:Poiseuille}
\end{figure}
\section{Flujo de Hagen-Poiseuille.}
Teniendo en cuenta el esquema de la figura \ref{fig:Hagen-Poiseuille} que, a diferencia del flujo de Poiseulle que era bidimensional, en este caso se analiza el flujo a través de una tubería cilíndrica. Se trabaja en coordenadas cilíndricas (r,z,$\theta$) debido a que hay simetría de revolución. En este caso, desarrollando quedaría:
\[ \left\{
\begin{matrix}
	v_z\ne f(z) \\
	P_l=-\dfrac{\partial}{\partial z} (P+\rho U)\\
	0=P_l+\dfrac{\mu}{r}\left[\dfrac{\partial}{\partial r}\left(r\dfrac{\partial v_z}{\partial r}\right)\right]
\end{matrix}
\right.
\]


Integrando en estas coordenadas y teniendo en cuenta las condiciones de contorno de la tubería cilíndrica, la expresión del caudal de Hagen-Poiseuille queda:
\[Q_{H-P}=-\dfrac{\pi D^4(z)}{128\mu}\dfrac{\partial}{\partial z}(P+\rho U)\]
\begin{figure}[H]
	\centering
		\begin{circuitikz}
			\tikzstyle{every node}=[font=\normalsize]
			\draw [short] (3.5,9) -- (12.75,9);
			\draw [short] (4.75,9) -- (4.5,8.25);
			\draw [short] (5,8.25) -- (5.25,9);
			\draw [short] (5.5,8.25) -- (5.75,9);
			\draw [short] (6,8.25) -- (6.25,9);
			\draw [short] (6.75,9) -- (6.5,8.25);
			\draw [short] (7,8.25) -- (7.25,9);
			\draw [short] (7.5,8.25) -- (7.75,9);
			\draw [short] (8,8.25) -- (8.25,9);
			\draw [short] (8.75,9) -- (8.5,8.25);
			\draw [short] (9,8.25) -- (9.25,9);
			\draw [short] (9.5,8.25) -- (9.75,9);
			\draw [short] (10,8.25) -- (10.25,9);
			\draw [short] (10.75,9) -- (10.5,8.25);
			\draw [short] (11,8.25) -- (11.25,9);
			\draw [short] (11.5,8.25) -- (11.75,9);
			\draw [short] (12,8.25) -- (12.25,9);
			\draw [short] (3.5,11.75) -- (12.75,11.75);
			\draw [short] (5,12.5) -- (4.75,11.75);
			\draw [short] (5.25,11.75) -- (5.5,12.5);
			\draw [short] (5.75,11.75) -- (6,12.5);
			\draw [short] (6.25,11.75) -- (6.5,12.5);
			\draw [short] (7,12.5) -- (6.75,11.75);
			\draw [short] (7.25,11.75) -- (7.5,12.5);
			\draw [short] (7.75,11.75) -- (8,12.5);
			\draw [short] (8.25,11.75) -- (8.5,12.5);
			\draw [short] (9,12.5) -- (8.75,11.75);
			\draw [short] (9.25,11.75) -- (9.5,12.5);
			\draw [short] (9.75,11.75) -- (10,12.5);
			\draw [short] (10.25,11.75) -- (10.5,12.5);
			\draw [short] (11,12.5) -- (10.75,11.75);
			\draw [short] (11.25,11.75) -- (11.5,12.5);
			\draw [short] (11.75,11.75) -- (12,12.5);
			\draw [short] (12.25,11.75) -- (12.5,12.5);
			\draw [ color={rgb,255:red,0; green,255; blue,238}, ->, >=Stealth, dashed] (6.25,11.5) -- (6.5,11.5);
			\draw [ color={rgb,255:red,0; green,255; blue,238}, ->, >=Stealth, dashed] (6.25,11.25) -- (6.75,11.25);
			\draw [ color={rgb,255:red,0; green,255; blue,238}, ->, >=Stealth, dashed] (6.25,11) -- (7,11);
			\draw [ color={rgb,255:red,0; green,255; blue,238}, ->, >=Stealth, dashed] (6.25,10.75) -- (7.25,10.75);
			\draw [ color={rgb,255:red,0; green,255; blue,238}, ->, >=Stealth, dashed] (6.25,10.5) -- (7.5,10.5);
			\draw [ color={rgb,255:red,0; green,255; blue,238}, ->, >=Stealth, dashed] (6.25,10) -- (7.25,10);
			\draw [ color={rgb,255:red,0; green,255; blue,238}, ->, >=Stealth, dashed] (6.25,9.75) -- (7,9.75);
			\draw [ color={rgb,255:red,0; green,255; blue,238}, ->, >=Stealth, dashed] (6.25,9.5) -- (6.75,9.5);
			\draw [ color={rgb,255:red,0; green,255; blue,238}, ->, >=Stealth, dashed] (6.25,9.25) -- (6.5,9.25);
			\draw [ color={rgb,255:red,0; green,255; blue,238}, ->, >=Stealth, dashed] (6.25,10.25) -- (7.5,10.25);
			\draw [dashed] (6.25,9) .. controls (8,10.25) and (8,10.5) .. (6.25,11.75);
			\draw [<->, >=Stealth, dashed] (4.5,7.75) -- (12.75,7.75);
			\draw [<->, >=Stealth, dashed] (10.75,11.75) -- (10.75,9);
			\node [font=\normalsize] at (11.25,10.25) {D};
			\node [font=\normalsize] at (8.5,7.25) {L};
			\draw [->, >=Stealth] (1.5,10.25) -- (2.75,10.25);
			\draw [->, >=Stealth] (1.5,10.5) -- (2.75,10.5);
			\node [font=\LARGE] at (2,11) {Q};
			\draw [ color={rgb,255:red,177; green,3; blue,240}, short] (4.25,10.75) -- (4.25,9.75);
			\draw [ color={rgb,255:red,176; green,0; blue,240}, ->, >=Stealth] (4.25,10.25) -- (5.25,10.25);
			\node [font=\normalsize, color={rgb,255:red,176; green,0; blue,240}] at (5.5,10.25) {z};
			\draw [short] (12.75,11.75) .. controls (11.5,10.5) and (13.5,10) .. (12.75,9);
			\draw [short] (12.75,11.75) .. controls (13.25,11) and (13.25,10.75) .. (12.75,10);
			\node [font=\normalsize] at (13,9.75) {};
			\draw [short] (4.25,9) -- (4,8.25);
			\draw [short] (3.75,9) -- (3.5,8.25);
			\draw [short] (4.25,11.75) -- (4.5,12.5);
			\draw [short] (3.75,11.75) -- (4,12.5);
			\draw [short] (3.5,9) .. controls (4.75,10.25) and (2.75,10.75) .. (3.5,11.75);
			\draw [short] (3.5,10.75) .. controls (3,10) and (3,9.75) .. (3.5,9);
		\end{circuitikz}
	\caption{Representación del flujo de Hagen-Poiseuille.}
	\label{fig:Hagen-Poiseuille}
\end{figure}
\newpage
\section{Espesor de capa límite.}
Cuando se transiciona entre distintos regímenes hay un espesor $\delta_g$ (Ver figura \ref{fig:capa límite.}) en el cual el fluido sigue comportándose como un fluido ideal que se puede cuantificar a través del número de Reynolds. 

\[Re=\dfrac{\rho_c v^2_c/L_e}{\mu_c v_c/D^2_c} =1 \ \text{(Criterio región de transición.)}
\rightarrow \dfrac{\rho_c v^2_c}{L_e} \approx \dfrac{\mu_c v_c}{D^2_c}\]
\[Re=\dfrac{\rho_c v_c D}{\mu}\cdot\dfrac{D}{L_e}\approx 1 \rightarrow L_e \approx D\cdot Re_D\]



Por otro lado, para que se pueda asumir viscosidad dominante en la mayoría del conducto debe cumplirse que:
\[\dfrac{L_e}{L}\approx\dfrac{D}{L} Re_D << 1\]
\begin{figure}[H]
	\centering
		\begin{circuitikz}
			\tikzstyle{every node}=[font=\normalsize]
			\draw [ color={rgb,255:red,245; green,0; blue,0} ] (10.75,9) circle (2.25cm);
			\draw [short] (2,14.75) -- (2,12.25);
			\draw [short] (2,12.25) -- (4,12.25);
			\draw [short] (4,12.25) -- (4,7.5);
			\draw [short] (5,12.25) -- (5,7.5);
			\draw [short] (5,12.25) -- (7,12.25);
			\draw [short] (7,12.25) -- (7,14.75);
			\draw [ color={rgb,255:red,0; green,41; blue,245}, dashed] (2,14) -- (7,14);
			\node [font=\normalsize, color={rgb,255:red,241; green,0; blue,245}] at (4.25,14.25) {$P_a$};
			\node [font=\normalsize, color={rgb,255:red,241; green,0; blue,245}] at (4.25,13.75) {$P_1$};
			\node [font=\normalsize, color={rgb,255:red,241; green,0; blue,245}] at (4.5,13) {$P_2$};
			\node [font=\normalsize, color={rgb,255:red,241; green,0; blue,245}] at (4.5,11.5) {$P_3$};
			\node at (4.5,12) [circ, color={rgb,255:red,241; green,0; blue,245}] {};
			\node at (4.5,12.5) [circ, color={rgb,255:red,241; green,0; blue,245}] {};
			\node at (4.5,13.75) [circ, color={rgb,255:red,241; green,0; blue,245}] {};
			\node at (4.5,14.25) [circ, color={rgb,255:red,241; green,0; blue,245}] {};
			\draw [ color={rgb,255:red,245; green,0; blue,0}, ->, >=Stealth, dashed] (4.5,12.25) -- (9,10.75);
			\draw [ color={rgb,255:red,245; green,0; blue,0} ] (4.5,12.25) circle (0.5cm);
			\node [font=\normalsize] at (11,11.5) {Región de transicion};
			\draw [short] (8.75,10) -- (10,10);
			\draw [short] (10,10) -- (10,7);
			\draw [short] (11.5,7) -- (11.5,10);
			\draw [short] (11.5,10) -- (12.75,10);
			\draw [dashed] (10,8.5) .. controls (10.75,7.5) and (10.75,7.5) .. (11.5,8.5);
			\draw [ color={rgb,255:red,0; green,238; blue,255}, ->, >=Stealth, dashed] (10.25,8.5) -- (10.25,8.25);
			\draw [ color={rgb,255:red,0; green,238; blue,255}, ->, >=Stealth, dashed] (10.5,8.5) -- (10.5,8);
			\draw [ color={rgb,255:red,0; green,238; blue,255}, ->, >=Stealth, dashed] (11.25,8.5) -- (11.25,8.25);
			\draw [ color={rgb,255:red,0; green,238; blue,255}, ->, >=Stealth, dashed] (11,8.5) -- (11,8);
			\draw [ color={rgb,255:red,0; green,238; blue,255}, ->, >=Stealth, dashed] (10.75,8.5) -- (10.75,7.75);
			\draw [ color={rgb,255:red,0; green,238; blue,255}, ->, >=Stealth, dashed] (10.75,9.5) -- (10.75,9);
			\draw [ color={rgb,255:red,0; green,238; blue,255}, ->, >=Stealth, dashed] (10.5,10) -- (10.5,9.5);
			\draw [ color={rgb,255:red,0; green,238; blue,255}, ->, >=Stealth, dashed] (10.75,10) -- (10.75,9.5);
			\draw [ color={rgb,255:red,0; green,238; blue,255}, ->, >=Stealth, dashed] (11,10) -- (11,9.5);
			\draw [ color={rgb,255:red,0; green,238; blue,255}, ->, >=Stealth, dashed] (10,10.75) -- (10,10);
			\draw [ color={rgb,255:red,0; green,238; blue,255}, ->, >=Stealth, dashed] (10.25,10.75) -- (10.25,10);
			\draw [ color={rgb,255:red,0; green,238; blue,255}, ->, >=Stealth, dashed] (10.5,10.75) -- (10.5,10);
			\draw [ color={rgb,255:red,0; green,238; blue,255}, ->, >=Stealth, dashed] (10.75,10.75) -- (10.75,10);
			\draw [ color={rgb,255:red,0; green,238; blue,255}, ->, >=Stealth, dashed] (11,10.75) -- (11,10);
			\draw [ color={rgb,255:red,0; green,238; blue,255}, ->, >=Stealth, dashed] (11.25,10.75) -- (11.25,10);
			\draw [ color={rgb,255:red,0; green,238; blue,255}, ->, >=Stealth, dashed] (11.5,10.75) -- (11.5,10);
			\draw [ color={rgb,255:red,255; green,0; blue,234}, short] (10,10) -- (11.5,10);
			\draw [ color={rgb,255:red,255; green,0; blue,234}, short] (10.25,9.5) -- (11.25,9.5);
			\draw [<->, >=Stealth, dashed] (3.5,12.25) -- (3.5,7.5);
			\draw [<->, >=Stealth, dashed] (7.25,12.25) -- (7.25,14);
			\draw [<->, >=Stealth, dashed] (2,13.25) -- (7,13.25);
			\node [font=\normalsize] at (7.75,13) {h(t)};
			\node [font=\normalsize] at (5.5,13.5) {$A_d$};
			\node [font=\normalsize] at (3.25,10) {L};
			\draw [<->, >=Stealth, dashed] (4,8.75) -- (5,8.75);
			\node [font=\normalsize] at (4.5,9) {D};
			\node [font=\normalsize] at (12,9.5) {$L_e$};
			\draw [ color={rgb,255:red,255; green,0; blue,234}, ->, >=Stealth, dashed] (9.25,9) -- (10.25,9.5);
			\node [font=\normalsize, color={rgb,255:red,255; green,0; blue,234}] at (9,9) {$\delta_g$};
			\draw [dashed] (11.5,10) .. controls (10.75,8.25) and (10.75,8.25) .. (10,10);
			\draw [ color={rgb,255:red,0; green,238; blue,255}, ->, >=Stealth, dashed] (11.25,10) -- (11.25,9.5);
			\draw [ color={rgb,255:red,0; green,238; blue,255}, ->, >=Stealth, dashed] (10.25,10) -- (10.25,9.5);
			\draw [<->, >=Stealth, dashed] (11.75,10) -- (11.75,8.75);
			\draw [ color={rgb,255:red,0; green,238; blue,255}, ->, >=Stealth, dashed] (11,9.5) -- (11,9);
			\draw [ color={rgb,255:red,0; green,238; blue,255}, ->, >=Stealth, dashed] (10.5,9.5) -- (10.5,9);
		\end{circuitikz}
	\caption{Representación del espesor de capa límite.}
\label{fig:capa límite.}
\end{figure}