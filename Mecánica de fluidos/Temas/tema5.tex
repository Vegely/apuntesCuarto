\chapter{Flujos de viscosidad dominante.}
\section{Introducción.}
Se trabaja con líquidos newtonianos donde se cumple que:
\[\rho, \mu =cte\]
\[\vec{\nabla}\cdot\vec{v}=0\]
\[\rho\dfrac{\partial \vec{v}}{\partial t}
+
\red{\underbrace{\black \rho\left(\vec{v}\cdot\vec{\nabla}\right)\vec{v}}_{\text{Inercia convectiva}}} \black
=
-
\vec{\nabla}P
+
\red{\underbrace{\black \mu\vec{\nabla}^2\vec{v} }_{\text{Esfuerzos viscosos}}} \black
+
 \vec{f}_v\]


Para conocer si el flujo es de viscosidad dominante se emplea el número de Reynolds, que como se definió en temas anteriores:
\[Re=\dfrac{\text{Orden de magnitud de la inercia convectiva}}{\text{Orden de magnitud de fuerzas viscosas}}=\dfrac{\rho v_c L_c}{\mu}\]
\begin{itemize}
	\item Si $Re\uparrow\uparrow$ efectos viscosos despreciables.
	\item Si $Re\downarrow\downarrow$ efectos viscosos dominantes.
\end{itemize}

\section{Flujos unidireccionales.}
Se habla de flujo unidireccional cuando un flujo cuya expresión de la velocidad es de la forma:
\[\vec{v}=v_x(x,y,z,t)\vec{i}+v_y(x,y,z,t)\vec{j}+v_z(x,y,z,t)\vec{k}  \leftrightarrow v_y=v_z=0\]


Como además, ha de cumplirse la conservación de la masa:
\[\vec{\nabla}\cdot\vec{v}=0=\dfrac{\partial v_x(x,y,z,t)}{\partial x} \rightarrow v_x \ne f(x)\]


Por tanto, la expresión de la velocidad:
\[\vec{v}=v_x(y,z,t)\vec{i}\]

\begin{figure}[!ht]
	\centering
	\begin{circuitikz}
		\tikzstyle{every node}=[font=\normalsize]
		\draw [short] (2.25,13) -- (9.5,13);
		\draw [short] (2.25,10) -- (9.5,10);
		\draw [->, >=Stealth] (2.25,10.25) -- (2.75,10.25);
		\draw [->, >=Stealth] (2.25,10.5) -- (3,10.5);
		\draw [->, >=Stealth] (2.25,10.75) -- (3.25,10.75);
		\draw [->, >=Stealth] (2.25,11) -- (3.5,11);
		\draw [->, >=Stealth] (2.25,12.75) -- (2.75,12.75);
		\draw [->, >=Stealth] (2.25,12.5) -- (3,12.5);
		\draw [->, >=Stealth] (2.25,12.25) -- (3.25,12.25);
		\draw [->, >=Stealth] (2.25,12) -- (3.5,12);
		\draw [->, >=Stealth] (2.25,11.75) -- (3.75,11.75);
		\draw [->, >=Stealth] (2.25,11.25) -- (3.75,11.25);
		\draw [->, >=Stealth] (2.25,11.5) -- (4,11.5);
		\node [font=\normalsize] at (4.75,11.5) {$\vec{v}$};
	\end{circuitikz}
	\label{fig:my_label}
\end{figure}

\section{Flujo de Covette.}
\section{Flujo de Poiseulle.}
\section{Flujo de Hagen-Poiseuille.}
\section{Espesor de capa límite.}